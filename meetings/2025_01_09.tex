\documentclass[11pt]{article}
\usepackage{../template}

\graphicspath{ {./img/} }
\addbibresource{../references.bib}

\begin{document}

\begin{enumerate}
    \item T-Type channels
    \begin{enumerate}
        \item First order kinetics
        \begin{itemize}
            \item We need data for deinactivation (no idea about the function to fit). Slow time
            constant is scaled version of $\tau_{deinactivation}$
            \item First order kinetics might be sufficient for our purposes
        \end{itemize}
        
        \item Fitting Jeong 2015
        \begin{itemize}
            \item (2.2.1) Fitting steady state inactivation/activation: I doubt that they scaled
            activation, so I assumed, that they plotted $m^3$ for activation
            \item (2.2.2) $m^3$ dynamics for activation is also indicated by the time constant
            of tail currents (apart from similar models in the literature). Time constant for
            one gate will be three times slower, as the deactivation will occur if any of the
            three channels is deinactivated. Three times tau deactivation at -50mV is approximately
            equal to tau activation at the same test potential.
            \item Took values for deinactivation from another paper. Currently modelled as the
            discontinuous jump at -50mV. But this might be not very good approach
            (will be important when I move to the neuronal model)
        \end{itemize}

        \item Current
        \begin{itemize}
            \item I-V relationship should be important to have biologically plausible model.
            Especially for the T-Type current, as the bursts lay on top of them
            \item Ohmic current cannot reproduce I-V relationship correclty. Switching from Ohmic
            current to Constant field equation did better job
            \item This was described in one of the papers. Was also true iin our case
            \item I assumed that I-V in Jeong 2015 is transient I-V (not steady sate). It was not
            explicitely mentioned what they plotted, but from the text it seemed so.
        \end{itemize}

        \item Need to scale gating functions
        \begin{itemize}
            \item Patch clamp in Jeong was done with 10mM Ca$^{2+}$ concentration. In drosophila
            $[Ca]_{o}$ is $0.5mM$
            \item The gating variables depend on the Ca$^{2+}$ concentration outside membrane
            \item Here I only fit $m_\infty$, $\tau_m$ (using function 'curve\_fit' to fit the
            IV relationship). Generally, other variables might need shift, but I am not sure whther
            it is a good idea to use 'curve\_fit' instead of correcting them based on the
            literature.
        \end{itemize}
    \end{enumerate}


    \item Model
    \begin{enumerate}
        \item Wang 1994. Idea - start simple $\rightarrow$ make more complex: What are the main components?
        \begin{itemize}
            \item Resting membrane potential is at -60 mV (instead of -49mV as in R5)
            \item h-current is not necessary for LTS (generally, removal of h current
            did not change much in the presented set of parameters)
            \item Block Na channels results in LTS, but amplitude is <-40mV
            \item Authors specifically set persistent sodium to drive postinhibitory reboud
            spike (Na spike). Without it neuron will not reach spiking threshold
            \item If one does not block potassium channels along with sodium ones, HH-type Ik takes over
            LTS and Ca spike is not visible.
        \end{itemize}
    \end{enumerate}

\end{enumerate}


\end{document}