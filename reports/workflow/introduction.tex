\documentclass[./workflow.tex]{subfiles}
\graphicspath{{\subfix{../img/}}}

\begin{document}
    \section{Introduction}

    \begin{itemize}
        \item We predict that $I_T$ should provide the inward current for the generation of low-threshold
        $Ca^{2+}$ spikes, with rapidly activating and inactivating $K^+$ current $I_A$ modulating the
        initial components of these $Ca^{2+}$ spikes. In contrast, the slower kinetics of activation
        and inactivation of the $K^+$ current $I_{K2}$ suggest that this current may affect more the
        later portions of low-threshold $Ca^{2+}$ spikes. The properties of $I_h$ suggest that it is 
        critical to modulation of the voltage-time course of the cell at hyperpolarized membrane potentials and
        may provide a "pacemaker" potential for rhythmic burst generation
        \parencite{huguenardSimulationCurrentsInvolved1992}.

        \item these currents include $I_T$, the low-threshold, transient $Ca^{2+}$ current that underlines
        burst firing; the rapidly inactivating and transient $K^{+}$ current, $I_A$, which may fascilitate
        slow repetitive firing and may interact with $I_T$ during $Ca^{2+}$-dependent burst firing;
        a slowly inactivating transient $K^+$ current, $I_{K2}$, which may control repetitive firing rate;
        and a hyperpolarization-activated, mixed cationic conductance with slow kinetics, $I_h$, which activates
        on hyperpolarization and which may generate a "pacemaker" potential for the generation of slow
        oscillations \parencite{huguenardSimulationCurrentsInvolved1992}.

        \item \textcolor{red}{The modeled currents were derived wither from Drosophila neurons (specify),
        or (e.g. rat thalamic neurons, etc)...}

        \item \textcolor{red}{We assumed that $I_T$ was composed of a uniform populatio of channels
        whose inactivation could be ompletely described by the Boltzmann function.}

        \item Current-Voltage relationship was better reproduced when constant-field equation was
        used instead of the ohmic one \parencite{huguenardSimulationCurrentsInvolved1992}.

        \item T-Type: $m^3h$ format \parencite{wangModelTtypeCalcium1991}
        
        \item Model: slow deinactivation \parencite{wangModelTtypeCalcium1991}
        
        \item T-Type Ca current together with the leakage current suffices to describe the low-threshold
        spike (LTS)... Outward currents are not required to reproduce the basic shape of the LTS\dots
        Each LTS trigerres a burst of fast action potentials that ride on its crest. As such,
        LTS plays a critical role in linking synaptic input to intrinsic membrane mechanisms of bursting in
        the relay cell and in supporting the slow membrane oscillations underlying the spondling rhythm
        \parencite{wangModelTtypeCalcium1991}.

        \item $I_T$ is de-inactivated by hyperpolarization, thus providing an ionic basis for the so-called
        post-inhibitory rebound excitation \parencite{wangMultipleDynamicalModes1994}.

        \item This "3 Hz" bursting is primarily due to the interplay between a T-type calcium current
        $I_T$, and a non-specific cation "sag" current $I_h$ which has much slower kinetics thatn $I_T$
        and is activated by hyperpolarization \parencite{wangMultipleDynamicalModes1994}.
        
        \item Physiologists have been interested in the action of calcium on excitable tissues since
        the days of Ringer (1883). Some of the main facts established (see Brink, 1954) are that
        increasing the external calcium concentration raises the threshold, increases membrane
        resistance (Cole, 1949) and accelerates accommodation. Reducing the calcium concentration
        has the converse effects, and frequently leads to spontaneous oscillations or
        repetitive activity (e.g. Adrian \& Gelfan, 1933; Arvanitaki, 1939). Other observations
        which may be less well known are that removal of calcium reduces rectification
        (Loligo nerve, Steinbach, Spiegelman \& Kawata, 1944) and increases the fraction of the
        sodium carrying system which is in a refractory or inactive condition
        (Purkinje fibres, Weidmann, 1955). In connexion with the last observation, it is interesting
        that tissues which do not normally give an anode break response can be made to do so by
        reducing the concentration of calcium ions in the external medium
        (Frankenhaeuser, 1957) \parencite{frankenhaeuserActionCalciumElectrical1957}.
    \end{itemize}
\end{document}