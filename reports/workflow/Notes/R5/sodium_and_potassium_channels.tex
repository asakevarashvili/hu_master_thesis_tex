\documentclass[../../workflow.tex]{subfiles}
\graphicspath{{\subfix{../../img/}}}

\begin{document}

\section{Sodium and Potassium Channels}

\subsection{Sodium Channels}

\begin{itemize}
    \item Drosophila genome has two genes predicted to encode $Na_v$ proteins: para, and
    60E (NaCP60E). However, NaCP60E null mutants showed no loss in inward sodium current,
    suggesting that para is the putative $Na_v$ channel \parencite{ravenscroftDrosophilaVoltageGatedSodium2020}.
    \item para has 60 predicted isoforms via alternative splicing \parencite{ravenscroftDrosophilaVoltageGatedSodium2020}
    \item Para is only expressed in active neurons \parencite{ravenscroftDrosophilaVoltageGatedSodium2020}
    \item Currently I chose vales from \parencite{warmkeFunctionalExpressionDrosophila1997} where
    $DmNa_v$ was expressed in Xenopus ooscyte (frog). The similar was done in
    \parencite{linAlternativeSplicingVoltageGated2009}, but the values are different.
    \item "heterologous expression indicates that the inclusion of specific exons imparts
    characteristic gating properties to individual splice variants of $DmNa_v$ channels.
    These differences likely contribute to, and may even explain, the diversity in action
    potential firing between different neurons observed in the Drosophila CNS"
    \item \parencite{ravenscroftDrosophilaVoltageGatedSodium2020} stated in Fig. 5 that
    4th power exponential fit for activation resutled in time constant equal to 2.1.
    This means, that for one gate it should be 8.4.
\end{itemize}


The model and parameters for the transient and persistent Na$^+$ channels,
as well as fast and slow K$^+$ currents were taken from
\parencite{gunayDistalSpikeInitiation2015}. All currents are modelled according to Ohm's law:
\begin{equation*}
    I_{c}(V,m,h) = g_{c} m_{c}^{p_c}(V) h_{c}(V) (V - V_{c})
\end{equation*}
where $c$ denotes the current type ($c \in {NaT,NaP,Kf,Ks}$), where $T$, $P$, $f$, and $s$
stand for 'Transient', 'Persistent', 'fast', and 'slow' correspondingly, $p$ is the number of activation gates.

Activation and inactivation variables are governed by ($x \in {m,h}$):
\begin{equation*}
    \frac{dx}{dt}(V) = \frac{x_{c,\infty}(V) - x}{\tau_{c}(V)}
\end{equation*}
where the steady states $x_{c,\infty}$ are defined as:
\begin{equation*}
    x_{c,\infty}(V) = \frac{1}{1 + \exp{([V - V_{c,x,1/2}]/k_{c,x,1/2})}}
\end{equation*}

The parameters are given in Table \ref{table_parameters_r5_sodium_and_potassium_channels}.

\begin{table}[h!]
    \centering
    \resizebox{\textwidth}{!}{
        \begin{tabular}{@{}lccccccc@{}}
        \toprule
        & \multicolumn{4}{c}{\textbf{Activation}} & \multicolumn{3}{c}{\textbf{Inactivation}} \\
        \cmidrule(lr){2-5}
        \cmidrule(lr){6-8}
        \textbf{Current} & $p$ & $V_{1/2,m}$ (mV) & $k_m$ (mV) & $\tau_m$ (ms) & $V_{1/2,h}$ (mV) & $k_h$ (mV) & $\tau_h$ (ms) \\ \midrule
        NaT & 3 & $-29.13$ & $-8.92$ & $0.13 + \frac{3.43}{1+\exp{((V + 45.35)/5.98)}}$ & $-47$ & 5 & $0.36 + \exp{(\frac{V+20.65}{-10.47})}$ \\
        NaP & 1 & $-48.77$ & $-3.68$ & 1 & -- & -- & -- \\
        Ks & 4 & $-12.85$ & $-19.91$ & $2.03 + \frac{1.96}{1 + \exp{( (V - 29.83)/3.32 )}}$ & - & - & - \\
        Kf & 4 & $-17.55$ & $-7.27$ & $1.94 + \frac{2.66}{1 + \exp{( (V - 8.12)/7.96 )}}$ & $-45$ & $6$ & $1.79 + \frac{515.8}{1 + \exp{( (V + 147.4)/28.66 )}}$ \\
        \bottomrule
        \end{tabular}
    }
    \caption{Channel parameters for activation and inactivation of sodium and potassium channels.}
    \label{table_parameters_r5_sodium_and_potassium_channels}
\end{table}

\begin{note}
    Similar parameters (small differences) is also given in \parencite{linActivityDependentAlternativeSplicing2012}.
\end{note}

\begin{itemize}
    \item K Reversal potential
    \begin{itemize}
        \item \parencite{gunayDistalSpikeInitiation2015}: -80mV
        \item \parencite{qiuCollectiveDynamicsNeural2021}: -90mV
    \end{itemize}
\end{itemize}

\end{document}