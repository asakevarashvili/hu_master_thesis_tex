\documentclass[../workflow.tex]{subfiles}
\graphicspath{{\subfix{../img/}}}

\begin{document}

\section{Literature}
\etocignoretoctocdepth % of course, if we want to see something in local TOC...
\etocsettocstyle{\subsection*{\contentsname}}{}
\localtableofcontents

\subsection{Papers to Read}
\begin{itemize}
   
    
    \item A Model of Spindle Rhythmicity in the Isolated Thalamic Reticular Nucleus
    \begin{itemize}
        \item \url{https://www.fuw.edu.pl/~suffa/Modelowanie/Zaliczenie/Destexhe_Isolated_RE_JNeurophysiol_1994.pdf}
        \item Model containing Ca2+ dependent channels
        \item T-Type current: model similar to what I had in mind
        \item Maybe they have similar publications on such models, but single neurons???
    \end{itemize}

    \item Ionic Mechanisms for Intrinsic Slow Oscillations in Thalamic Relay Neurons
    \begin{itemize}
        \item \url{https://www.cell.com/biophysj/pdf/S0006-3495(93)81190-1.pdf}
        \item Bursting with Ca2+ dependent channels
        \item Full Wang 1991 model for T-Type Ca2+ channels
    \end{itemize}

    \item Synthesis of Models for Excitable Membranes, Synaptic Transmission and Neuromodulation Using a Common Kinetic Formalism
    \begin{itemize}
        \item Interesting to read: different models for ion channels
        \item Analytic expressions in Appendix
        \item Reference to Rall 1967: Introduced alpha function
    \end{itemize}

    \item Astroglial Calcium Signaling Encodes Sleep Need in Drosophila
    \begin{itemize}
        \item \url{https://www.cell.com/current-biology/pdf/S0960-9822(20)31516-5.pdf}
        \item Modulation of sleep need in R5 in Ca2+ dependent manner through astrocite and Toll receptors on R5 neurons
    \end{itemize}

    \item Signal Propagation in Drosophila Central Neurons
    \begin{itemize}
        \item https://pmc.ncbi.nlm.nih.gov/articles/PMC2709801/pdf/zns6239.pdf
        \item May be important for parameters
    \end{itemize}

    \item A biophysical exploration of the motion vision pathway in Drosophila
    \begin{itemize}
        \item \url{https://edoc.ub.uni-muenchen.de/30141/1/Malis_Jonatan.pdf}
        \item Interesting for an overview of Drosophila vision pathway
    \end{itemize}

    \item Integration of sleep homeostasis and navigation in Drosophila
    \begin{itemize}
        \item \url{https://pure.mpg.de/rest/items/item_3330717_6/component/file_3335935/content}
        \item IMPORTANT FOR YOU!!!!!!!
    \end{itemize}

    \item Circadian- and Light-Dependent Regulation of Resting Membrane Potentialand Spontaneous Action Potential Firing of Drosophila CircadianPacemaker Neurons
    \begin{itemize}
        \item \url{https://journals.physiology.org/doi/epdf/10.1152/jn.00930.2007}
        \item Might be important!!!
    \end{itemize}

    \item Fast Calcium Signals in Drosophila Motor Neuron Terminals
    \begin{itemize}
        \item \url{https://journals.physiology.org/doi/epdf/10.1152/jn.00515.2002}
        \item We provide an estimate for theresting intracellular calcium concentration
        \item the first description of calciumkinetics for a single action potential (AP)
    \end{itemize}

    \item Extracellular Ca2+ Modulates the Effects of Protons on Gating and Conduction Properties of the T-type Ca2+ Channel $\alpha$1G (CaV3.1)
    \begin{itemize}
        \item \url{https://rupress.org/jgp/article-abstract/121/6/511/44434/Extracellular-Ca2-Modulates-the-Effects-of-Protons?redirectedFrom=fulltext&utm_source=chatgpt.com}
        \item Ca concentration outside and T-Type channel
    \end{itemize}

    \item Permeation and Gating in Ca V3.1 ($\alpha$1G) T-type Calcium Channels Effects of Ca 2+, Ba 2+, Mg 2+, and Na +
    \begin{itemize}
        \item \url{https://rupress.org/jgp/article-pdf/132/2/223/1785870/jgp_200809986.pdf}
        \item \item Ca concentration outside and T-Type channel
    \end{itemize}

    \item Single or multiple synchronization transitions in scale-free neuronal networks with electrical or chemical coupling
    \begin{itemize}
        \item Model to connect HH neurons via synaptic input/gap junctions
    \end{itemize}

    \item Model design for networks of heterogeneous Hodgkin–Huxley neurons
    \begin{itemize}
        \item Model to connect HH neurons via synaptic input/gap junctions
        \item \url{https://www.sciencedirect.com/science/article/pii/S0925231222005148?ref=pdf_download&fr=RR-2&rr=902dd1ee0c15e532}
    \end{itemize}

    \item Recurrent Circuitry for Balancing Sleep Need and Sleep
    \begin{itemize}
        \item \cite{donleaRecurrentCircuitryBalancing2018}: "R2 neuron activity generates sleep pressure that is communicated
        to dFB neurons via currently unidentified synaptic connections or non-synaptic mechanisms".
    \end{itemize}
    
    \item Sleep state switching: mutual inhibition between wake- and sleep-promoting
    inhibitory nuclei \item \cite{liuTwoDopaminergicNeurons2012,saperSleepStateSwitching2010}
    
    \item Circadian Rhythms and Sleep in Drosophila melanogaster \item \cite{dubowyCircadianRhythmsSleep2017}

\end{itemize}


\subsection{Saved papers}

\begin{itemize}
    \item Ca(v)2 channels mediate low and high voltage-activated calcium currents in Drosophila motoneurons
    \begin{itemize}
        \item \url{https://pubmed.ncbi.nlm.nih.gov/22183725/}
    \end{itemize}

    \item Properties and possible function of a hyperpolarisation-activated chloride current
    in Drosophila
    \begin{itemize}
        \item \url{https://cob.silverchair-cdn.com/cob/content_public/journal/jeb/210/14/10.1242_jeb.006361/3/2489.pdf?Expires=1739751548&Signature=KHX~uoNNKTFcg9bq0dkC2SXViuWXAT~hD~CuFoy54p4-jXcd65UDFvbpeBBWRIf1syQFm1mbfuwq55R0lcWyD50dysp-DIy2od7O-eRV8KgfDN4-ecenVLyf89PATOW54v3LwN96WV8dwN0x9ovbFCosBjMzgpwMoFnD6DGWCO7AVCdr9HGMt14zZQgNYlKB1XWZVlhlTy8kIDlDV6egOYwc-4OUpcN870TXGEjTvFyUqxP5dmRlKt4-3PienpiR~u1flconu1Xs3seAEr8b5JFQd8sMCmGFaGEKRoqbmac6T8KVZtxZmxTEAFMQyoAJ~vVjivJWIMSJ4ORoeGJ2Qg__&Key-Pair-Id=APKAIE5G5CRDK6RD3PGA}
    \end{itemize}

    \item Drosophila cacophony Channels: A Major Mediator of
    Neuronal Ca2+ Currents and a Trigger for K+ Channel
    Homeostatic Regulation
    \begin{itemize}
        \item \url{https://pmc.ncbi.nlm.nih.gov/articles/PMC6673189/pdf/zns1072.pdf}
    \end{itemize}

    \item

    
\end{itemize}


\newpage
\subsection{Notes on Papers}
%%%%%%%%%%%%%%%%%%%%%%%%%%%%%
%             ?             %
%%%%%%%%%%%%%%%%%%%%%%%%%%%%%
\begin{itemize}
    \item \textbf{(Raccuglia et al (submitted)) \textcolor{red}{(TODO: Add to references))}. Network
    synchrony creates neural filters that switch brain state from navigation to sleep in Drosophila.}

    \begin{itemize}
        \item Maniscript provided by Prof. Kempter
        
        \item Sleep in Drosophila, R5 and Helicon cell activity during sleep/wakefullness, modulation
        of the activity and it's correlation to behavior, synchronization of neuronal networks
        
        \item Results suggest that SWA could represent a neural filtering mechanism that regulates sensory processing
        and behavioral reponsiveness.

        \item In the morning setting, the observed networks (dFSB-helicon-R5) act independent of each
        other, allowing gating of locomotion and updating of the head direction system (Fig. 6e). In
        the night setting, circadian and homeostatic regulation promote the entrainment of
        synchronized SWA between networks that opposingly regulate behavioral responses to visual
        stimuli.

    \end{itemize}
\end{itemize}

\begin{center}
    
    
    
\end{center}



%%%%%%%%%%%%%%%%%%%%%%%%%%%%%
%             R             %
%%%%%%%%%%%%%%%%%%%%%%%%%%%%%


\end{document}