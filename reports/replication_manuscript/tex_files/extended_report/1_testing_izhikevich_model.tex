\documentclass[../../extended_report.tex]{subfiles}
\graphicspath{{\subfix{../img/}}}

\begin{document}

\section{Testing Izhikevich Model}

\subsection{Normalizing $\sigma$ of the External Current}\label{remark_wiener_process}

Generally, if one changes the simulation step size, one should adapt the $\sigma$ of the Gaussian white noise. The reason behind this is, that the variance of the Wiener process (or, Brownian motion) is:
\begin{equation*}
    var(W_{\Delta t})=\Delta t
\end{equation*}
where $\Delta W(t) \sim N(0, \Delta t)$
For the scaled Wiener process ($X = \sigma W$):
\begin{equation*}
    var(X_{\Delta t}) = var(\sigma W_{\Delta t}) = \sigma^2 var(W_{\Delta t}) = \sigma^2 \Delta t
\end{equation*}
Now, let's $\Delta t_1$ be the step size and $\sigma_1$ be the standard deviation. Variance within the time unit equal to $\Delta t_1$ will be:
\begin{equation*}
    var(X_{\Delta t_1}) = \sigma_1^2 \Delta t_1
\end{equation*}
If we change the step size to $\Delta t_2 = k \Delta t_1$, then the variance within the time unit $\Delta t_1$ will equal to:
\begin{todo}
    Finish Remark 1
\end{todo}
\qed

\begin{remark}\label{remark-normalizing_std_external_current}
    For the simulations, the standard deviation was normalized with regard to the simulation step size ONLY for helicon cells.
    \begin{todo}
        Write the reasoning and reference the pictures
    \end{todo}
\end{remark}


\end{document}