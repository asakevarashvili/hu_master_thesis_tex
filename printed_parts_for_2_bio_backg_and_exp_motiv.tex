\color{red}

\textbf{dFSB}

During course of the day activity-dependent build-up of ROS in dFSB neurons increases their excitability (Kempf et al 2019),
leading to increased inhibition of the helicon cells during the night (Donela et al 2018). Helicon cells process
light cues to induce locomotion and their inhibition therefore reduces the flow of visual information and
suppress locomotion (Donela et al 2018). Here, R5 ring neurons, which have reciprocal synaptic connections to the
helicon cells provide another layer of homeostatic sleep regulation that is directly linked to processing
compex visual and orientation behavior \parencite{suarez-grimaltNeuralArchitectureSleep2021}
(Suarez et al 2021)

dFB neurons in sleep-deprived flies tend to be electrically active, with high input resistances and long membrane time constants, while neurons in rested flies tend to be  electrically silent3.
\parencite{pimentelOperationHomeostaticSleep2016} (Pimentel et al 2016)

Helicon Cells: Targets of dFB Neurons with Projections to the Ellipsoid Body
\parencite{donleaRecurrentCircuitryBalancing2018} (Donlea et al 2018)

dFB Neurons Inhibit Helicon Cells and Their Visual Responses
\parencite{donleaRecurrentCircuitryBalancing2018} (Donlea et al 2018)

optogenetically silencing the dFSB reduced oscillatory
power of the R5 network (Extended Data Fig. 2k,l), further supporting the notion that
increased activity within the neurons of the dFSB facilitates SWA in R5. 
\parencite{raccugliaCoherentMultilevelNetwork2022} (Manuscript)

At night, the dFSB increases activity to generate compound
oscillations (Extended Data Fig. 1c). Importantly, these in turn switch on coherent oscillatory
activity in the R5 network (Fig. 1).
\parencite{raccugliaCoherentMultilevelNetwork2022} (Manuscript)

Consistent with previous
observations showing that dFSB neurons inhibit visually evoked responses in helicon cells5, we
found that helicon cells are also connected to the dFSB
\parencite{raccugliaCoherentMultilevelNetwork2022} (Manuscript)


\noindent\hrulefill

\textbf{Helicon cells}

neurons — helicon cells — that are activated by visual stimuli.
\parencite{shaferRegulationDrosophilaSleep2021} (Shafer and Kenee 2021)

Helicon Cells Gate Locomotion
\parencite{donleaRecurrentCircuitryBalancing2018} (Donlea et al 2018)

Helicon Cells Excite R2 Ring Neurons
\parencite{donleaRecurrentCircuitryBalancing2018} (Donlea et al 2018)

Helicon Cell Activation Induces Rebound Sleep
\parencite{donleaRecurrentCircuitryBalancing2018} (Donlea et al 2018)

These results demonstrate a sleep-promoting effect of inhibiting helicon cells,
but they also suggest that helicon cells are only one of several dFB outputs
used to induce sleep.
\parencite{donleaRecurrentCircuitryBalancing2018} (Donlea et al 2018)

Thus, R5 oscillatory activity is likely regulated via a complex interplay of
sensory input [20], circadian rhythms [15, 17], and homeostatic sleep pressure regulation.
\parencite{raccugliaNetworkSpecificSynchronizationElectrical2019} (Raccuglia et al 2019)

Helicon cells provide excitation to R2 neurons of the ellipsoid body, whose
activity-dependent plasticity signals rising sleep pressure to the dFB.
\parencite{donleaRecurrentCircuitryBalancing2018} (Donlea et al 2018)

TuBu neurons provide parallel synaptic input to helicon cells and R5 neurons (Hulse et al 2020).
While R5 neurons might use this visual information for navigation, light information in helicon cells might be
decisive whether locomotion is induced or not. Interacions between the helicon cells and R5
neurons could therefore integrate use-dependent sleep need, the presence of absence of light and the circadian time.
\parencite{suarez-grimaltNeuralArchitectureSleep2021}
(Suarez et al 2021)

\noindent\hrulefill


\textbf{R5}
Thus, R5 oscillatory activity is likely regulated via a complex interplay of
sensory input [20], circadian rhythms [15, 17], and homeostatic sleep pressure regulation.
\parencite{raccugliaNetworkSpecificSynchronizationElectrical2019} (Raccuglia et al 2019)

Two subtypes of EB ring neurons, that is R5 and ring\_B were affected by either
processes in opposing ways, in accordance with previous findings, showing that
R2/R4m neurons (probably part of ring\_B) received circadian timing
information from clock neurons, while
\textbf{R5 neurons themselves encoded the sleep homeostat (Liu et al 2016)}.
(Dopp et al 2024)


\noindent\hrulefill



\textbf{Other cell subtypes}

The sLNvs and LNds appear to communicate to R5 EB neurons through an intermediate set of dopaminergic PPM3 neurons based largely on correlated calcium oscillations (Liang et al., 2019)
\parencite{andreaniCircadianProgrammingEllipsoid2022} (Andreani et al 2022)

Mushroom body - The mushroom bodies, a higher-order brain
network curcially involved in olfactory memore in Drosophila, have
also been implicated in sleep regulation. Some mushroom body neurons have been
shown to be wake-promoting, while others seem sleep-promoting \parencite{suarez-grimaltNeuralArchitectureSleep2021}
(Suarez et al 2021)

mushroom body contains both sleep-promoting and wake-promoting cells
\parencite{dubowyCircadianRhythmsSleep2017} (Dubowy and Sehgal 2017)

Cell types involved in sleep: five glial cell subtypes, Kenyon cells (KCs), clock neurons and cell types containing
known sleep/wakefullness regulating curcuits such as non-protocerebral anterior medial (PAM) dopaminergic neurons (DANs),
tyraminergic (Tyr) and octopaminergic (Oct) neurons, and ellipsoid body (EB) ring neurons. Another tell type involved in sleep,
the dorsal fan-shaped bidy (dFB) neuron... \parencite{doppSinglecellTranscriptomicsReveals2024}
(Dopp et al 2024)

Using genetic tools recent study reported neuronal clusters whose gene expression was
correlated with the level of sleep drive
\parencite{doppSinglecellTranscriptomicsReveals2024}. Four clusters (\gls{dfb}, \gls{oct}, \gls{tyr}, and non-\gls{pam} \glspl{dan}) with the highest number of
correlates (i.e. genes whose expression correlated with the sleep drive) were those associated
with sleep homeostasis. \gls{eb} R5 neurons, which have been suggested to encode sleep homeostat \parencite{liuSleepDriveEncoded2016}, also showed high number
of sleep drive correlates.

Sleep need and subjectie tiredness also need to be regulated by other internal drives such as hunger and sexual arousal,
which generally are wake-promoting \parencite{suarez-grimaltNeuralArchitectureSleep2021}
(Suarez et al 2021)

Cell type-specific transciptomic changes, with glia displaying the largest variation \parencite{doppSinglecellTranscriptomicsReveals2024}
(Dopp et al 2024)

KCs and glia had the highest number of differentially expressed genes among
annotated cell types (??? Circadian, sleep pressure or what? Check).

These data indicate that the glial clock is required for normal sleep homeostasis and
suggest that processes S and C directly influence each other in glial cells to determin
sleep-wake cycles (Dopp et al 2024)

Sleep is also potently modulated by social experience. Flies reared in isolation sleep less during
the daytime, revealing long-term plasticity in sleep modulation that is dependent on canonical memory pathways.
\parencite{shaferRegulationDrosophilaSleep2021} (Shafer and Kenee 2021)


Two subtypes of EB ring neurons, that is R5 and ring\_B were affected by either processes in opposing ways, in
    accordance with previous findings, showing that R2/R4m neurons (probably part of ring\_B) received circadian timing
    information from clock neurons, while \textbf{R5 neurons themselves encoded the sleep homeostat (Liu et al 2016)}.
    (Dopp et al 2024)

\noindent\hrulefill


\textbf{misc}

In most cases we found that a given cell type was more affected by only
one process (C or S). For example, dFB, Oct/Tyr, non-PAM DAN and R5 neurons
had many sleep driv correlates but few circadian cyclers. On the
other hand, cell types with many circadian correlates (examples listed
in paper: some KC subtypes and PGs) had no sleep drive correlates
(Dopp et al 2024)

Critically, the R5 neurons are at the core of sleep homeostasis in Drosophila (Liu et al., 2016).
\parencite{andreaniCircadianProgrammingEllipsoid2022} (Andreani et al 2022)

R5 is modulated by circadian, homeostatic and sensory input \parencite{raccugliaNetworkSpecificSynchronizationElectrical2019}

The manipulations of R2 neurons that affect sleep rebound have no effect on sleep at 
baseline, however, again supporting the idea that regulation of the
homeostatic response to sleep deprivation is mechanistically different from the regulation
of baseline sleep.
\parencite{dubowyCircadianRhythmsSleep2017} (Dubowy and Sehgal 2017)

Blocking R5 synaptic output also reduced rebound in both morning and evening
\parencite{andreaniCircadianProgrammingEllipsoid2022} (Andreani et al 2022)

Optogenetic stimulation of R5 reliably induced or amplified dFSB presynaptic SWA
following the end of our activation protocol (Extended Data Fig. 2a-c,g), indicating that
synaptic output of R5 suffices to induce ongoing SWA.
\parencite{raccugliaCoherentMultilevelNetwork2022} (Manuscript)


Helicon, as a purely excitatory recurrent network
that is also connected to R5, was modeled using neurons that switch from a more depolarized
and active state in the morning ("upstate") to a more hyperpolarized, and less active state
("downstate") at nighttime (Fig. 4a). Indeed, providing experimental support for our model,
we found that single helicon cells were significantly more hyperpolarized at night compared
to the morning (Extended Data Fig. 6d). Our simulation is in line with hyperpolarizing signals,
potentially mediated by the dFSB5 and/or other sources, paving the path for R5-mediated
entrainment of helicon activity at night (Fig. 4b,c) by rendering helicon less responsive to
inhibitory (while still responsive to excitatory) input from R5 when helicons resting membrane
potential is closer to the chloride reversal potential32. 
\parencite{raccugliaCoherentMultilevelNetwork2022} (Manuscript)

To further experimentally validate our model, we performed ex vivo helicon imaging
experiments, while briefly optogenetically stimulating R5 at 1 Hz for 20 seconds during the
day. We observed that even at this short time scale, transient, and sometimes persistent,
entrainment worked per se (Extended Data Fig. 7g-i). Because R5 activation can also entrain
dFSB activity during the day (Extended Data Fig. 2a-c), we suspect that this interaction would
effectively set helicon cells to the downstate (night setting), allowing for entrainment of
helicon by R5. Indeed, this is also in line with our behavior experiments, where, when
stimulating R5 for a significantly longer period, we were able to induce behavioral quiescence
that could even endure post stimulation (Fig. 2d-g).
\parencite{raccugliaCoherentMultilevelNetwork2022} (Manuscript)

"shifted state". We
reasoned that a main difference between ex vivo and in vivo recordings would be the absence
of sensory input or motor feedback (compare Fig. 3e, flies in DD). To simulate this, we applied
additional input to helicon (see Supplementary information for details). Indeed, this
additional input shifted the activity peaks of helicon and R5 networks by approximately 100
ms (Extended Data Fig. 7f), closely resembling the experimentally observed "shifted state" (Fig.
3c,d and Extended Data Fig. 5b). This is also in line with our experimental data, which show
that the balance controls the degree of synchronization between excitatory and inhibitory
drive and determines whether the networks are in the shifted or synchronized configuration.
\parencite{raccugliaCoherentMultilevelNetwork2022} (Manuscript)

Interestingly, compared to R5 activation alone, synchronous activation of R5 and helicon led
to a somewhat stronger reduction of locomotor activity (Fig. 4g,h) and only a small fraction of
flies was awakened by green light (Extended Data Fig. 8a-c). Indeed, activating R5 and helicon
simultaneously even blocked air puff responses (Extended Data Fig. 8d-g), suggesting that R5
and helicon synchronization can establish a strong neural filter that might also extend to other
modalities
\parencite{raccugliaCoherentMultilevelNetwork2022} (Manuscript)

Together, these findings indicate that sleep drive mediated by R5 oscillations overrides
gating of locomotion by helicon cells and further demonstrates that synchronization between
these networks locks the locomotion-initiating networks into a state that, dependent on sleep
need, reduces sensory processing and behavioral responsiveness.
\parencite{raccugliaCoherentMultilevelNetwork2022} (Manuscript)

Our connectome analysis revealed that helicon cells
as well as R5 neurons are connected to downstream EPG neurons (Fig. 5a), which represent
the flys heading direction and initiate body turns towards or away from sensory stimuli and
thus partake in reflecting the external world and controlling navigation
\parencite{raccugliaCoherentMultilevelNetwork2022} (Manuscript)

(Helicon and R5 antagonistically regulate downstream head direction neurons) Strikingly, we found that
optogenetic R5 activation led to net hyperpolarization of EPG neurons, while activation of
helicon cells led to strong depolarizations (Fig. 5f-h), indicating that R5 and helicon cells
modulate EPG activity via antagonistic neurotransmitter inputs.
\parencite{raccugliaCoherentMultilevelNetwork2022} (Manuscript)

Depolarizing EPG neurons at 1 Hz, the frequency band of synchronized R5 and helicon, had no
effect on locomotion (Extended Data Fig. 9a). In line with strong drive to EPG steering animal
behavior away from quiescence, activating a set of EPG neurons at higher frequencies (10 Hz),
approximating high excitatory input via helicon, altered locomotor activity (Fig. 5i,j and
Extended Data Fig. 9b,c) along with turning behavior (Fig. 5j,k).
\parencite{raccugliaCoherentMultilevelNetwork2022} (Manuscript)

we directly tested the sensory filtering abilities of R5-based activity. optogenetic stimulation of
R5 induces hyperpolarization of EPG in vivo. We next13
widened the illumination to cover the brain and both eyes (Fig. 6b). In control flies, EPG
neurons showed robust activation to visual input during the day (Fig. 6a,b). At night, however,
visually-evoked depolarization was clearly attenuated. Strikingly, optogenetic activation of R5
during the delivery of the visual input at daytime (Fig. 6c,d) attenuated visually-evoked
responses in EPG, reminiscent of night-time recordings and confirming the visual filtering
properties of R5.
\parencite{raccugliaCoherentMultilevelNetwork2022} (Manuscript)

Together, our study uncovers a mechanism in which circadian and homeostatic sleep need
create coherent electrical activity across different networks at night (oscillating around 1 Hz,
Fig. 1, 3). Facilitated through the dFSB (Fig. 1), the R5 network can overrule input of
locomotion-promoting helicon to EPG (Fig. 4), by temporally associating helicon activity (Fig.
3) and thus creating a neural filter (Fig. 6) to promote quiescent behavior (Fig. 2).
\parencite{raccugliaCoherentMultilevelNetwork2022} (Manuscript)

We here show
a mechanism that creates SWA across networks involved in regulating sleep (e.g. homeostatic
sleep regulation), setting up a sensory filter that attenuates visually evoked activity in
downstream navigational neurons.
\parencite{raccugliaCoherentMultilevelNetwork2022} (Manuscript)

Underlining this multi-layered regulation, artificial stimulation of the dFSB
only induced SWA in R5 at night. This places the dFSB as a prime candidate to exert inhibitory
drive to helicon5, as a prerequisite for coherence of R5 and helicon activity.
\parencite{raccugliaCoherentMultilevelNetwork2022} (Manuscript)

Synaptic plasticity of these EB neurons is both necessary and suffucuent for generating sleep drive \parencite{liuSleepDriveEncoded2016}
(Liu et al 2016)

Inhibition of neurotransmitter release reduced "rebound sleep" as well as sleep depth following mechanical sleep deprivation
(R2 neurons are required for generating homeostatic sleep drive) \parencite{liuSleepDriveEncoded2016}
(Liu et al 2016)

Inactivation of R2 neurons did not significantly affect baseline sleep, indicating a specific role for these neurons in
homeostatic regulation of sleep \parencite{liuSleepDriveEncoded2016}
(Liu et al 2016)

Activation of R2 neurons generates sleep drive even in fully rested animals \parencite{liuSleepDriveEncoded2016}
    (Liu et al 2016)

%%%%%%%%%% Oscillations

Optical multi-unit voltage recordings reveal that single R5 neurons get synchronized by activating circadian input pathways.
\parencite{raccugliaNetworkSpecificSynchronizationElectrical2019} (Raccuglia et al 2019)

(R5) We show that this synchronization depends on NMDA receptor (NMDAR) coincidence detector function, and that an interplay of cholinergic and glutamatergic inputs regulates oscillatory frequency.
Genetically targeting the coincidence detector function of NMDARs in R5, and thus the uncovered mechanism underlying synchronization, abolished network-specific compound slow-wave oscillations. It also disrupted sleep and facilitated light-induced wakening, establishing a role for slow-wave oscillations in regulating sleep and sensory gating.
\parencite{raccugliaNetworkSpecificSynchronizationElectrical2019} (Raccuglia et al 2019)

Individual R5 neurons showed oscillatory activity with peak frequencies similar to the compound signal.
Temporal correlation analysis between electrical patterns of simultaneously recorded R5 neurons showed that most depolarization phases occurred with a time lag of <50 ms (median = 13 ms; Figure 2G). Temporal overlap of single-unit activity could therefore be at the basis of the observed compound oscillations.
\parencite{raccugliaNetworkSpecificSynchronizationElectrical2019} (Raccuglia et al 2019)

single-unit delta-band oscillations require network activity potentially generated by NMDAR-mediated signaling. Interestingly, sleep deprivation leads to an upregulation of NMDAR transcripts in R5 neurons.
\parencite{raccugliaNetworkSpecificSynchronizationElectrical2019} (Raccuglia et al 2019)



When sleep drive is dissipated following 26 h of recovery sleep, increased BRP signal is lowered to levels seen in rested
    control animals -> these changes are reversible. Increased BRP signal following sleep
    deprivation was not observed in other neurons in the brain \parencite{liuSleepDriveEncoded2016}
    (Liu et al 2016)
    
    The data argue that levels of molecular markers of synaptic strength in R2 neurons correlate with levels of sleep
    drive and suggest that "sleep-need"-dependent plastic changes in the R2 circuit may be relevant for generating
    homeostatic sleep drive \parencite{liuSleepDriveEncoded2016}
    (Liu et al 2016)

    Data suggest, that Ca levels are specifically increased within R2 neurons following sleep deprivation (For other neurons
    it was unchanged) \parencite{liuSleepDriveEncoded2016}

    Ca levels in the R2 neurons correlate with varying levels of sleep drive in a scalable manner \parencite{liuSleepDriveEncoded2016}
    (Liu et al 2016)

Moreover, homeostatic R5 EB neurons integrate circadian timing and homeostatic drive;
we demonstrate that activity dependent and presynaptic gene expression, BRP expression,
neuronal output, and wake sensitive calcium levels are all elevated in the morning
compared to the evening, providing an underlying mechanism for clock programming of
time-of-day dependent homeostasis.
\parencite{andreaniCircadianProgrammingEllipsoid2022} (Andreani et al 2022)

We found that one subcluster of EB ring neurons (ring\_2) had a 
substantial number of sleep drive correlates, while the
other (ring\_1) showed only a few. We found a high number of sleep
drive-correlated genes specifically in R5 neurons,
while few to no genes were identified in the other two subclusters.
(Dopp et al 2024)


dFB neurons in the ON state expressed two types of potassium  current: voltage-dependent
        A-type16 and voltage-independent non-A-type currents. d Extended Data Fig. 6a-c).
        The current–voltage (I-V) relation of IA resembled that of Shaker, the  prototypical
        A-type channel. Non-A-type currents showed weak outward rectification with a reversal
        potential of -80 mV (Fig. 3e, g), consistent with potassium as the permeant ion, and
        no inactivation
        \parencite{pimentelOperationHomeostaticSleep2016} (Pimentel et al 2016)

(!!!!) In dFB neurons, we found many sleep drive correlates that were involved in synaptic
        formation and function. This is consistent with previous evidence linking neuronal
        activity of dFB neurons to levels of sleep pressure (Dopp et al 2024)

Elevated sleep need triggers reversible increases in cytosolic Ca levels, NMDA expresion and structural markers
    of synaptic strength, suggesting these EB neurons undergo sleep-need-dependent plasticity \parencite{liuSleepDriveEncoded2016}
    (Liu et al 2016)

    The resting membrane potential of R2 neurons
    was also more depolarized in animals following sleep deprivation, while input resistance was not altered \parencite{liuSleepDriveEncoded2016}
    (Liu et al 2016)

    Data indicate that the activity and excitability of the R2 circuit increase under conditions of greater
    sleep need and that these neurons exhibit burst firing specifically following sleep deprivation \parencite{liuSleepDriveEncoded2016}
    (Liu et al 2016)

    Two hours after mechanical sleep deprivation had ended, BRO signal in R2 ring structure was significantly increased in sleep-deprived
    flies compared to rested control flies, and this greater BRP signal was due to increases in both the number and size
    of BRP puncta in the R2 ring \parencite{liuSleepDriveEncoded2016}

    When sleep drive is dissipated following 26 h of recovery sleep, increased BRP signal is lowered to levels seen in rested
    control animals -> these changes are reversible. Increased BRP signal following sleep
    deprivation was not observed in other neurons in the brain \parencite{liuSleepDriveEncoded2016}
    (Liu et al 2016)

    The data argue that levels of molecular markers of synaptic strength in R2 neurons correlate with levels of sleep
    drive and suggest that "sleep-need"-dependent plastic changes in the R2 circuit may be relevant for generating
    homeostatic sleep drive \parencite{liuSleepDriveEncoded2016}
    (Liu et al 2016)

    Data suggest, that Ca levels are specifically increased within R2 neurons following sleep deprivation (For other neurons
    it was unchanged) \parencite{liuSleepDriveEncoded2016} (Liu et al 2016).

    Ca levels in the R2 neurons correlate with varying levels of sleep drive in a scalable manner \parencite{liuSleepDriveEncoded2016}
    (Liu et al 2016)

    Blocking the rise of intracellular Ca levels in R2 neurons significantly impared the increases in number and size of BRP puncta
    seen in these neurons following sleep deprivation \parencite{liuSleepDriveEncoded2016}.
    Ok, but what happens to synchronization? And bursting? Not stated in the paper
    (Liu et al 2016)

    We found, that a gene encoding a potassium channel ether-a-go-go (eag) correlated negatively with sleep drive in R5 neurons.
    Potassium channels, including Eag, reduce neuronal excitability (\textbf{Brüggemann et al 1993})
    This is consistent with the finding that the neuronal activity of R5 increases with the levels of sleep drive
    (Dopp et al 2024)

    The Shaker potassium channel (Cirelli et al. 2005; Bushey et al. 2007) and its modulator 
    sleepless (Koh et al. 2008) were two early hits with extreme short-sleeping phenotypes from 
    large-scale genetic screens. Both genes are expressed throughout the fly brain (Wu et al. 2009),
    and neither of these phenotypes has been fully mapped to specific neuroanatomic loci, suggesting
    that they exert widespread effects on brain activity or metabolism that feed back onto sleep 
    regulation. Shaker is a voltage-gated potassium channel involved in membrane repolarization.
    sleepless is a Ly6 neurotoxin-like molecule that, in the years since its discovery as a 
    sleep regulator, has been found to promote Shaker expression and activity and inhibit 
    nicotinic acetylcholine (nAChR) function, such that loss of sleepless might lead to 
    increased neuronal activity through multiple mechanisms 
    (Wu et al. 2009; Shi et al. 2014; Wu et al. 2014).
    \parencite{dubowyCircadianRhythmsSleep2017} (Dubowy and Sehgal 2017)

    Extended sleep deprivation (12–24 hr) elevates calcium, the critical presynaptic protein BRUCHPILOT (BRP), and action potential firing rates in R5 neurons. The changes in BRP in this region not only reflect increased sleep drive following SD but also knockdown (KD) of brp in R5 decreases rebound (Huang et al., 2020) suggesting it functions directly in regulating sleep homeostasis.
    \parencite{andreaniCircadianProgrammingEllipsoid2022} (Andreani et al 2022)

    Moreover, homeostatic R5 EB neurons integrate circadian timing and homeostatic drive; we demonstrate that activity dependent and presynaptic gene expression, BRP expression, neuronal output, and wake sensitive calcium levels are all elevated in the morning compared to the evening, providing an underlying mechanism for clock programming of time-of-day dependent homeostasis.
    \parencite{andreaniCircadianProgrammingEllipsoid2022} (Andreani et al 2022)

    We also observed significant upregulation of genes involved in ionic transport across the plasma membrane, including para, a voltage-gated sodium channel (Catterall, 2000; Loughney et al., 1989), and CG5890, a predicted potassium channel-interacting protein (KChIP) (Figure 8e and g). Mammalian KChIPs have been shown to interact with voltage-gated potassium channels, increasing current density and conductance and slowing inactivation (An et al., 2000). Two sodium:potassium/calcium antiporters, CG1090 and Nckx30C, were also upregulated (Figure 8e and g). These antiporters function primarily in calcium homeostasis by using extracellular sodium and intracellular potassium gradients to pump intracellular calcium out of the cell when calcium levels are elevated (Haug-Collet et al., 1999).
    \parencite{andreaniCircadianProgrammingEllipsoid2022} (Andreani et al 2022)

    Amongst the most significantly upregulated genes in our dataset, we found six genes that were previously identified as activity-regulated genes in Drosophila (ARGs; sr, Cdc7 (also known as l(1)G0148), CG8910, CG14186, CG17778, hr38)
    \parencite{andreaniCircadianProgrammingEllipsoid2022} (Andreani et al 2022)

    SD/extended wake results in the upregulation of many synaptic proteins (Gilestro et al., 2009). Most notable is the presynaptic scaffolding protein BRP, which is important for synaptic release (Matkovic et al., 2013), and is upregulated in the R5 neurons following 12 hr of SD (Liu et al., 2016). KD of brp in R5 neurons decreases rebound response to SD (Huang et al., 2020), suggesting that it is necessary for accumulating and/or communicating homeostatic drive. We hypothesized that differences in the propensity for R5 to induce sleep rebound in the morning/evening may be due to changes in synaptic strength that can be observed by tracking levels of BRP.
    \parencite{andreaniCircadianProgrammingEllipsoid2022} (Andreani et al 2022)

    The calcium concentration in R5 neurons increases following twelve hours of SD, suggesting that extended wakefulness can induce calcium signaling in these neurons. Blocking the induction of calcium greatly reduces rebound, supporting a critical role for calcium signaling in behavioral output (Liu et al., 2016). Furthermore, R5 neurons display morning and evening cell-dependent peaks in calcium activity across the course of the day indicating that calcium is also modulated by the clock network (Liang et al., 2019)
    \parencite{andreaniCircadianProgrammingEllipsoid2022} (Andreani et al 2022)

    DN1p activation of TuBu

    ROS build-up in TuBu


    Thermogenetic activation of the ExF/2 neurons in the dorsal fanshaped body, a 
region of the central complex, is strongly sleep-promoting (Donlea et al. 2011). Sleep 
deprivation changes the electrophysiologic properties of these neurons to favor activity,
suggesting they may play a role in output of homeostatic sleep signals (Donlea et al. 2014).

We found, that a gene encoding a potassium channel ether-a-go-go (eag) correlated negatively with sleep drive in R5 neurons.
    Potassium channels, including Eag, reduce neuronal excitability (\textbf{Brüggemann et al 1993})
    This is consistent with the finding that the neuronal activity of R5 increases with the levels of sleep drive
    (Liu et al 2016).
    (Dopp et al 2024)

    Two subtypes of EB ring neurons, that is R5 and ring\_B were affected by either processes in opposing ways, in
    accordance with previous findings, showing that R2/R4m neurons (probably part of ring\_B) received circadian timing
    information from clock neurons, while \textbf{R5 neurons themselves encoded the sleep homeostat (Liu et al 2016)}.
    (Dopp et al 2024)


    We were surprised to find that neither morning nor evening SD had much of an effect on gene expression in the R5 neurons
    \parencite{andreaniCircadianProgrammingEllipsoid2022} (Andreani et al 2022)

    single-unit delta-band oscillations require network activity potentially generated by NMDAR-mediated signaling. Interestingly, sleep deprivation leads to an upregulation of NMDAR transcripts in R5 neurons.
    \parencite{raccugliaNetworkSpecificSynchronizationElectrical2019} (Raccuglia et al 2019)


