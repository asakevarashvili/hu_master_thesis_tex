\documentclass[../main.tex]{subfiles}
\graphicspath{{\subfix{../img/}}}

\begin{document}

\subsection{Preface} \label{subsec:results_preface}

R5 neurons exhibit tonic spiking during the daytime in rested flies and switch to bursting during sleep and following sleep deprivation \parencite{liuSleepDriveEncoded2016,raccugliaCoherentMultilevelNetwork2022,raccugliaNetworkSpecificSynchronizationElectrical2019}. They are thought to be the central part of the sleep homeostasis \parencite{liuSleepDriveEncoded2016}.
However, the cellular mechanisms underlying R5 neuronal activity remain poorly understood.

Experimental studies suggest that the rhythmic bursting activity in R5 neurons may be generated by intrinsic properties of these neurons, rather than driven by oscillatory input \parencite{raccugliaNetworkSpecificSynchronizationElectrical2019} (see also Section \ref{subsubsec:circuits_in_droso_sleep}).
Furthermore, it was hypothesized that the observed bursting activity in \textit{Drosophila} R5 neurons is mediated by T-type calcium channels (Section \ref{subsubsec:experiment_ttx_t_type_block}). However, as described in Section \ref{subsubsec:experiment_ttx_t_type_block}, experimentals conducted by David Owald, Anatoli Ender and colleagues demonstrated that blocking these channels did not abolish bursting behavior.

Interestingly, when the T-type channels were blocked, the minimum membrane potential during bursts was higher compared to control conditions. Moreover, blocking sodium channels in control animals preserved slow oscillatory activity (below 1 Hz), likely mediated by calcium currents (see Section \ref{subsubsec:experiment_ttx_t_type_block}).

The aim of this thesis is to identify potential mechanisms that explain these experimental observations using a modeling approach. A conductance-based models was selected, as it can be used to explicitly model interactions between the membrane potential and ionic currents, and can provide insights into the biological mechanisms underlying neuronal behaviour.

For simplicity, single-compartment models were investigated. While such models do not account for the complex neuronal morphology, or the difference in the ion channel concentration across compartments, they can still provide a valuable insight into the cellular mechanisms governing neuronal acitivty.

The above-mentioned considerations guided the selection of previously published models to be used for the current thesis. The models required to satisfy all of the following criteria in the reported parameter regimme: (1) Exhibit bursting with a voltage trace where the membrane potential during spiking remains above that of the resting state, similar to what was observed in the R5 neurons (see Section \ref{subsubsec:experiment_ttx_t_type_block}); (2) Include a T-type calcium channel; (3) Produce approximately $1$ Hz periodic bursting under constant input current.

Unless otherwise noticed, four parameter regimes across three different models have been investigated. Throughout the following, these models are referred to as the Wang model \parencite{wangMultipleDynamicalModes1994}, Goldman model \parencite{franciRobustTunableBursting2018,goldmanGlobalStructureRobustness2001}, and Park model \parencite{parkMathematicalModelSubthalamic2021}. The models differ in their composition of ion channels, the
channels responsible for burst initiation and termination, their maximal conductances and gating kinetics. 
Model equations and parameter values were taken from the original publications unless otherwise specified, and are detailed in Appendix \ref{appendix:functions_and_parameters}.
Figure \ref{fig:model_conductances} summarizes the included ion channels, their maximal conductances, and their relative contributions to total ionic current across the investigated four parameter regimes.

Wang model is the simplest out of the three, comprising six voltage-gated ion channels.
The corresponding system of differential equations is four-dimensional: one equation governs the membrane potential, and the other three describe gating variables. Although the model includes seven unique activation and inactivation gates, three are assumed to be instantaneous and two are coupled, yielding three dynamical gating variables. This makes the model easier to be subjected to fast-slow decomposition and bifurcation analysis.

The Goldman model contains seven ion channels and ten gating variables. None of the gates are treated as instantaneous. It also includes a variable governing kinetics of intracellular calcium concentration, resulting in a 12-dimensional system when combined with the membrane potential.
Unlike the Wang model, the Goldman model incorporates a calcium-activated potassium channel. Due to the geometric structure of its nullclines, the Goldman model is more robust to parameter perturbations \parencite{franciRobustTunableBursting2018}.
Out of the parameter regimes reported by Franci and colleagues, two were selected that qualitatively reproduced the bursting dynamics observed in R5 neurons (see Figure \ref{fig:sleep_r5_knock_ttx_voltage}), with membrane potential during spiking state remaining above the membrane potential at resting state. The corresponding models are referred to as Goldman 1 and Goldman 2.

The Park model is 14-dimensional and, similar to the Goldman model, includes intracellular calcium dynamics and a calcium-activated potassium channel. However, the relative contribution of this potassium current to the total ionic current across the membrane is much smaller in the park model ($0.1\%$) compared to the Goldman model ($2.4\%$ and $2.9\%$ for the two regimes studied).

Fitting the parameters of conductance-based models to reproduce specific dynamical behaviors is generally a challenging task due to the high dimensionality of the system, the large number of parameters, and the nonlinear interactions between them \cite{alonsoVisualizationCurrentsNeural2019} (see also \textit{Considerations for modeling R5 neurons} in Section \ref{sec:discussion}). Therefore, the goal of this section is not to precisely replicate all features of R5 neuronal dynamics, but rather to investigate general mechanisms underlying the key experimental observations.
Specifically, this section aims to address the following questions: How can R5 neurons switch from bursting to tonic spiking? What mechanisms might underlie the slow-frequency oscillations following sodium channel blockade? Why does the minimal membrane potential during bursting increase following T-type calcium channel knockdown?

Accordingly, the section focuses on identifying plausible biophysical explanations for these observations, rather than reproducing the exact features of the R5 neurons such as the number of spikes per burst, burst width, or spike width after sodium channel blockade.
\end{document}