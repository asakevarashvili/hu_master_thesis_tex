\documentclass[../main.tex]{subfiles}
\graphicspath{{\subfix{../img/}}}

\begin{document}
\section{Introduction} \label{sec:introduction}

Sleep-wake cycle in animals is thought to have evolved due to evolutionary
adaptation to alteration of day and night \parencite{suarez-grimaltNeuralArchitectureSleep2021}.
Sleep is characterized by increased periods of inactivity, reduced responses to external stimuli,
tendency for recovery sleep after sleep deprivation, and rapid reversibility, making it
distinct from hibernation or coma
\parencite{shaferRegulationDrosophilaSleep2021,andreaniCircadianProgrammingEllipsoid2022,donleaRecurrentCircuitryBalancing2018}.
These behavioural hallmarks of sleep are shared between both vertebrates and invertebrates
including flies,
despite the huge variability in the complexity and size of the nervous systems of the organisms
\parencite{shaferRegulationDrosophilaSleep2021,andreaniCircadianProgrammingEllipsoid2022}.

Although the regulation of sleep at the cellular and molecular levels is not yet fully understood, across species, sleep has been shown to be important for various cellular processes (e.g. regulation of synaptic strength, cellular metabolism, etc.). Furthermore, poor sleep quality has been linked to numerous health conditions (e.g. diabetes, depression), as well as negative effects on cognitive functions, such as learning, memory, selective attention and social behaviour \parencite{shaferRegulationDrosophilaSleep2021,dubowyCircadianRhythmsSleep2017,suarez-grimaltNeuralArchitectureSleep2021}.

Apart from behavioural similarities, the genetic and molecular mechanisms underlying
sleep are also similar across species \parencite{dubowyCircadianRhythmsSleep2017}.
In vertebrates, sleep and sleepiness are thought to be linked to large-scale synchronizations in the cortex resulting in \gls{swa} and reduction of functional connectivity between brain areas \parencite{suarez-grimaltNeuralArchitectureSleep2021,raccugliaNetworkSpecificSynchronizationElectrical2019}.
Interestingly, \gls{swa} with similar characteristics have also been observed in \textit{Drosophila} melanogaster (fruit flies), where a lower degree of synchronization correlates with reduced sleep duration and arousability
threshold, as well as rebound sleep after sleep deprivation
\parencite{raccugliaNetworkSpecificSynchronizationElectrical2019}. Moreover, across species brain-wide
\gls{swa} has been shown to be important for filtering sensory information, thus increasing the
arousability threshold during sleep \parencite{raccugliaCoherentMultilevelNetwork2022}. These
striking similarities between evolutionary distinct organisms might indicate to existence of
fundamental processes governing sleep regulation \parencite{suarez-grimaltNeuralArchitectureSleep2021}.

\textit{Drosophila} is a well-established model to study sleep 
\parencite{liuSleepDriveEncoded2016,andreaniCircadianProgrammingEllipsoid2022,shaferRegulationDrosophilaSleep2021,dubowyCircadianRhythmsSleep2017}.
Although its brain is relatively simple (consisting of around 100,000 neurons \parencite{donleaRecurrentCircuitryBalancing2018} in comparison to
estimated 86 billion in the human brain \parencite{herculano-houzelRemarkableNotExtraordinary2012}), \textit{Drosophila} still
shares many similarities in sleep-regulation with vertebrates \parencite{liuSleepDriveEncoded2016}, including
mammals \parencite{suarez-grimaltNeuralArchitectureSleep2021,dubowyCircadianRhythmsSleep2017} despite long
evolutionary distance (approximately 800 million years \parencite{williamsLongReachNAAG2021}).

A network of approximately 32 neurons has been found in the central complex of \textit{Drosophila} brain that is important in sleep regulation \parencite{liuSleepDriveEncoded2016,raccugliaNetworkSpecificSynchronizationElectrical2019}. This network of neurons is commonly referred to as R5 (originally termed as R2) \parencite{raccugliaNetworkSpecificSynchronizationElectrical2019}. Studying R5 neurons has been of central interest as they are considered to encode sleep drive in fruit flies. Furthermore, expression of \gls{brp} protein, which is important for activity-dependent plasticity, has been reported to increase only in R5 following sleep deprivation \parencite{liuSleepDriveEncoded2016}.

The function of the R5 network has been characterized as analogous to the mammalian thalamus
\parencite{suarez-grimaltNeuralArchitectureSleep2021,raccugliaNetworkSpecificSynchronizationElectrical2019}.
\gls{tc} relay cells integrate sensory information prior to relaying it to the cortex \parencite{sampathkumarIntegrationSignalsDifferent2021}.
Similarly, R5 neurons are part of the \gls{eb}, which integrates sensory information to guide locomotion \parencite{yanSubtypeSpecificRolesEllipsoid2023}. 
Furthermore, both R5 and \gls{tc} neurons are filtering sensory information and acting as a
sensory gate that controls shifts between wakefulness and sleep
\parencite{raccugliaCoherentMultilevelNetwork2022,gentThalamicDualControl2018}. Both
synchronize electrical patterns and switch from tonic to burst firing (defined as slow alternating transitions between steady and spiking states
\parencite{rinzelFormalClassificationBursting1987}) with increasing sleep need, as well as promote transition to sleep
\parencite{suarez-grimaltNeuralArchitectureSleep2021, raccugliaNetworkSpecificSynchronizationElectrical2019}.

R5 neurons exhibit tonic spiking activity during the daytime and bursting activity at night and following sleep deprivation \parencite{liuSleepDriveEncoded2016,raccugliaNetworkSpecificSynchronizationElectrical2019}. Additionally, they show increased synchronization at night, which results in compound \gls{swa} \parencite{raccugliaNetworkSpecificSynchronizationElectrical2019}. Bursting is thought to facilitate neuronal synchronization, as they inhance reliability of signal transmission, spikes more reliably in postsynaptic neurons, and influence synaptic plasticity more effectively in comparison to single spikes \parencite{lismanBurstsUnitNeural1997,kimBurstSynchronizationScalefree2019}.
These suggest that the transition from spiking to bursting between day and night may represent a critical hallmark in the generation of \gls{swa} and the induction of sleep.

The cellular mechanism underlying the switch from tonic to bursting activity in R5 neurons remains poorly understood. A recent study on gene expressions in \textit{Drosophila} reported that expression of the \gls{eag} channel, which is a voltage-gated potassium channel, is negatively correlated with sleep drive in \textit{Drosophila} \parencite{doppSinglecellTranscriptomicsReveals2024}.
Increased expression of genes encoding these channels may reflect higher channel concentrations during the daytime compared to the night. Since potassium channels are generally thought to reduce neuronal exitability \parencite{bruggemannEtheragogoEncodesVoltagegated1993}, \gls{eag} mediated modulation of neuronal excitability might be a potential mechanism of tonic-to-bursting transition of the R5 activity between day and night.

It is also unknown which ionic mechanisms are involved in bursting in R5 neurons. Although there are various mechanisms of burst generation, one of the commonly discussed mechanisms involves T-type Ca$^{2+}$ and \gls{hcn} channels. The \gls{hcn} channels are activated upon hyperpolarization and mediate depolarizing current \parencite{destexheModelInwardCurrent1993}. On the other hand, activation of T-type Ca$^{2+}$ ($Ca_T$) channels requires first deinactivation to hyperpolarized potentials, followed by depolarization to the membrane potentials where its activation gate opens. Interplay between \gls{hcn}-$Ca_T$ channels has been hypothesized to underline bursting in many bursting neurons across phila \parencite{wangMultipleDynamicalModes1994,liuMultipleConductancesCooperatively2008}.

A recent unpublished study by Anatoli Ender, and David Owald investigated the role of the T-type Ca$^{2+}$ channels in the activity of the R5 neurons. Interestingly, in animals which were deprived of these channels, R5 neurons still exhibited bursting behaviour. However, their resting membrane potential was found to be at more depolarized levels. Moreover, slow oscillations of the membrane potential observed after blockade of Na$^{+}$ channels was diminished in flies where T-type channels were knocked down.

The slow oscillations observed following sodium channel blockade have been hypothesized to be mediated by the T-type calcium channels and, by analogy with other models, were thought to facilitate bursting. However, the observation that flies with T-type channel knockdown still exhibit bursting remains unexplained. It is unclear whether this is due to incomplete removal of the channels or whether other ionic mechanisms are responsible for bursting in R5 neurons.

Furthermore, observing more depolarized membrane potentials after knockdown of the T-type Ca$^{2+}$ channels may appear counterintuitive. Since calcium is a depolarizing current  one might therefore expect a hyperpolarizing effect following its reduction. A potential mechanism underlying this observation might be explained with involvement of calcium-activated potassium channels, which mediate hyperpolarizing current. A reduction in calcium influx may lead to decreased activation of these channels, thereby reducing hyperpolarizing drive and effectively having depolarizing effect.

This thesis aims to investigate whether (1) the \gls{eag} channels can underline spiking-to-bursing transision in R5 activity, (2) the slow oscillations following Na$^+$ channel blockade are mediated by T-type channels, and (3) calcium activated potassium channels can explain the increase in resting membrane potential following T-type channel knockdown. Within the scope of this thesis, these questions are explored using several existing conductance-based bursting models, and the generality of the findings is assessed with respect to the model implementations.

The work is organized as follows: Section \ref{sec:sleep_and_r5_network} provides an overview of the literature on sleep in \textit{Drosophila}, followed by experimental findings relevant to the current work and the possible mechanisms underlying the observations; Section \ref{sec:math_background} gives an overview on bursting neurons from the mathematical perspective, based on the studies on simplified dynamical systems; Section \ref{sec:materials_and_methods} describes methods used in this work; Section \ref{sec:results} provides the motivation behind chosen computational models and presents the results of the current thesis; Finally, Section \ref{sec:discussion} discusses the results from the perspective of existing literature, outlines the limitations of the current approach, and suggests directions for future work.

% \color{red}
% \begin{itemize}
%     \item Posed questions:
%     \begin{itemize}
%         \item What is the mechanism behind switching between bursting (1Hz) and (low frequency) tonic spikes?
%         \item Why is there bursting for T KD? (partial knockdown? Or L type channels?)
%         \item Why do we observe increased resting membrane potential with T KD? Ca channels are depolarizing, so
%         intuitively reducing depolarizing current should result in hyperpolarization of the membrane
%         (there was a modeling paper stating this)
%     \end{itemize}
% \end{itemize}
% \color{black}

% %%%%%%%%%%%%%%%%%%%%%%%%%%%%%%%%%%%%%%%%%%%%%%%%%%%%%%%%%%%%%%%%%%%%%%%%%%%%%%%%%%%

% \noindent\hrulefill

% \noindent\hrulefill

% "Bursting can - enhance SNR, facilitate neuropeptide release,
% increase reliability of synaptic transmission \parencite{vickstromTTypeCalciumChannels2020}
% "

% "Whether cell-autonomous conductances contribute to sustained rhythmic activities of single R5 neuron remains an open question."
% \parencite{raccugliaNetworkSpecificSynchronizationElectrical2019} (Raccuglia et al 2019)


% \begin{itemize}
%     \item Number of neurons and their size in Drosophila (Size: 2-6$\mu m$ in 
%     comparison to $10-30\mu m$ for pyramidal cells in rodents) \parencite{tuthillLessonsCompartmentalModel2009}.
% \end{itemize}

% Write a few workds about the model in Lara's thesis -> it is not
% biologically plausible model -> How can it be done with more biologically plausible model?

\end{document}