\documentclass[../main.tex]{subfiles}
\graphicspath{{\subfix{../img/}}}

\begin{document}

\section*{Abstract}

Behavioural hallmarks of sleep are shared between vertebrates and
invertebrates. \textit{Drosophila} melanogaster (commonly known as the
fruit fly) is a well-established model for studying sleep, due to 
striking similarities in sleep regulation
and architecture despite major differences in neuroanatomy.
R5 neurons, which are thought to be functionally similar to mammalian
thalamocortical (TC) relay cells, are thought to be the center of
sleep homeostasis. They exhibit tonic firing rate during the day, and switch to
bursting activity at night and following sleep deprivation of the fruit flies.
At night, they synchronize their oscillatory electrical patterns to
generate coherent Slow Wave Activity (SWA), which has been found to be
important for filtering sensory information during sleep across animal species,
increasing arousability while still allowing strong stimuli to induce awakening.
The switch from tonic spiking to bursting has been hypothesized to be
important for synchronizing R5 cells. However, the cellular mechanisms
underlying this transition are poorly understood.

Based on reports from other animal models, it has been hypothesized that bursting in R5 neurons is mediated by the T-type calcium channels.
However, a recent unpublished study by Anatoli Ender and David Owald
reported that knockdown of these channels did not abolish bursting in R5 neurons. Although the knockdown can be incomplete, the study observed an increase in minimal resting membrane potential (defined as the average of the minimal membrane potential within one oscillation cycle) following T-type channel knockdown. Since calcium is a depolarizing current, knockdown of T-type channels would be expected to lower (hyperpolarize) the resting membrane potential, not raise it.

Moreover, the same study reported that slow oscillations (on the order of a few seconds) persisted after sodium channel blockade in control flies, but not in flies with T-type channel knockdown. This suggests that T-type channels may contribute to the generation of these slow oscillations.

In this thesis, three key findings are presented.
First, we demonstrate that a potassium channel known as ether-`a-go-go (EAG) is a potential mechanism by which the R5 neurons could change their activity from tonic spiking during the day to bursting at night. This is motivated by the recent findings that the expression of the gene encoding this channel is negatively correlated with sleep drive.

Second, we show that an increase in minimal resting membrane potential following T-type channel knockdown may indicate involvement of the calcium-activated potassium channels. Computational models that included such a channel with a relatively high conductance could reproduce the experimental finding.

Third, we show that T-type channels alone cannot explain slow-frequency oscillations following sodium channel blockade. An additional slow mechanism should be present in the cell to allow stable low-frequency oscillations.



\end{document}