\documentclass[../main.tex]{subfiles}
\graphicspath{{\subfix{../img/}}}

\begin{document}

\section{Discussion}
    

\color{orange}

\cite{krummSlowlyOscillatingBrain2021} implemented R5 and helicon cells based
on Izhikevich model of burstin neurons \cite{izhikevichSimpleModelSpiking2003}.

\cite{krummSlowlyOscillatingBrain2021} suggested the inhibitory synapses between R5 and
helicon cells to explain this observation.

\color{red}

------------------------------
IMPORTANT!!!:
\begin{itemize}
    \item Citations from Lauras thesis
    \begin{itemize}
        \item In the Down state, Helicon is entrained to R5's compound rhythmicity via excitatory
        coupling. This leads to a relatively short offset (7 ms) between the two signals
        \item For Helicon in the Down state, we find a much larger and negative offset of 
        -77 ms (fig. 2.34a). We assume this is because Helicon now also receives inhibitory inputs
        from R5 neurons which prevent Helicon from firing and therefore lead to a small anti-phase
        correlation between the two signals.
    \end{itemize}
    \item From Manuscript:
    \begin{itemize}
        \item (Simulations). This is also in line with our experimental data, which show
        that the balance controls the degree of synchronization between excitatory and inhibitory
        drive and determines whether the networks are in the shifted or synchronized configuration
    \end{itemize}
    \item Remarks
    \begin{itemize}
        \item In the Lauras thesis, in the second note it should be written "Up State" instead of
        "Down State". However, this state
        corresponds to daytime rather than night. Thus this will not explain the experimental
        observations (shifted state at night)
        \item In manuscript, 1) there is no inhibition from R5 to helicon at night. Thus,
        the temporal shift might be due to the synaptic time constant between helicon and R5,
        rather than interplay between excitation and inhibition between R5 and Helicon.
        Synaptic time constant was set to be 100 ms (similar to resulted time delay between helicon and R5).
        Thus, when additional input was provided to helicon, here helicon might drive R5 and R5 might burst due to intrinsic properties.
    \end{itemize}
\end{itemize}

\color{black}


\noindent\hrulefill

\color{orange}

Facts:
\begin{itemize}
    \item Ca1T-null mutants showed increased sleep \cite{jeongCaa1TFlyTtype2015}
    \item Ca1T in drosophila are located at presynaptic terminals of R5
    
\end{itemize}

\color{red}

Discussion:
\begin{itemize}
    \item Blocking NMDR - irregular interburst interval, controls - regular.
    Chaos??? (further support for square-wave)
    \cite{raccugliaNetworkSpecificSynchronizationElectrical2019,izhikevichNEURALEXCITABILITYSPIKING2000}
    
    \item Although h current is associated with the repolarizing current (blcking sometimes reduces it),
    h current is not ncessary to observe this phase (examples of simulations).

    \cite{jeongCaa1TFlyTtype2015} Flies lacking T-Type channels increase amount of sleep. If bursting in
    R5 requires sleep, than this is on the one hand counterintuitive result. Although, knocking down
    expression of T-Type channels in whole fly might have more complex effects on sleep, as other
    circuits that affect R5 acitivty also will lack the similar channel that might result in this
    observation.

    \cite{blumAstroglialCalciumSignaling2021}: astrocytes and Toll receptors on R5 neurons.
    Toll might trigger gene expression (??? Need a bit more literature research)

    BRP - Important for regulation of calcium channels (???)

    RMP depolarized following SD (Connection to T-Type channels and increased RMP in knockdown.
    Effect of affector circuits following SD, as Ca gated K channels should have the opposite effect ???)
    \cite{liuSleepDriveEncoded2016}

    \item R5 neurons exhibit 1Hz tonic firing during day and 1Hz tonic firing during night
    \begin{itemize}
        \item According to Liu et al 2016 bursting occurs only in sleep-deprived files (Liu et al 2016). However,
        Raccuglia et al 2019 reported bursting activity at the evening (ZT8-13). The difference might be because
        Liu et al reported above-mentioned results for ZT0. As R5 is modulated by both cyrcadian and homeostatic
        processes (circadian by clock neurons \cite{doppSinglecellTranscriptomicsReveals2024})
        - this might explain the difference.
        \item Furthermore, I could not find the original paper stating that R5 neurons
        exhibit 1Hz tonic firing during day (Figures \ref{fig:tmp_single_unit_r5_day_night}, \ref{fig:tmp_frequency_vs_zt})
    \end{itemize}

    \item Raccuglia: frequency of R5 activation and locomotion. Actitation was done by optogenetics. If bursting is
    mediated by hyperpolarization activated current, then it can be that optogeneetically one directly
    activates fast system. Thus, you will need specific frequency of activation to induce similar
    effect (intrinsic bursting 1Hz).

    \item Other mechanisms are likely to be involved during normal, undisturbed sleep \cite{liuSleepDriveEncoded2016}.
    
    \item Manuscript: "Because R5 activation can also entrain
    dFSB activity during the day (Extended Data Fig. 2a-c), we suspect that this interaction would
    effectively set helicon cells to the downstate (night setting), allowing for entrainment of
    helicon by R5."
    \begin{itemize}
        \item This can also be due to DN1p clock neurons, not through dFSB
        \item While helicon cells can be set to the downstate through DN1p-dFSB circuit,
        the SWA could be achieved through R5-helicon-dFSB circuit, where
        excitatory synapses from helicon drive SWA in dFSB. It will be interesting
        to see the time lag correlation between dFSB-helicon and dFSB-R5 in the SD condition.
        Or even better - granger causality (as correlation does not tell us about causality).
        (Paper for method: 
        Reassessing hierarchical correspondences between brain and deep networks through direct interface
        \url{https://www.science.org/doi/10.1126/sciadv.abm2219})
    \end{itemize}

\end{itemize}

\color{black}


    

%% Not ture :)))
% Notably, out of the identified sleep-pressure-inducing neurons, only the inactivation of R5 neurons had effect on
% rebound sleep,  while inactivation of other neurons had no effect. \textcolor{red}{\textbf{(This can be related to the Raccuglia 2019 - reduced neurotransmitter release
% might affect R5 synchronization. Interesting question: as excitatory drive is important for synchronization,
% what will happen if only inhibitory synapses are blocked? No reduction in rebound sleep?)}}

Oscillations after TTX block - neuron might receive external input from other sources
than synaptic current (e.g. gap junctions to other neurons or astrocytes - add citations)

Although for many bursting neurons H current is necessary (blocking of H current blabla, find literature)
it is not mandatory for \gls{ahp} (image from Izhikevich book and models that have ahp but still
have very nice bursts or ahp).

\begin{itemize}
    \item In conntrast to three-compartment mmodel, single-compartment model could only estimate either I-V relationship, or LTS, but not both,
    with LTS observed in case of increased maximal conductance of T-Type channels. \cite{destexheDendriticLowthresholdCalcium1998}
\end{itemize}


We assume that the R5 neurons are intrinsic bursters, i.e. they can exhibit bursting activity due to
cell-autonomous conductances, even in case of constant input current.
Thus, when concentrating on simulations of R5 neurons to study either transition between
bursting and tonic spiking or effects of blocking specific ion channels, external input to R5 neuron
is modelled as a constant. Furthermore, for simplicity, single compartment conductance based model was chosen.
Such model 1) does not account for dendritic computations, and 2) assumes uniform distribution of
ion channels. \textcolor{green}{Nonlinear interactions in the multi-compartment model might better explain
the experimental results without need of L-Type $Ca^{2+}$ currents.}

\end{document}