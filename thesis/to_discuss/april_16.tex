\documentclass[11pt]{article}
\usepackage{../../template}

\addbibresource{../references.bib}

\begin{document}

\begin{enumerate}
    \item Comparing several models: Why we observe increase in resting membrane potential
    for one of the models (Godlman 2001), but not for others?

    \item How can one switch from bursting to spiking? Target parameters:
    \begin{itemize}
        \item intracellular calcium concentration (effectively setting V\_ca if model is Ohmic)
        \item max conductance of T-Type channels
        \item max conductance of L-type channels
        \item max conductance for persistent sodium channel
    \end{itemize}
    
    \item Helicon-R5 synchronization: why did we observe 100ms delay in the model after
    helicon activation?
    \begin{itemize}
        \item Shifted state was observed, with helicon cells leading oscillation. Hoever, this
        does not align with the assummption, that R5 causes SWA in helicon.
        \item Possible mechanism: 'early activation' via gap junctions (as described in section 1)
        \item For now, correlation was used to observe synchronization. However,
        correlation does not mean causality.
        \item It might be interesting
        to see the time lag correlation between dFSB-helicon and dFSB-R5 in the SD condition.
        Or even better - granger causality.
        (Reassessing hierarchical correspondences between brain and deep networks through direct interface
        \url{https://www.science.org/doi/10.1126/sciadv.abm2219})
        \item Citations from Lauras thesis:
        \begin{itemize}
            \item "In the Down state, Helicon is entrained to R5's compound rhythmicity via excitatory
            coupling. This leads to a relatively short offset (7 ms) between the two signals"
            \item "For Helicon in the Down state, we find a much larger and negative offset of 
            -77 ms (fig. 2.34a). We assume this is because Helicon now also receives inhibitory inputs
            from R5 neurons which prevent Helicon from firing and therefore lead to a small anti-phase
            correlation between the two signals."
        \end{itemize}
        \item Citation From Manuscript:
        \begin{itemize}
            \item (Simulations). "This is also in line with our experimental data, which show
            that the balance controls the degree of synchronization between excitatory and inhibitory
            drive and determines whether the networks are in the shifted or synchronized configuration"
        \end{itemize}
        \item Remarks
        \begin{itemize}
            \item In the Lauras thesis, in the second note it should be written "Up State" instead of
            "Down State". However, this state
            corresponds to daytime rather than night. Thus this will not explain the experimental
            observations (shifted state at night)
            \item In manuscript, 1) there is no inhibition from R5 to helicon at night (in the model). Thus,
            the temporal shift might be due to the synaptic time constant between helicon and R5,
            rather than interplay between excitation and inhibition between R5 and Helicon.
            Synaptic time constant was set to be 100 ms (similar to resulted time delay between helicon and R5).
            Thus, when additional input was provided to helicon, here helicon might drive R5 and R5 might burst due to intrinsic properties.
        \end{itemize}
    \end{itemize}

    \item Raccuglia: frequency of R5 activation is important. Actitation was done by optogenetics. If bursting is
    mediated by hyperpolarization activated current, then it can be that optogeneetically one directly
    activates fast system. Thus, you will need specific frequency of activation to induce similar
    effect (intrinsic bursting 1Hz).

    \item What are the "connection paths" from R5 to dFSB cells based on connectome (directed graph)?
    E.g. R5 - Helicon - dFSB. Should be easy to look into
\end{enumerate}



\end{document}