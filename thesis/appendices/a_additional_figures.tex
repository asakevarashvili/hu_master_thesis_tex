\documentclass[../main.tex]{subfiles}
\graphicspath{{\subfix{../img/}}}

\begin{document}

\section{Additional Figures}

\begin{figure}[!b]
    \centering
    \includegraphics[width=0.75\linewidth]{../img/modeling_r5/examples/point_circle_bursters.png}
    \caption[Codimension-1 bifurcations of resting and spiking states for "2+1" point-circle bursters.]{
        Codimension-1 bifurcations of resting and spiking states for "2+1" point-circle bursters.
        The figure shows possible bifurcations for the case, when the fast (slow) system is
        two- (one-) dimensional. The inset on the top-left indicates the axis of the plots.
        $V$ - membrane potential, $n$ ($n_{slow}$) - fast (slow) variable of dynamical system.
        Adapted from \parencite{izhikevichDynamicalSystemsNeuroscience2006}, with modifications.
    }
    \label{fig:izhikevich_point_circle_bursters}
\end{figure}

\begin{figure}[!t]
    \centering
    \includegraphics[width=0.85\linewidth]{../img/spiking_to_bursting/eag_default.png}
    \caption[EAG Channel Activation Variable and Kinetics]{
        \textbf{EAG Channel Activation Variable and Kinetics}. \textcolor{red}{TEXT!!!}
    }
    \label{fig:spiking_to_bursting_eag_params_default}
\end{figure}


\begin{figure}[!t]
    \centering
    \includegraphics[width=0.8\linewidth]{../img/spiking_to_bursting/eag_goldman.png}
    \caption[EAG Channel Activation Variable and Kinetics for Goldman Model]{
        \textbf{EAG Channel Activation Variable and Kinetics for Goldman Model}. \textcolor{red}{TEXT!!!}
    }
    \label{fig:spiking_to_bursting_eag_params_goldman}
\end{figure}

\begin{figure}[!t]
    \centering
    \includegraphics[width=0.9\linewidth]{../img/spiking_to_bursting/nonlinearity_eag.png}
    \caption[Investigating region with opposite effect of EAG channel]{
        \textbf{Investigating region with opposite effect of EAG channel}. Burst detection was implemented as described in the supplementary material of \parencite{franciRobustTunableBursting2018}. Left: same as Figure \ref{fig:spiking_to_bursting_wang_phase_diagram}. Right: Representative voltage traces obtained from simulations using the values of I$_{\text{ext}}$ and $g_{\text{EAG}}$ indicated by the orange 'x' markers on the heatmap. The region where increasing maximal conductance of EAG channels increase the number of spikes, correspond to the region with small-amplitude oscillations (see also Figure \ref{fig:nonlinearity_eag_small_amplitude_oscillations})
    }
    \label{fig:nonlinearity_eag}
\end{figure}


\begin{figure}[!t]
    \centering
    \includegraphics[width=0.6\linewidth]{../img/spiking_to_bursting/low_oscillations.png}
    \caption[Small amplitude oscillations in I$_{\text{ext}}$-$g_{EAG}$ parameter plane]{
        \textbf{Small amplitude oscillations in I$_{\text{ext}}$-$g_{EAG}$ parameter plane}
        The region where EAG channel had an opposite effect (increasing $g_{EAG}$ causes increase in the number of spikes per burst) corresponds to the region with low-amplitude oscillations.
    }
    \label{fig:nonlinearity_eag_small_amplitude_oscillations}
\end{figure}

\end{document}