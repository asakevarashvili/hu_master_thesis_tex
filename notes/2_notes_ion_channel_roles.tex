\documentclass[11pt]{article}
\usepackage{../template}

\addbibresource{../references.bib}

\begin{document}

%%%%%%%%%%%%%%%%%%%%%%%%%%%%%%%%%%%%%%%%%%%%%%%%%%%
\textbf{General}

% Bursting: T-Type + i\_h (sag) \cite{wangMultipleDynamicalModes1994}

% T-Type and h current are largly responsible for subthreshold
% properties and slow oscillations in thalamic cells \cite{wangMultipleDynamicalModes1994}

% Wang 1994: Bistability due to h-current - bursring vs subthreshold 
% oscillations based on initial step current (0.25 vs -1.2 to -0.47) \cite{wangMultipleDynamicalModes1994}

% Activation of K currents - response to depolarizing current got smaller \cite{mccormickModelElectrophysiologicalProperties1992}

% CA1 pyramidal cells experimentally exhibited very slow dynamics (> 1s) that
% may cause rhythmic bursting to change to repetitive spikes \cite{golombContributionPersistentNa2006}  

% Reducing Ca outside concentration is not the same as blocking Ca currents \cite{golombContributionPersistentNa2006}

% Ns increased with gCa for small gC values and decreased with gCa fir karge gC values \cite{golombContributionPersistentNa2006}

% Bursting can - enhance SNR, facilitate neuropeptide release,
% increase reliability of synaptic transmission \cite{vickstromTTypeCalciumChannels2020}


%%%%%%%%%%%%%%%%%%%%%%%%%%%%%%%%%%%%%%%%%%%%%%%%%%%
% \textbf{T-Type channel}

% Unique features: low-voltage activation, rapid voltage-dependent
% inactivation, generation of transent currents that can trigger
% burst of high-frequency APs \cite{vickstromTTypeCalciumChannels2020}.

% Are inactivated at typical resting membrane potentials, can be
% de-inactivated by hyperpolarization - "rebound bursting" \cite{vickstromTTypeCalciumChannels2020}


% Blocking T-Type channels with antagonist Z944 significantly reduced or abolished
% bursting activity \cite{vickstromTTypeCalciumChannels2020}

% Imparts possibility to bursting at specific freqeucny \cite{wangMultipleDynamicalModes1994}

% T Should provide the inward current for the generation of low-threshold
% Ca sppikes \cite{huguenardSimulationCurrentsInvolved1992}

% Underlines burst firing \cite{huguenardSimulationCurrentsInvolved1992}

% Rate of activation of T is considerably more rapid than its rate of inactivation,
% a property that should facilitate the generation of prolonged low-threshold
% Ca spikes  \citeyear{huguenardSimulationCurrentsInvolved1992}

% Mediate LVA current in ET cells \cite{liuMultipleConductancesCooperatively2008}

% "Window T-Type Ca current plays no obvious role in ET cell bursting"
% \cite{liuMultipleConductancesCooperatively2008}

% "Activation of T-Type Ca channels is required for low-threshold Ca spikes and
% spontaneous bursting" \cite{liuMultipleConductancesCooperatively2008}

% "Localizing T-channels in the dendrites decreases the excitability of the cell with
% respect to LTS generation" \cite{destexheDendriticLowthresholdCalcium1998}

% T-Type calcium channels are located at presynaptic terminals of R5 neurons
% (unpublished study by David Owald, Anatoli Ender
% \textcolor{red}{How to properly give credits?}).

% I\_T resulted in a large rebound Ca spike after hyperpolarization \cite{mccormickModelElectrophysiologicalProperties1992}

% LTS: Low threshold Ca spike \cite{mccormickModelElectrophysiologicalProperties1992}

%%%%%%%%%%%%%%%%%%%%%%%%%%%%%%%%%%%%%%%%%%%%%%%%%%%
% \textbf{H current}

% (Computational) Research of interest because suggested importance \cite{destexheModelInwardCurrent1993}

% It is critical to modulation of the voltage-time course
% of the cell at hyperpolarized membrane potentials and may provide
% a "pacemaker" potential for rhythmic burst generation.
% \cite{huguenardSimulationCurrentsInvolved1992}

% One function is to generate pacemaker activity \cite{liuMultipleConductancesCooperatively2008}

% Contributes to setting minimal membrane potential (MMP) - makes it more depolarized
% level \cite{liuMultipleConductancesCooperatively2008}

% Contributes to burst frequency and duration (higher frequency and slower duration)
% \cite{liuMultipleConductancesCooperatively2008}

% Ih functions to limit postburst hyperpolarization and depolarizes the membrane to the
% activation voltages of other
% conductances (INaP and IT/L) required for burst generation.
% \cite{liuMultipleConductancesCooperatively2008}

% I\_h resulted in a depolarizing sag in response to hyperpolarization \cite{mccormickModelElectrophysiologicalProperties1992}

% Apparent afterhyperpolarization after Ca spike \cite{mccormickModelElectrophysiologicalProperties1992}

% Increases burst frequency \cite{mccormickModelElectrophysiologicalProperties1992}

%%%%%%%%%%%%%%%%%%%%%%%%%%%%%%%%%%%%%%%%%%%%%%%%%%%%%%%%%%%%%%%%%
\textbf{BK current (Large-conductance Ca Activated K)}
"Activation of BK is both voltage and Ca dependent, so that deactivation of this outward
conductance can be very rapid and complete on membrane hyperpolarization" \cite{liuMultipleConductancesCooperatively2008}        

"BK channels have large conductance, so that their activation can produce strong membrane hyperpolarization" \cite{liuMultipleConductancesCooperatively2008}

What activates I\_BK? As I\_T and I\_L play important role in spontaneous bursting,
both might provide both - depolarization and intracellular Ca elevation for BK channel activation.
In this case, BK should play a role in terminating LTS. Alternatively, I\_T and I\_L might be insufficient
to increase intracellular Ca for BK channel activation within the voltage range of depolarizing envelope.
In this case, transient sodium (I\_NaT)-mediated action potentials rather than I\_L I\_T mediated LTS might
be requred to activate BK
channels because, in addition to strong depolarization, action potentials can also increase intracellular
Ca by activating HVA Ca channels. \cite{liuMultipleConductancesCooperatively2008}

For ET cells, depolarizing voltage and Ca associated with I\_T and I\_L are insufficient to
activate I\_BK. \cite{liuMultipleConductancesCooperatively2008}

I\_HVA is required for BK channel activation. BK channels are likely activated by both intracellular
Ca increase and strong depolarization during action potentials \cite{liuMultipleConductancesCooperatively2008}

BK and HVA Ca currents contribute to repolarization of action potentials \cite{liuMultipleConductancesCooperatively2008}

BK current regulates burst duration by contributing to burst termination \cite{liuMultipleConductancesCooperatively2008}

After inactivation of IT, the outward current generated by IBK may override the residual
inward currents attributable to INaP and/or IL. The resulting hyperpolarization would
deactivate these two inward currents and drive the membrane toward the MMP. \cite{liuMultipleConductancesCooperatively2008}


%%%%%%%%%%%%%%%%%%%%%%%%%%%%%%%%%%%%%%%%%%%%%%%%%%%
% \textbf{I\_A - Rapidly activating and inkactivating K (Rapidly inactivating ant transient)}

% Modulating initial components of low-threshold Ca spikes \cite{huguenardSimulationCurrentsInvolved1992}

% May fascilitate slow repetitive fring and may interact with I\_T during Ca-dpendent
% burst fiing \cite{huguenardSimulationCurrentsInvolved1992}

% May contribute to both - low-threshold Ca spikes and fast Na spikes \cite{huguenardNovelTtypeCurrent1992}

% Activation kinetics are probably sufficiently fast to allow contribution
% ti the repolarization of Na-dependent action potentials \cite{huguenardSimulationCurrentsInvolved1992}

% Addtion of I\_A resulted in a delay in the response of the model cell to a depolarizing
% current pulse \cite{mccormickModelElectrophysiologicalProperties1992}

% Slowed rate of action potential generation \cite{mccormickModelElectrophysiologicalProperties1992}

%%%%%%%%%%%%%%%%%%%%%%%%%%%%%%%%%%%%%%%%%%%%%%%%%%%
% \textbf{I\_K2 - Slowly activating and inkactivating K}

% May affect more later portions of low-threshold Ca spikes
% (later than initial components modulated by I\_A) \cite{huguenardSimulationCurrentsInvolved1992}

% May control repetitive firing rate \cite{huguenardSimulationCurrentsInvolved1992}

% LTS: Repolarization \cite{mccormickModelElectrophysiologicalProperties1992}

% Slowed rate of action potential generation \cite{mccormickModelElectrophysiologicalProperties1992}

%%%%%%%%%%%%%%%%%%%%%%%%%%%%%%%%%%%%%%%%%%%%%%%%%%%
% \textbf{Intracellular Ca concentration}
% Intracellular Ca buffering stabilizes spontaneous bursring activity"
% \cite{liuMultipleConductancesCooperatively2008}
% In Mouse Olfactory Bulb External tufted (ET) cells.

% Intracellular Ca buffer plays imporant role in maintaining resting membrane potential
% and input resistance \cite{liuMultipleConductancesCooperatively2008}

%%%%%%%%%%%%%%%%%%%%%%%%%%%%%%%%%%%%%%%%%%%%%%%%%%%
% \textbf{Extracellular Ca concentration}

% Lowering Ca from 2 to 0 mM caused increase in spike ADP and majority
% of nueonrs culminated in bursts. Also lowering ca reduced spike frequency
% adaptation \cite{golombContributionPersistentNa2006}


%%%%%%%%%%%%%%%%%%%%%%%%%%%%%%%%%%%%%%%%%%%%%%%%%%%%%%%%%%%%%%%%%%%
% \textbf{L-Type current}

% Mediate Ca LVA current in PG cells \cite{liuMultipleConductancesCooperatively2008}

% "Have traditionally been regarded as HVA Ca channels"
% \cite{liuMultipleConductancesCooperatively2008}

% "L-Type channels inactivate slowly" (Catterall et al 2005)
% \cite{liuMultipleConductancesCooperatively2008}

% "L-Type Ca channels prolong the duration of the depolarizing envelope and
% increase the number of spikes per burst" \cite{liuMultipleConductancesCooperatively2008}
% \textcolor{red}{Also was similar result for Drosophila motor neurons - find it !!!}

% LTS: Activation of I\_L during the peak of Ca spike lead to activation of I\_C \ref{mccormickModelElectrophysiologicalProperties1992}
%%%%%%%%%%%%%%%%%%%%%%%%%%%%%%%%%%%%%%%%%%%%%%%%%%%%%%%%%%%%%%%%%%%
% \textbf{Persistent Na current}

% NaP is required to trigger depolarizing event in mouse ET cells
% \cite{liuMultipleConductancesCooperatively2008}

% TTX sensitive NaP \cite{liuMultipleConductancesCooperatively2008}

% Increasing NaP in the model increases number of spikes within a burst
% but is neither necessary nor sufficient for bursting \cite{golombContributionPersistentNa2006}

% Bursting in low ca outside concentration is driven predominantly by NaP \cite{golombContributionPersistentNa2006}

% Bursting in truncated neurons bathed in ca free is due to interplay between NaP and I\_M \cite{golombContributionPersistentNa2006}

% Increasgin g may induce bursting behaviour and increase Ns \cite{golombContributionPersistentNa2006}

% As g NaP increased the propensity for bursting increased, so that at 0.08 neuron became periodic burster \cite{golombContributionPersistentNa2006}

% For given Ns, f increased as gNaP or Iapp increased \cite{golombContributionPersistentNa2006}

% For other parameter regime Ns increased with gNaP and decreaed with gM \cite{golombContributionPersistentNa2006}

%%%%%%%%%%%%%%%%%%%%%%%%%%%%%%%%%%%%%%%%%%%%%%%%%%%%%%%%%%%%%%%%%
% \textbf{SK current (Small-conductance Ca Activated K)}

% SK channels do not play an obvious role in the regulation of spontaneous bursting
% of ET cells. \cite{liuModelNeuronActivityDependent1998}


%%%%%%%%%%%%%%%%%%%%%%%%%%%%%%%%%%%%%%%%%%%%%%%%%%%%%%%%%%%%%%%%
% \textbf{M-Type K current (Not Written anything about that)}

% M-type K current allows bursting by shifting neuronal behavior between a silent and
% a tonically active state provided the kinetics of the spike generating
% currents are sufficiently, although not extremely fast \cite{golombContributionPersistentNa2006}

% For other parameter regime Ns increased with gNaP and decreaed with gM \cite{golombContributionPersistentNa2006}

% For given Ns, f decreased as gM increased \cite{golombContributionPersistentNa2006}
%%%%%%%%%%%%%%%%%%%%%%%%%%%%%%%%%%%%%%%%%%%%%%%%%%%%%%%%%%%%%%%%%%
% \textbf{Leak Currents}

% Two leak currents K and Na determined in large part the resting membrane potential and
% apparent input resistance of te model neuron \cite{mccormickModelElectrophysiologicalProperties1992}.

% RMP and resting input resistance \cite{mccormickModelElectrophysiologicalProperties1992}

\end{document}