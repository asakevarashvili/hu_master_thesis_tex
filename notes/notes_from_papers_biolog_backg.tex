\documentclass[11pt]{article}
\usepackage{../template}

\graphicspath{ {./img/} }
\addbibresource{../references.bib}

\begin{document}
\begin{itemize}
    \item Common ancestor of humans and Drosophila lived around 800 million
    years ago \parencite{williamsLongReachNAAG2021}

    % Suarez et al 2021
    \item Studying less complex brain of
    Drosophila melanogaster led to a deeper understangind of the neural
    principles of sleep regulation \parencite{suarez-grimaltNeuralArchitectureSleep2021}
    (Suarez et al 2021)

    \item Slow wave activity provides the neurophysiological basis to establish a
    sensory gate that suppreses sensory processing to
    provide a resting phase which promotes synaptic rescaling and
    clearance of metabolites from the brain \parencite{suarez-grimaltNeuralArchitectureSleep2021}
    (Suarez et al 2021)
    \item Evolutionary adaptation to day/night - Hunting and resting \parencite{suarez-grimaltNeuralArchitectureSleep2021}
    (Suarez et al 2021)

    \item Distored balance between waking and sleep has been linked to various
    health conditions, like hypertension, diabetes, depression and deficits in
    attention and learning (Alhola andPolo-Kantola 2007, Mullington et al 2021)
    \parencite{suarez-grimaltNeuralArchitectureSleep2021}
    (Suarez et al 2021)

    \item Common between spicies: SWA resulting from large scale synchronization
    -> reduction of functional connectivity -> low arousability;
    \parencite{suarez-grimaltNeuralArchitectureSleep2021}
    (Suarez et al 2021)

    \item Differences: For some birds and marine mammals one half of the brain
    generates SWA, while other half remains more easily arousable, allowing
    the sleeping animal to maintain a certain level of alertness \parencite{suarez-grimaltNeuralArchitectureSleep2021}
    (Suarez et al 2021)

    \item In mammals, deep sleep occurring during NREM sleep is characterized
    by a particularly high arousal threshold and slow-wave activity (up to 4.5 Hz)
    resulting from cortical neurons synchronizing their electrical patterns (Bonnet et al 1978,
    Buzsaki and Draguhn 2004) \parencite{suarez-grimaltNeuralArchitectureSleep2021}
    (Suarez et al 2021)

    \item Slow-wave sleep has been shown to be important for memory consolidation (Klinzing et al 2019),
    and the clearance of metabolites (Fultz et al 2019), which would otherwise accumulate and increase the
    susceptibility for neurodegenerative diseases \parencite{suarez-grimaltNeuralArchitectureSleep2021}
    (Suarez et al 2021)

    \item Mechanisms must exist for specific sensory signatures to awaken us even from our
    deepest slumber. How neural networks can differentiate sensory stimuli
    and mediate awakening from deep sleep is very complex and far from understood \parencite{suarez-grimaltNeuralArchitectureSleep2021}
    (Suarez et al 2021)

    \item Evolutionary distant brains, to probe for fundamental processes regulating
    sleep and wakefulness \parencite{suarez-grimaltNeuralArchitectureSleep2021}
    (Suarez et al 2021)

    \item Cognitive impariments in sleep-deprived humans have been linked to local
    synchronization of SWA, indicating a role for SWA in shutting down sensory
    processing and signaling sleep need to undergo synaptic plasticity (Quercia et al 2018)
    \parencite{suarez-grimaltNeuralArchitectureSleep2021}
    (Suarez et al 2021)

    \item Sleep-like state indicates that slow wave sleep might be an inheret property
    of neural networks (Krueger et al 2008). In fact, even isolated cortical slices retain the ability
    to generate synchronized SWA (Sanchez-Vuves 2020) \parencite{suarez-grimaltNeuralArchitectureSleep2021}
    (Suarez et al 2021)

    \item In Drosophila, synchronized SWA does not only mediate sleep need but also
    fascilitates consolidates sleep phases and decreases arousability during the
    night \parencite{suarez-grimaltNeuralArchitectureSleep2021,raccugliaNetworkSpecificSynchronizationElectrical2019}
    (Suarez et al 2021)

    \item In 2017 the Nobel prize in hysiology was awarded to the Drosophila
    researchers that made significant contributions to understanding the
    modelcular clockwork of circadian rhythms (Hardin et al 1990, Zehring et al 1984)
    \parencite{suarez-grimaltNeuralArchitectureSleep2021}
    (Suarez et al 2021)

    \item The transciptional and translational feedback loops that represent the
    mechanistic basis for generating circadian rhythms at the cellular level
    are found in all animals and plants \parencite{suarez-grimaltNeuralArchitectureSleep2021}
    (Suarez et al 2021)

    \item Ensuring tha sleep always occurs during the same phase of the day \parencite{suarez-grimaltNeuralArchitectureSleep2021}
    (Suarez et al 2021)

    \item Fruit flies are mostly active during dusk and dawn, having an extended period
    of rest in the middle of the day and exhibiting deeper sleep throughtout
    the night (Shaw et al 2000) \parencite{suarez-grimaltNeuralArchitectureSleep2021}
    (Suarez et al 2021)

    \item (In Drosophila) Transition frin wajubg ti skeeo has been assicuated with an increase
    in oscillatory activity in the centran brain (Yap et al 2017) \parencite{suarez-grimaltNeuralArchitectureSleep2021}
    (Suarez et al 2021)

    \item As in humans, sleep deprivation leads to deficits in selective visual attention (Kirszenblat et al 2018)
    and memory performance (Donela 2019) as well as to the accumulation of metabolites in the brain
    (van Alphen et al 2021) \parencite{suarez-grimaltNeuralArchitectureSleep2021}
    (Suarez et al 2021)

    \item (Dorosphila) Neural networks involved in mediating hunber and sexual arousal reduce
    sleep while neural networks processing visual information and social cues have
    been shown to be sleep-promoting. The mushroom bodies, a higher-order brain
    network curcially involved in olfactory memore in Drosophila, have
    also been implicated in sleep regulation. Some mushroom body neurons have been
    shown to be wake-promoting, while others seem sleep-promoting \parencite{suarez-grimaltNeuralArchitectureSleep2021}
    (Suarez et al 2021)

    \item It is within the central complex that we find an oscillatory activity that is
    tightly linked to sleep need and sleep quality \parencite{suarez-grimaltNeuralArchitectureSleep2021,raccugliaNetworkSpecificSynchronizationElectrical2019}
    (Suarez et al 2021)

    \item In fruit files caffeine promotes wakefulness by directly activating dopaminergic neurons
    (Nall et al 2016) \parencite{suarez-grimaltNeuralArchitectureSleep2021}, impressiely demonstraing that humans and flies
    must share ancient neural mechanisms for homeostatic sleep regulation \parencite{suarez-grimaltNeuralArchitectureSleep2021}
    (Cite first source + one about humans)
    (Suarez et al 2021)

    \item The dFSB is often referred to as "master sleep switch" because its optogenetic activation ha been tightly
    linked to reducing arousability (Tropup et al 2018) and inducing sleep (Donela et al 2009, Pimentel et al 2019).
    During course of the day activity-dependent build-up of ROS in dFSB neurons increases their excitability (Kempf et al 2019),
    leading to increased inhibition of the helicon cells during the night (Donela et al 2018). Helicon cells process
    light cues to induce locomotion and their inhibition therefore reduces the flow of visual information and
    suppress locomotion (Donela et al 2018). Here, R5 ring neurons, which have reciprocal synaptic connections to the
    helicon cells provide another layer of homeostatic sleep regulation that is directly linked to pricessing
    compex visual and orientation behavior \parencite{suarez-grimaltNeuralArchitectureSleep2021}
    (Suarez et al 2021)

    \item R5 neurons display intriguing functional and physiological analogies to the mammalian thalamus.
    Thalamus can filter sensory information, essentially acting as a sensory gate that regulates sleep/wake transitions
    (Gent et al 2018). As sleep need increases, thalamic reticular neurons switch from tonic firing to bursting and promote network
    synchronization that generates SWA and facilitates the transition to sleep (Llinas and Steriade 2006), Similarly,
    with increasing sleep need, R5 neurons start bursting and synchronize their electrical patterns, generating
    NMDA receptor-dependent compound SWA (Liu et al 2016, Raccuglia et al 2019) \parencite{suarez-grimaltNeuralArchitectureSleep2021}
    (Suarez et al 2021)

    \item Similar to the thalamus or the local sleep phenomenon, network-specific slow-wave synchronizations in Drosophila
    can establih a gate at the level of neural networks that suppresses sensory processing and mediates sleep need \parencite{suarez-grimaltNeuralArchitectureSleep2021}
    (Suarez et al 2021)

    \item Sleep need and subjectie tiredness also need to be regulated by other internal drives such as hunger and sexual arousal,
    which generally are wake-promoting \parencite{suarez-grimaltNeuralArchitectureSleep2021}
    (Suarez et al 2021)

    \item Homeostatic sleep need in mammals and flies is genearted by an accumulation of byproducts of neural acivity, which often correlates
    with the amount fond compexity of sensory information \parencite{suarez-grimaltNeuralArchitectureSleep2021}
    (Suarez et al 2021)

    \item In Drosophila, spatial learning depending on ring neurons has been shown to deteriorate in sleep-deprived flies
    (Melnattur et al 2020, Neuster et al 2008) \parencite{suarez-grimaltNeuralArchitectureSleep2021}.
    (Suarez et al 2021)

    \item Spatial information could "accumulate" in ring neurons to the point where they switch from processing sensory information
    to mediating sleep need which might require R5 neurons to start bursting and synchronize their electrical patterns
    (Liu et al 2016, Raccuglia et al 2019) \parencite{suarez-grimaltNeuralArchitectureSleep2021}
    (Suarez et al 2021)

    \item Prolonged optogenetic activation of TuBu neurons leads to synchronized SWA in R5 neurons and dramatically increases
    sleep (Guo et al 2018, Raccuglia et al 2019), illustrating the link between visual processing, network synchronization
    and sleep need \parencite{suarez-grimaltNeuralArchitectureSleep2021}
    (Suarez et al 2021)

    \item As our molecular clocks are not precisely running on a 24 h cycle, so-called clock neurons need light input to constantly reset the molecular
    clocks (Dunlap 1999) \parencite{suarez-grimaltNeuralArchitectureSleep2021}
    (Suarez et al 2021)

    \item Dorso-posterior clock nerons (DN1p) provide circadian information to the TuBu neurons, illustrating two important
    aspects of circadian retulation (Guo et al 2018) \parencite{suarez-grimaltNeuralArchitectureSleep2021}
    (Suarez et al 2021)

    \item Circadian time influences sensory processing. Secondly, the circadian time could determine when and to
    what extent sensory processing leads to the accumulation of sleep need or subjective tiredness \parencite{suarez-grimaltNeuralArchitectureSleep2021}
    (Suarez et al 2021)

    \item Circadian and homeostatic sleep regulation have coevolved and are thus inexorabl linked to ech other \parencite{suarez-grimaltNeuralArchitectureSleep2021}
    (Suarez et al 2021)

    \item Optigenetic activation of DN1p clock neurons leads to oscillatory activity in ring neurons,
    demonstraing the circadian influence of generating sleep need at the level of neural networks (Guo et al 2018) \parencite{suarez-grimaltNeuralArchitectureSleep2021}
    (Suarez et al 2021)

    \item Circadian input to visually sensitive TuBu neurons might promote sensory processing during the day while facilitating
    the switch of R5 neurons from tonic firing to bursting in the night \parencite{suarez-grimaltNeuralArchitectureSleep2021}
    (Suarez et al 2021)

    \item TuBu neurons provide parallel synaptic input to helicon cells and R5 neurons (Hulse et al 2020).
    While R5 neurons might use this visual information for navigation, light information in helicon cells might be
    decisive whether locomotion is induced or not. Interacions between the helicon cells and R5
    neurons could therefore integrate use-dependent sleep need, the presence of absence of light and the circadian time.
    \parencite{suarez-grimaltNeuralArchitectureSleep2021}
    (Suarez et al 2021)

    % Liu et al 2016
    \item We identify a subset of EB neurons whose activation generates sleep drive \parencite{liuSleepDriveEncoded2016}
    (Liu et al 2016)
    
    \item Elevated sleep need triggers reversible increases in cytosolic Ca levels, NMDA expresion and structural markers
    of synaptic strength, suggesting these EB neurons undergo sleep-need-dependent plasticity \parencite{liuSleepDriveEncoded2016}
    (Liu et al 2016)

    \item Synaptic plasticity of these EB neurons is both necessary and suffucuent for generating sleep drive \parencite{liuSleepDriveEncoded2016}
    (Liu et al 2016)

    \item Substantial accumulation of sleep drive in many animals takes hours of wakefulness and is often
    maintained for hours even after sleep is initiated (Daan et al 1984, Huber et al 2000, 2004) \parencite{liuSleepDriveEncoded2016}
    (see - what is meant by accumulation)
    (Liu et al 2016)

    \item From an engineering perspective, hoeostatic systems require three components: 1) sensor that periodically samples
    the state variable 2) an integrator that processes this information to determine homeostatic drive and 3) an
    effector that responds to this drive by directly manipulating the state variable (Enderle and Bronzino 2012) \parencite{liuSleepDriveEncoded2016}
    (Liu et al 2016)

    \item Drosophila is an established genetic model system for studying sleep and shares similar sleep-regulatory
    mechanisms with vertebrates (Cirelli 2009, Sehgal and Mignot 2011). For example, sleep in fruit flies is regulated by
    both the circadian clock and homeostatic sleep drive (Donela et al 2014, Kunst et al 2014, Liu et al 2014) \parencite{liuSleepDriveEncoded2016}
    (Liu et al 2016)

    \item Data suggests that activating R2 neurons genetically induces persistent sleep drive \parencite{liuSleepDriveEncoded2016}
    (Liu et al 2016)

    \item Inhibition of neurotransmitter release reduced "rebound sleep" as well as sleep depth following mechanical sleep deprivation
    (R2 neurons are required for generating homeostatic sleep drive) \parencite{liuSleepDriveEncoded2016}
    (Liu et al 2016)

    \item Inactivation of R2 neurons did not significantly affect baseline sleep, indicating a specific role for these neurons in
    homeostatic regulation of sleep \parencite{liuSleepDriveEncoded2016}
    (Liu et al 2016)

    \item Under baseline conditions, we recorded from two time points: ZT0-ZT2 (when speel pressure should be low)
    and TZ13-ZT15 when sleep pressure should be moderately elevated. We found that spontaneous AP firing rate of R2
    neurons was increased 3fold at ZT13-15, compared ZT0-2. Moreover, following sleep deprivaion, the spontaneous
    AP firing rate further increased, compared to rested flies assessed at ZT0-2. Strikingly, bursting activity
    was were seen in the majority of recordings from sleep-deprived flies, but never seen under baseline conditions at ZT0-2
    or ZT13-15, suggesting a switch to a potentiated state after sleep loss. \parencite{liuSleepDriveEncoded2016}
    (Liu et al 2016)

    \item The resting membrane potential of R2 neurons
    was also more depolarized in animals following sleep deprivation, while input resistance was not altered \parencite{liuSleepDriveEncoded2016}
    (Liu et al 2016)

    \item Data indicate that the activity and excitability of the R2 circuit increase under conditions of greater
    sleep need and that these neurons exhibit burst firing specifically following sleep deprivation \parencite{liuSleepDriveEncoded2016}
    (Liu et al 2016)

    \item Two hours after mechanical sleep deprivation had ended, BRO signal in R2 ring structure was significantly increased in sleep-deprived
    flies compared to rested control flies, and this greater BRP signal was due to increases in both the number and size
    of BRP puncta in the R2 ring \parencite{liuSleepDriveEncoded2016}
    (Liu et al 2016)

    \item When sleep drive is dissipated following 26 h of recovery sleep, increased BRP signal is lowered to levels seen in rested
    control animals -> these changes are reversible. Increased BRP signal following sleep
    deprivation was not observed in other neurons in the brain \parencite{liuSleepDriveEncoded2016}
    (Liu et al 2016)

    \item The data argue that levels of molecular markers of synaptic strength in R2 neurons correlate with levels of sleep
    drive and suggest that "sleep-need"-dependent plastic changes in the R2 circuit may be relevant for generating
    homeostatic sleep drive \parencite{liuSleepDriveEncoded2016}
    (Liu et al 2016)

    \item Data suggest, that Ca levels are specifically increased within R2 neurons following sleep deprivation (For other neurons
    it was unchanged) \parencite{liuSleepDriveEncoded2016}
    (Liu et al 2016)

    \item Ca levels in the R2 neurons correlate with varying levels of sleep drive in a scalable manner \parencite{liuSleepDriveEncoded2016}
    (Liu et al 2016)

    \item Blocking the rise of intracellular Ca levels in R2 neurons significantly impared the increases in number and size of BRP puncta
    seen in these neurons following sleep deprivation \parencite{liuSleepDriveEncoded2016}.
    Ok, but what happens to synchronization? And bursting? Not stated in the paper
    (Liu et al 2016)

    \item These data suggest that Ca-dependent plastic changes in R2 neurons are specifically required for proper
    homeostatic regulation of sleep \parencite{liuSleepDriveEncoded2016}
    (Liu et al 2016)

    \item Increasing temperature for animals expressing dTrpA1 in R2 neurons led to an increase in number and size of
    BRP puncta, similar to that seen following sleep deprivation, suggesting increase in synaptic strength. Moreover, as
    little as 30-min activation of dTrpA1 in the R2 circuit inducted a significant increase in sleep mimicking "rebound sleep"
    following the temperature elevation, despite the animals being fully rested. Thus, these data suggest that
    Ca-dependent changes in synaptic strength of the R2 circuit are not only necessary, but also sufficient for
    encoding sleep drive \parencite{liuSleepDriveEncoded2016}. (If dTrpA1 are non-selective cation channels, why
    increasing temperature would increase Ca concentration?)
    (Liu et al 2016)

    \item Activation of R2 neurons generates sleep drive even in fully rested animals \parencite{liuSleepDriveEncoded2016}
    (Liu et al 2016)

    \item While R2 neurons play a crucial role in the homeostatic regulation of sleep following sleep deprivation,
    additional mechanisms likely play a role under baseline conditions \parencite{liuSleepDriveEncoded2016}
    (Liu et al 2016)

    \item We propose that it is encoded by Ca and NMDA-receptor dependent plastic changes in a dedicated neural circuit \parencite{liuSleepDriveEncoded2016}
    (Liu et al 2016)

    \item (Okay, so, if Ca concentration affects the bursting, maybe one should model Ca dependent channels with
    Goldman-Hodgkin-Katz flux equation ??? It is even second argument for this equation, the first one being that
    the fit is better!!!)
    (Liu et al 2016)

    % Dopp et al 2024
    \item Sleep-wake cycle is determined by circadian and sleep homeostatic processes \parencite{doppSinglecellTranscriptomicsReveals2024}
    (Dopp et al 2024)

    \item Cell type-specific transciptomic changes, with glia displaying the largest variation \parencite{doppSinglecellTranscriptomicsReveals2024}
    (Dopp et al 2024)

    \item Glia are also among the few cell types whose gene expression correlates with both sleep hoeostat and circadian clock \parencite{doppSinglecellTranscriptomicsReveals2024}
    (Dopp et al 2024)

    \item Sleep is regulated by two independent processes: the circadian system and the sleep homeostatic system \parencite{doppSinglecellTranscriptomicsReveals2024}
    (Dopp et al 2024)

    \item Circadian clock primarily regulates the timing of sleep, known as process C. It consists of a transciptomial-translational
    feedback loop of core clock genes. However, it is unclear whether and how process C affects the transciptomes of any given brain cell
    population apart from pacemaker regions and neurons \parencite{doppSinglecellTranscriptomicsReveals2024}
    (Dopp et al 2024)

    \item The sleep homeostat monitors the sleep need that accumulates with the amount of time that an animal has been awake
    to determine the sleep drive, known as process S \parencite{doppSinglecellTranscriptomicsReveals2024}
    (Dopp et al 2024)

    \item We found that sleep/wakefulness states, sleep homeostasis and circadian rhythm have different transcriptional correlates depending
    on cell identity \parencite{doppSinglecellTranscriptomicsReveals2024}
    (Dopp et al 2024)

    \item Gene expression in most cell populations correlates either with process C or process S, with the exception of glial cells,
    which instead are affeced by both processes simultaneously \parencite{doppSinglecellTranscriptomicsReveals2024}
    (Dopp et al 2024)

    \item Cell types involved in sleep: five glial cell subtypes, Kenyon cells (KCs), clock neurons and cell types containing
    known sleep/wakefullness regulating curcuits such as non-protocerebral anterior medial (PAM) dopaminergic neurons (DANs),
    tyraminergic (Tyr) and octopaminergic (Oct) neurons, and ellipsoid body (EB) ring neurons. Another tell type involved in sleep,
    the dorsal fan-shaped bidy (dFB) neuron... \parencite{doppSinglecellTranscriptomicsReveals2024}
    (Dopp et al 2024)

    \item period (per) and timeless (tim) - expressed at higher levels in the early night compared to the early day;
    cryptochrome (cry) and Clock (Clk) mRNA - the opposite. Core clock genes are expressed and cycle specifically in Drosophila clock
    neurons and glia. While Clk expression is restricted to clock neurons and glia, other core clock genes per, tim and
    Cycle (Cyc) are expressed in more cell types. No cell type expresses Clk without expression of other core circadian genes is
    consistent with the notion that Clk is a circadian master regulator \parencite{doppSinglecellTranscriptomicsReveals2024}.
    (Dopp et al 2024)

    \item Molecular clock runs with different phases depending on the cell type \parencite{doppSinglecellTranscriptomicsReveals2024}
    (Dopp et al 2024)
    
    \item The expression of key clock genes in glia in addition to clock neurons, suggests that these cells are directly
    involved in circadian regulation of rhythmic behaviors, including sleep. \parencite{doppSinglecellTranscriptomicsReveals2024}
    (Dopp et al 2024)

    \item Glial cells stand out, this time by showing the highest number of cyclers (transcipts), especially considering that
    they typically express fewer genes than neurons \parencite{doppSinglecellTranscriptomicsReveals2024}
    (Dopp et al 2024)

    \item KCs and glia had the highest number of differentially expressed genes among annotated cell types.
    Considering that glia express the lowest number of genes among all cell types, these cells may be even more affected by
    sleep/wakefullness relative to their total expressed genes compared to neurons \parencite{doppSinglecellTranscriptomicsReveals2024}
    (Dopp et al 2024)

    \item Different cell populations have different sleep/wakefullness correlates \parencite{doppSinglecellTranscriptomicsReveals2024}
    (Dopp et al 2024)

    \item The sleep/wakefullness correlates in glia included metabolism related genes, genes involved in protein synthesis
    and homeostasis, and genes regulating the core circadian clock. In contrast, sleep/wakefullness correlates in KCs included many
    genes involved in axon and synapse development and function.
    (Dopp et al 2024)

    \item Cell types involved in process S: the four annotated clusters with the highest amount of sleep drive correlates
    were cell populations associated with sleep homeostasis. 121 correlates by dFB neurons. Similarly, Oct, Tyr and non-PAM DAN
    neurons each had more than 100 sleep drive corelates. In contrast, the related dopaminergic subtype of PAM neurons only had
    14 correlates.
    (Dopp et al 2024)

    \item (!!!!) In dFB neurons, we found many sleep drive correlates that were involved in synaptic formation and function. This is
    consistent with previous evidence linking neuronal activity of dFB neurons to levels of sleep pressure
    (Dopp et al 2024)

    \item R5 neurons contain only approximately 32 cells in an adult fly brain.
    (Dopp et al 2024)

    \item We found that one subcluster of EB ring neurons (ring\_2) had a substantial number of sleep drive correlates, while the
    other (ring\_1) showed only a few. We found a high number of sleep drive-correlated genes specifically in R5 neurons,
    while few to no genes were identified in the other two subclusters.
    (Dopp et al 2024)

    \item We found, that a gene encoding a potassium channel ether-a-go-go (eag) correlated negatively with sleep drive in R5 neurons.
    Potassium channels, including Eag, reduce neuronal excitability (\textbf{Brüggemann et al 1993})
    This is consistent with the finding that the neuronal activity of R5 increases with the levels of sleep drive
    (Liu et al 2016).
    (Dopp et al 2024)

    \item Clock neurons are a heterogeneous population consisting of four dorsal neuron subtypes (DN1a, DN1p, DN2 and DN3)
    and three lateral neuron (LN) types (LNv, LNd and LPN).
    In clock neuron subclusters... We identified significant sleep drive correlates only in the glutamatergic-positive
    cluster, not in the glutamatergic-negative subtype nor in DN3 neurons. This is consistent with findings that
    glutamatergic DN1p neurons are involved in sleep/wakefullness regulation. Among the sleep drive correlates of the DN1p cluster are
    two genes that regulate the function of a potassium channel: slowpoke-binding protein (slop) and dyschronic(dysc).
    This is consistent with the finding that the potassium channel (slowpoke) is important in glutamatergic DN1p neurons to regulate
    sleep quality.
    (Dopp et al 2024)

    \item In most cases we found that a given cell type was more affected by only one process (C or S). For example,
    dFB, Oct/Tyr, non-PAM DAN and R5 neurons had many sleep driv correlates but few circadian cyclers. On the
    other hand, cell types with many circadian correlates (examples listed in paper: some KC subtypes and PGs) had
    no sleep drive correlates
    (Dopp et al 2024)

    \item Two subtypes of EB ring neurons, that is R5 and ring\_B were affected by either processes in opposing ways, in
    accordance with previous findings, showing that R2/R4m neurons (probably part of ring\_B) received circadian timing
    information from clock neurons, while \textbf{R5 neurons themselves encoded the sleep homeostat (Liu et al 2016)}.
    (Dopp et al 2024)

    \item Both the number of cyclers and sleep drive correlates were high in all glia with the exception of PGs. This
    demonstrated that a simultaneous convergence of both circadian and homeostatic processes takes place in glial cells,
    as their transcriptome s affected by both
    (Dopp et al 2024)

    \item Flies with disrupted glial clock showed significantly reduced rebound sleep after SD compared to control flies
    (Dopp et al 2024)

    \item These data indicate that the glial clock is required for normal sleep homeostasis and suggest that processes S and C
    directly influence each other in glial cells to determin sleep-wake cycles
    (Dopp et al 2024)

    \item Our findings of sleep drive correlates illustrate the high specificity of the method to identify relevant sleep
    homeostasis regulating circuits, even when they are small (sub)populations. Therefore, other yet unannotated clusters
    with a high number of sleep drive correlates may also be involved in the homeostatic regulation of sleep.
    (Dopp et al 2024)

    \item We propose that the sleep–wake cycle affects the regulators of core clock genes in glia, and that the
    molecular clock in glia is required for sleep homeostasis. How do these two processes interact in glial cells?
    We and others previously demonstrated that glial Ca2+ signaling encodes the level of sleep needed18,59.
    In addition, Ca2+ signaling has an important role in regulating the oscillation of core clock genes, with
    many Ca2+ channels and transporters rhythmically expressed in mammalian clock neurons62. Thus, the reciprocal
    interaction between Ca2+ signaling and the molecular clock in glia may be the molecular substrate of the
    interaction of homeostatic and circadian processes to ultimately instruct downstream neurons and appropriate
    behavior.
    (Dopp et al 2024)

    % Dubowy and Sehgal 2017
    \item Circadian rhythms are daily rhythms in behavior or physiology that reoccur approximately
    every 24 hr. Circadian rhythms can be entrained by external environmental cues (i.e., light and
    temperature), but persist in the absence of these cues, with free-running periods that deviate
    slightly from the expected 24 hr in constant environmental conditions.
    \parencite{dubowyCircadianRhythmsSleep2017} (Dubowy and Sehgal 2017)

    \item In addition to eclosion and locomotor activity, circadian rhythms also drive other aspects
    of physiology and behavior, including sleep and an increasingly appreciated role in metabolism.
    Circadian control of all of these processes relies not only on the intracellular clock, but also
    on networks of cells that interact to influence circadian outputs.
    \parencite{dubowyCircadianRhythmsSleep2017} (Dubowy and Sehgal 2017)

    \item the two mRNAS (\textbf{per and tim}) cycle in phase and the PER and TIM proteins interact
    directly and affect their own transcription.
    \parencite{dubowyCircadianRhythmsSleep2017} (Dubowy and Sehgal 2017)

    \item while a number of period-altering mutations have been identified in per, tim, and the relevant
    kinases, we still do not have a complete understanding of how the 24 hr period is generated.
    \parencite{dubowyCircadianRhythmsSleep2017} (Dubowy and Sehgal 2017)

    \item Inherent to clock function in flies is a feedback loop in which the per and tim genes are
    expressed cyclically and negatively regulated by their own protein products (Hardin et al. 1990;
    Sehgal et al. 1995) The Neurospora clock is likewise comprised of a negative feedback loop
    generated through cyclic activity of the frequency gene product (Aronson et al. 1994).
    \parencite{dubowyCircadianRhythmsSleep2017} (Dubowy and Sehgal 2017)

    \item The transcriptional mechanisms discussed earlier do not just maintain rhythmic expression of
    per and tim, but also drive cycling of many output genes that contain enhancer elements recognized
    by CLK-CYC. For example, cycling of the Nlf-1 transcript results in time-of-day dependent
    changes in sodium leak current that, in turn, drive rhythmic neuronal activity in a subset
    of clock neurons (Flourakis et al. 2015).
    This example shows how clock-dependent cycling transcription can be the basis for rhythms in
    physiology that, in this case, likely contribute to behavior.
    however, clocks can also drive rhythms in physiology of other cells non-cellautonomously, by
    controlling signaling through neuronal circuits and release of secreted factors
    (Jaramillo et al. 2004; Cavey et al. 2016; Erion et al. 2016).
    \parencite{dubowyCircadianRhythmsSleep2017} (Dubowy and Sehgal 2017)

    \item In Drosophila, the core molecular clock components are coexpressed only in a restricted
    set of 150 neurons, which serve a function similar to the mammalian superchiasmatic nucleus
    (SCN) in regulating circadian rhythms in behavioral activity.
    \parencite{dubowyCircadianRhythmsSleep2017} (Dubowy and Sehgal 2017)

    \item Clock cells also exhibit cycling neuronal activity (Cao and Nitabach 2008; Sheeba et al.
    2008b; Flourakis et al. 2015), and the peak of neuronal activity, as reflected by intracellular 
    calcium levels, occurs at different phases for different groups (Liang et al. 2016). However, in 
    wild-type flies kept in conditions that approximate the natural world (including LD and early DD), 
    the core molecular clocks in nearly all groups of clock neurons cycle approximately in phase with 
    each other (Yoshii
    et al. 2009a; Roberts et al. 2015). The mechanisms through which the cycling of the molecular clock 
    generates complex and flexible behavioral outputs are thus probably related to these other properties
    of the clock circuit, rather than to differences in the cycling of the molecular clock itself.
    \parencite{dubowyCircadianRhythmsSleep2017} (Dubowy and Sehgal 2017)

    \item An important signaling molecule necessary for keeping clock cells synchronized with each 
    other and orchestrating behavioral activity is the neuropeptide PDF.
    Without PDF, the molecular clock in some groups of clock neurons run fast, while others dampen 
    as individual cells fall out of phase with each other, and yet others may slow 
    (Klarsfeld et al. 2004; Lin et al. 2004; Yoshii et al. 2009b; L. Zhang et al. 2010).
    \parencite{dubowyCircadianRhythmsSleep2017} (Dubowy and Sehgal 2017)


    \item A molecular clock in a subset of DN1 is sufficient to drive morning anticipatory activity
    in LD cycles, and, in certain temperature conditions, can drive evening anticipation as well 
    (Y. Zhang et al. 2010).
    \parencite{dubowyCircadianRhythmsSleep2017} (Dubowy and Sehgal 2017)

    \item Drosophila has been essential for identifying the key clock molecules and the negative 
    feedback loop mechanism that produces 24 hr cycles of gene expression and overt rhythms.
    \parencite{dubowyCircadianRhythmsSleep2017} (Dubowy and Sehgal 2017)

    \item Importantly, many genetic and molecular regulators of sleep are conserved across species 
    (reviewed in Crocker and Sehgal 2010). Thus, sleep in flies closely resembles sleep in other 
    organisms, and researchers can take advantage of the benefits of this small, genetically 
    tractable model organism to advance our understanding of the molecular neuroscience of sleep.
    \parencite{dubowyCircadianRhythmsSleep2017} (Dubowy and Sehgal 2017)

    \item Based on initial studies of arousal threshold, sleep in Drosophila is commonly defined 
    as a period of inactivity lasting 5 min or longer (Shaw et al. 2000; Huber et al. 2004; Andretic 
    and Shaw 2005).
    \parencite{dubowyCircadianRhythmsSleep2017} (Dubowy and Sehgal 2017)

    \newpage
    \item The Shaker potassium channel (Cirelli et al. 2005; Bushey et al. 2007) and its modulator 
    sleepless (Koh et al. 2008) were two early hits with extreme short-sleeping phenotypes from 
    large-scale genetic screens. Both genes are expressed throughout the fly brain (Wu et al. 2009),
    and neither of these phenotypes has been fully mapped to specific neuroanatomic loci, suggesting
    that they exert widespread effects on brain activity or metabolism that feed back onto sleep 
    regulation. Shaker is a voltage-gated potassium channel involved in membrane repolarization.
    sleepless is a Ly6 neurotoxin-like molecule that, in the years since its discovery as a 
    sleep regulator, has been found to promote Shaker expression and activity and inhibit 
    nicotinic acetylcholine (nAChR) function, such that loss of sleepless might lead to 
    increased neuronal activity through multiple mechanisms 
    (Wu et al. 2009; Shi et al. 2014; Wu et al. 2014).
    \parencite{dubowyCircadianRhythmsSleep2017} (Dubowy and Sehgal 2017)

    \item knocking down Shaker in sleep-promoting cells actually lengthens the inter-spike 
    interval and reduces neuronal activity in these populations to favor wake (Pimentel et al. 2016).
    \parencite{dubowyCircadianRhythmsSleep2017} (Dubowy and Sehgal 2017)

    \item Recent studies of sleepless have also suggested that it in part regulates sleep by 
    noncell-autonomously promoting metabolism of GABA in glia, perhaps also through its effect 
    on neural activity (Chen et al. 2014; Maguire et al. 2015).
    Shaker and sleepless thus both seem to interact in a nonstraightforward way with sleepregulatory
    genes and cells in the nervous system, and work with sleepless suggests a potential connection 
    between neuronal activity and metabolism of neurotransmitters, although the details of this 
    connection remain unclear.
    \parencite{dubowyCircadianRhythmsSleep2017} (Dubowy and Sehgal 2017)

    \item The mushroom body consists primarily of 2000 Kenyon cells, most of which receive input 
    from an average of six stochastically connected projection neurons, with each projection neuron 
    encoding input from a single type of odorant receptor neuron.
    \parencite{dubowyCircadianRhythmsSleep2017} (Dubowy and Sehgal 2017)

    \item The mushroom body was also the first neuroanatomic structure identified as a regulator of
     sleep in Drosophila (Joiner et al. 2006; Pitman et al. 2006).
     \parencite{dubowyCircadianRhythmsSleep2017} (Dubowy and Sehgal 2017)
    
    \item mushroom body contains both sleep-promoting and wake-promoting cells
    \parencite{dubowyCircadianRhythmsSleep2017} (Dubowy and Sehgal 2017)
    
    \item Perhaps the strongest parallel between mammalian and Drosophila sleep regulation
    identified so far is the strong wake promoting effects of the monoamine neurotransmitters
    dopamine and octopamine
    \parencite{dubowyCircadianRhythmsSleep2017} (Dubowy and Sehgal 2017)

    \item dopaminergic neurons are strongly wake-promoting when activated 
    (Shang et al. 2011; Liu et al. 2012)
    \parencite{dubowyCircadianRhythmsSleep2017} (Dubowy and Sehgal 2017)

    \item Thermogenetic activation of the ExF/2 neurons in the dorsal fanshaped body, a 
    region of the central complex, is strongly sleep-promoting (Donlea et al. 2011). Sleep 
    deprivation changes the electrophysiologic properties of these neurons to favor activity,
    suggesting they may play a role in output of homeostatic sleep signals (Donlea et al. 2014).
    \parencite{dubowyCircadianRhythmsSleep2017} (Dubowy and Sehgal 2017)

    \item Sleep homeostasis is often conceptualized as a continuous build-up of sleep need over
    periods of wakefulness and dissipation over periods of sleep, such that the same mechanisms
    should be invoked both when flies are spontaneously waking and during periods of forced 
    wakefulness (sleep deprivation). However, recent work in Drosophila has called this view 
    into question.
    \parencite{dubowyCircadianRhythmsSleep2017} (Dubowy and Sehgal 2017)

    \item a number of genetic perturbations have been identified that specifically affect sleep
    after sleep deprivation with little to no effect on baseline sleep, suggesting that sleep 
    following sleep deprivation is produced by an independent mechanism (Seugnet et al. 2011b;
    Seidner et al. 2015; Thimgan et al. 2015; Dubowy et al. 2016; Liu et al. 2016)
    \parencite{dubowyCircadianRhythmsSleep2017} (Dubowy and Sehgal 2017)

    \item electrophysiology suggests that the sleep-promoting dorsal fan-shaped body neurons have
    reduced input resistance and reduced membrane time constants, suggesting greater activity, 
    following sleep deprivation (Donlea et al. 2014); as discussed previously, this brain area 
    is well-positioned to act as an integrator or output for multiple sleep regulatory signals, 
    including, it seems, the response to sleep deprivation.
    \parencite{dubowyCircadianRhythmsSleep2017} (Dubowy and Sehgal 2017)

    \item These neurons were initially of interest because they produce a persistent sleep-promoting 
    signal when thermogenetically activated; while no changes in sleep are reported at the time of 
    activation, which can be as short as 30 min, a dramatic reboundlike increase in sleep is observed
    for the next 12 hr.
    \parencite{dubowyCircadianRhythmsSleep2017} (Dubowy and Sehgal 2017)

    \item showed greater synapse number and size for R2 neurons after sleep deprivation, and 
    genetic manipulations that block this plasticity partially block sleep rebound.
    \parencite{dubowyCircadianRhythmsSleep2017} (Dubowy and Sehgal 2017)

    \item The manipulations of R2 neurons that affect sleep rebound have no effect on sleep at 
    baseline, however, again supporting the idea that regulation of the
    homeostatic response to sleep deprivation is mechanistically different from the regulation
    of baseline sleep.
    \parencite{dubowyCircadianRhythmsSleep2017} (Dubowy and Sehgal 2017)

    \item Function of the sleep: bidirectional relationship with learning and memory 
    (both short ad long-term in drosophila),
    social behavior
    \parencite{dubowyCircadianRhythmsSleep2017} (Dubowy and Sehgal 2017)

    \item One hypothesis based on this data, put forth by Tononi and Cirelli (2006), proposes 
    that global synaptic downscaling occurs during sleep to counteract overpotentiation that 
    might occur during wake.
    Work from these authors shows evidence that, broadly and within specific circuits of the 
    adult fly brain, synaptic markers increase after wake or sleep deprivation and decrease 
    following sleep, suggesting changes in the number or size of synapses 
    (Gilestro et al. 2009; Bushey et al. 2011).
    \parencite{dubowyCircadianRhythmsSleep2017} (Dubowy and Sehgal 2017)

    % Shafer and Kenee 2021
    \item This research has revealed that the functions and neural principles of sleep regulation are largely conserved from flies to mammals
    \parencite{shaferRegulationDrosophilaSleep2021} (Shafer and Kenee 2021)

    \item sleep is required for numerous cellular processes that are critical for biological function, including brain-wide regulation of synaptic strength and immune function.
    Sleep also impacts cellular metabolism and communication between the brain and the peripheral organs it controls, as well as complex cognitive tasks, including learning and memory.
    \parencite{shaferRegulationDrosophilaSleep2021} (Shafer and Kenee 2021)

    \item our understanding of the neural and molecular basis of sleep regulation and how sleep is influenced by an animal’s internal and external environment remains incomplete.
    \parencite{shaferRegulationDrosophilaSleep2021} (Shafer and Kenee 2021)

    \item (Drosophila studies) These efforts have revealed neural principles and molecular mechanisms that are conserved within the animal kingdom, including widespread functional conservation of the neurotransmitter systems that are primary drivers of sleep and wakefulness in mammals
    \parencite{shaferRegulationDrosophilaSleep2021} (Shafer and Kenee 2021)

    \item The defining behavioral features of sleep — prolonged periods of quiescence, reduced responsiveness to sensory stimuli, species-specific posture, recovery sleep following deprivation, and rapid reversibility (which distinguishes sleep from hibernation or coma) — characterize both vertebrate and invertebrate sleep
    Remarkably, invertebrates with relatively simple nervous systems, including the jellyfish Cassiopea and the roundworm Caenorhabditis elegans, meet these simple criteria, suggesting deep evolutionary conservation of sleep
    \parencite{shaferRegulationDrosophilaSleep2021} (Shafer and Kenee 2021)

    \item flies that are immobile for five-minute periods or longer are considered to be sleeping because inactivity for this duration is associated with the aforementioned behavioral characteristics of sleep
    \parencite{shaferRegulationDrosophilaSleep2021} (Shafer and Kenee 2021)

    \item central complex is critical for the regulation of sleep duration and homeostasis. The central complex contains many neuron types, including dorsal fan-shaped body neurons, helicon cells, and ellipsoid body ring neurons, all of which have been implicated in sleep regulation.
    Dorsal fan-shaped body neurons are sleep-promoting neurons that receive inhibitory input from wake-promoting dopamine neurons.
    At the cellular level, expression levels of multiple ion channels within the dorsal fan-shaped body modulate neuronal activity and thereby promote sleep or wakefulness.
    \parencite{shaferRegulationDrosophilaSleep2021} (Shafer and Kenee 2021)

    \item Specifically, a class of neurons termed the R5 neurons (originally termed R2 neurons) fire synchronously during sleep in a manner similar to neuronal firing during mammalian slow-wave sleep in the mammalian neocortex.
    The neural activity and synaptic connectivity of these neurons is modified in accordance with sleep debt, sensory information, and circadian input, thereby driving recovery sleep.
    The sleep-promoting dorsal fan-shaped body neurons interact with the R5 neurons through a third population of
    neurons — helicon cells — that are activated by visual stimuli.
    \parencite{shaferRegulationDrosophilaSleep2021} (Shafer and Kenee 2021)

    \item The mushroom bodies are paired structures in the central brain comprising 2,500 intrinsic neurons, termed Kenyon cells, that were initially studied for their roles in learning and memory.
    Mushroom bodies play a central role in modulating sleep, likely through a circuit that integrates sensory information, arousal, and homeostatic sleep drive.
    Broad activation of the mushroom bodies promotes sleep and that sleep is reduced in flies with silenced or ablated mushroom bodies, revealing a critical role for this structure in governing sleep duration
    \parencite{shaferRegulationDrosophilaSleep2021} (Shafer and Kenee 2021)

    \item dynamic changes in local field potentials and their associated changes in sleeplike behavior require gap junctions, highlighting the presence of complex network-wide signatures that may not be mimicked by the optogenetic and thermogenetic manipulations available in the fly.
    \parencite{shaferRegulationDrosophilaSleep2021} (Shafer and Kenee 2021)

    \item It is clear that more work is necessary to understand how the various sleep/wake-modulating networks interact with one another. A recent connectomic reconstruction of the fly brain using serial electron microscopy will likely provide the synaptic resolution necessary to reveal the micro-anatomical connections between neurons within and between sleep networks, including those that regulate mushroom body and central complex physiology.
    \parencite{shaferRegulationDrosophilaSleep2021} (Shafer and Kenee 2021)

    \item As the field continues to identify neurons and networks that govern sleep, it will be critical to determine the physiological changes that occur within these networks during transitions between wakefulness and sleep, as well as during the various stages of sleep.
    \parencite{shaferRegulationDrosophilaSleep2021} (Shafer and Kenee 2021)

    \item The functions and subtypes of Drosophila glia are broadly similar to those in mammals and are thought to be evolutionarily conserved.
    Several subtypes of glia have been implicated in the control of sleep and circadian modulation of activity.
    \parencite{shaferRegulationDrosophilaSleep2021} (Shafer and Kenee 2021)

    \item Together, these observations reveal that astrocytes are critical modulators of sleep and the neural networks that govern it.
    \parencite{shaferRegulationDrosophilaSleep2021} (Shafer and Kenee 2021)

    \item Taurine is a GABA agonist that promotes sleep in flies, raising the possibility that ensheathing glia promote wakefulness by removing synaptic taurine.
    \parencite{shaferRegulationDrosophilaSleep2021} (Shafer and Kenee 2021)

    \item One likely function of the circadian system is to suppress the onset of sleep during times of the day when sleep pressure is high but when sleep would be dangerous or maladaptive, thereby delaying it until the appropriate time.
    For nearly four decades, the sleep field has conceptualized sleep through the two-process model, which posits that sleep is governed by interactions between homeostatic and circadian control processes.
    In this model, sustained wakefulness produces a homeostatic sleep pressure and increased slow-wave sleep, while a circadian system sets the thresholds for sleep pressure that correspond to sleep or wakefulness.
    \parencite{shaferRegulationDrosophilaSleep2021} (Shafer and Kenee 2021)

    \item Under light/dark cycles, Drosophila displays a bimodal pattern of locomotor activity with a rather narrow burst of activity commencing just before dawn and a larger bout of activity that commences several hours before dusk.
    Daytime sleep is characterized by shorter, more frequent sleep bouts compared with nighttime sleep68 and arousal thresholds are higher for nighttime sleep19. Thus, nighttime sleep appears to be significantly deeper than daytime sleep in flies.
    \parencite{shaferRegulationDrosophilaSleep2021} (Shafer and Kenee 2021)

    \item Approximately 150 neurons support the daily cycling of a core set of circadian clock genes.
    \parencite{shaferRegulationDrosophilaSleep2021} (Shafer and Kenee 2021)

    \item neural firing by a given class of clock neuron is expected to have either a sleep- or wake-promoting function, although some studies suggest that the effects of clock neurons on sleep might not be so simple.
    \parencite{shaferRegulationDrosophilaSleep2021} (Shafer and Kenee 2021)

    \item One possible explanation for these conflicting results could be that the effects of a specific set of clock neurons on sleep vary across the circadian cycle.
    Indeed, recent work on the DN1ps revealed that these neurons govern sleep quality not through a simple increase or decrease of activity, but rather by changes in their patterns of firing.
    Thus, a clock neuron firing in a particular mode might shift the balance toward sleep, whereas a distinct mode of firing by the same neuron might shift the balance toward wakefulness.
    \parencite{shaferRegulationDrosophilaSleep2021} (Shafer and Kenee 2021)

    \item As in humans, the fly’s circadian system maintains tight control over sleep even in the presence of significant sleep deprivation: flies that are allowed recovery sleep after deprivation maintain clock-timed decreases in sleep before dawn and dusk and display normal timing and amounts of nighttime sleep.
    Thus, the circadian clock network likely provides modulatory input into networks mediating homeostatic sleep drive. Indeed, several recent studies have converged on a discrete target of the clock neuron network: ring neurons of the ellipsoid body, a neuronal locus of sleep homeostasis in the central complex.
    \parencite{shaferRegulationDrosophilaSleep2021} (Shafer and Kenee 2021)

    \item Four independent groups recently identified neural pathways linking specific cell types within the circadian clock network to the ellipsoid bodies of the central complex.
    \parencite{shaferRegulationDrosophilaSleep2021} (Shafer and Kenee 2021)

    \item For example, the DN1p-to-AOTU connection appeared to be a wake-promoting pathway in one study102 and a sleep-promoting pathway in the
    other103.
    Notwithstanding the contrasting conclusions of these studies about the roles of different clock output neurons in regulating sleep, their 
    consensus regarding the key roles of the ring neurons is powerful evidence that sleep and circadian control converge in the ellipsoid body.
    \parencite{shaferRegulationDrosophilaSleep2021} (Shafer and Kenee 2021)

    \item The two-process model of sleep clearly does not hold when an animal faces acute physiological challenges or fleeting opportunities to mate. Sleep is strongly regulated by environmental factors, including the presence of conspecifics and food availability.
    \parencite{shaferRegulationDrosophilaSleep2021} (Shafer and Kenee 2021)

    \item Sleep represents a trade-off with time spent consuming or locating food. Starved flies, like starved rodents, suppress sleep and become hyperactive.
    \parencite{shaferRegulationDrosophilaSleep2021} (Shafer and Kenee 2021)

    \item Sleep is also potently modulated by social experience. Flies reared in isolation sleep less during the daytime, revealing long-term plasticity in sleep modulation that is dependent on canonical memory pathways.
    \parencite{shaferRegulationDrosophilaSleep2021} (Shafer and Kenee 2021)

    \item Finally, sleep can directly impact social behaviors. For example, sleep loss results in reduced aggression and courtship in flies.
    \parencite{shaferRegulationDrosophilaSleep2021} (Shafer and Kenee 2021)

    \item the study of sleep in flies has great potential to shed light on the fundamental biology of sleep and emerging problems related to the role of sleep in human health and disease.
    \parencite{shaferRegulationDrosophilaSleep2021} (Shafer and Kenee 2021)

    \item Drosophila will continue to be an important model system for understanding the relationships between the environment, the brain, and sleep.
    \parencite{shaferRegulationDrosophilaSleep2021} (Shafer and Kenee 2021)

    % Andreani et al 2022
    \item In the R5 ellipsoid body sleep homeostat, we identified elevated morning expression of activity dependent
    and presynaptic gene expression as well as the presynaptic protein BRUCHPILOT consistent with regulation by clock
    circuits. These neurons also display elevated calcium levels in response to sleep loss in the morning, but not the
    evening consistent with the observed time-dependent sleep rebound.
    \parencite{andreaniCircadianProgrammingEllipsoid2022} (Andreani et al 2022)

    \item The circadian process, via phased activity changes in central pacemaker neurons, times and consolidates sleep-wake (Patke et al., 2020). The less well-understood homeostatic process, often assayed after extended sleep deprivation, promotes sleep length, depth, and amount as a function of the duration and intensity of prior waking experience (Deboer and Tobler, 2000; Franken et al., 1991; Huber et al., 2004; Werth et al., 1996). Sleep homeostasis is thought to be mediated by the accumulation of various wake-dependent factors, such as synaptic strength (Tononi and Cirelli, 2014), which are subsequently dissipated with sleep.
    \parencite{andreaniCircadianProgrammingEllipsoid2022} (Andreani et al 2022)

    \item While homeostatic drive persists in the absence of a functioning circadian clock (Tobler et al., 1983), homeostatic drive can be modulated by the circadian clock.
    \parencite{andreaniCircadianProgrammingEllipsoid2022} (Andreani et al 2022)

    \item Yet the molecular and circuit mechanisms by which the circadian clock modulates sleep homeostasis remain unclear.
    \parencite{andreaniCircadianProgrammingEllipsoid2022} (Andreani et al 2022)

    \item Drosophila, a well-established model for investigating the molecular and neural basis of circadian rhythms and sleep.
    \parencite{andreaniCircadianProgrammingEllipsoid2022} (Andreani et al 2022)

    \item Sleep is characterized by quiescence, increased arousal thresholds, changes in neuronal activity, and circadian and homeostatic regulation (Campbell and Tobler, 1984). Flies display each of these hallmarks (Hendricks et al., 2000; Shaw et al., 2000; van Alphen et al., 2013) and have simple, well-characterized circadian and sleep neural networks (Dubowy and Sehgal, 2017; Shafer and Keene, 2021).
    \parencite{andreaniCircadianProgrammingEllipsoid2022} (Andreani et al 2022)

    \item About 150 central pacemaker neurons that express molecular clocks (Dubowy and Sehgal, 2017). Of these,
    four small ventral lateral neurons (sLNvs) (per hemisphere) that express pigment dispersing factor (PDF) are
    necessary for driving morning activity in anticipation of lights on and exhibit peak levels of calcium around
    dawn (~ZT0) (Grima et al., 2004; Liang et al., 2019; Liang et al., 2017; Stoleru et al., 2004).
    The dorsal lateral neurons (LNds) and a 5th PDF- sLNv are necessary for evening anticipation of lights off and
    show a corresponding evening calcium peak (ZT8ZT10)
    (Gossan et al., 2014; Grima et al., 2004; Liang et al., 2019; Liang et al., 2017; Stoleru et al., 2004).
    The posterior DN1 (DN1ps) consist of glutamate-positive (Glu+) subsets necessary for morning anticipation
    and Glu- necessary for evening anticipation under low light conditions (Chatterjee et al., 2018).
    Lateral posterior neurons (LPN) are not necessary for anticipation but are uniquely sensitive to temperature cycling
    (Miyasako et al., 2007). Specific pacemaker subsets have been linked to wake promotion
    (PDF+ large LNv (Chung et al., 2009; Parisky et al., 2008; Sheeba et al., 2008), diuretic hormone 31 (DH31+)
    DN1ps [Kunst et al., 2014]) and sleep promotion (Glu+ DN1ps (Guo et al., 2016),
    Allostatin A+ LPNs [Ni et al., 2019]), independently of their clock functions.
    How these neurons regulate homeostatic sleep drive itself remains unsettled.
    \parencite{andreaniCircadianProgrammingEllipsoid2022} (Andreani et al 2022)

    \item The sLNvs and LNds appear to communicate to R5 EB neurons through an intermediate set of dopaminergic PPM3 neurons based largely on correlated calcium oscillations (Liang et al., 2019)
    \parencite{andreaniCircadianProgrammingEllipsoid2022} (Andreani et al 2022)

    \item The anterior projecting subset of DN1ps provide sleep promoting input to other EB neurons (R2/R4M) via tubercular bulbar (TuBu) interneurons (Guo et al., 2018; Lamaze et al., 2018). Activation of a subset of these TuBu neurons synchronizes the activity of the R5 neurons which is important for sleep maintenance (Raccuglia et al., 2019)
    \parencite{andreaniCircadianProgrammingEllipsoid2022} (Andreani et al 2022)

    \item Critically, the R5 neurons are at the core of sleep homeostasis in Drosophila (Liu et al., 2016).
    \parencite{andreaniCircadianProgrammingEllipsoid2022} (Andreani et al 2022)

    \item Extended sleep deprivation (12–24 hr) elevates calcium, the critical presynaptic protein BRUCHPILOT (BRP), and action potential firing rates in R5 neurons. The changes in BRP in this region not only reflect increased sleep drive following SD but also knockdown (KD) of brp in R5 decreases rebound (Huang et al., 2020) suggesting it functions directly in regulating sleep homeostasis.
    \parencite{andreaniCircadianProgrammingEllipsoid2022} (Andreani et al 2022)

    \item R5 neurons stimulate downstream neurons in the dorsal fan-shaped body (dFB), which are sufficient to produce sleep (Donlea et al., 2014; Donlea et al., 2011; Liu et al., 2016).
    \parencite{andreaniCircadianProgrammingEllipsoid2022} (Andreani et al 2022)

    \item Yet how the activity of key clock neurons are integrated with signals from the R5 homeostat to determine sleep drive remains unclear.
    \parencite{andreaniCircadianProgrammingEllipsoid2022} (Andreani et al 2022)

    \item Moreover, homeostatic R5 EB neurons integrate circadian timing and homeostatic drive; we demonstrate that activity dependent and presynaptic gene expression, BRP expression, neuronal output, and wake sensitive calcium levels are all elevated in the morning compared to the evening, providing an underlying mechanism for clock programming of time-of-day dependent homeostasis.
    \parencite{andreaniCircadianProgrammingEllipsoid2022} (Andreani et al 2022)

    \item After sleep deprivation, flies display a robust sleep rebound throughout the 4.5 hr rebound period in the morning while evening rebound is suppressed (Figure 2a).
    \parencite{andreaniCircadianProgrammingEllipsoid2022} (Andreani et al 2022)

    \item we determined if these effects persist under constant darkness (DD). We observed elevated rebound in the morning (CT2.5) relative to the evening (CT10.5), indicating that these differences are not dependent on light (Figure 2c). Altogether, these data suggest that homeostatic rebound sleep is strongly modulated by the internal clock.
    \parencite{andreaniCircadianProgrammingEllipsoid2022} (Andreani et al 2022)

    \item TuBu neurons had no effect on rebound
    \parencite{andreaniCircadianProgrammingEllipsoid2022} (Andreani et al 2022)

    \item Blocking R5 synaptic output also reduced rebound in both morning and evening
    \parencite{andreaniCircadianProgrammingEllipsoid2022} (Andreani et al 2022)

    \item R5 neurons promote sleep in response to deprivation by activating the sleep promoting dFB (Liu et al., 2016).
    \parencite{andreaniCircadianProgrammingEllipsoid2022} (Andreani et al 2022)

    \item We were surprised to find that neither morning nor evening SD had much of an effect on gene expression in the R5 neurons
    \parencite{andreaniCircadianProgrammingEllipsoid2022} (Andreani et al 2022)

    \item In stark contrast, comparisons of morning and evening timepoints with or without sleep deprivation produces 46–128 differentially expressed genes. Notably, this time-of-day dependent regulation does not appear to be driven by core clock genes in these neurons.
    \parencite{andreaniCircadianProgrammingEllipsoid2022} (Andreani et al 2022)

    \item We also observed significant upregulation of genes involved in ionic transport across the plasma membrane, including para, a voltage-gated sodium channel (Catterall, 2000; Loughney et al., 1989), and CG5890, a predicted potassium channel-interacting protein (KChIP) (Figure 8e and g). Mammalian KChIPs have been shown to interact with voltage-gated potassium channels, increasing current density and conductance and slowing inactivation (An et al., 2000). Two sodium:potassium/calcium antiporters, CG1090 and Nckx30C, were also upregulated (Figure 8e and g). These antiporters function primarily in calcium homeostasis by using extracellular sodium and intracellular potassium gradients to pump intracellular calcium out of the cell when calcium levels are elevated (Haug-Collet et al., 1999).
    \parencite{andreaniCircadianProgrammingEllipsoid2022} (Andreani et al 2022)

    \item Amongst the most significantly upregulated genes in our dataset, we found six genes that were previously identified as activity-regulated genes in Drosophila (ARGs; sr, Cdc7 (also known as l(1)G0148), CG8910, CG14186, CG17778, hr38)
    \parencite{andreaniCircadianProgrammingEllipsoid2022} (Andreani et al 2022)

    \item R5 neurons are assembling a greater number of mature active zones for neuronal output. Upregulation of para and the predicted KChIP CG5890, which should increase the voltage-gated conductance of sodium and potassium ions across the membrane, supports the idea that R5 neurons may be primed for greater action potentials in the morning. Upregulation of the two sodium:potassium/calcium antiporters suggests that intracellular calcium levels are elevated in the morning, again consistent with the idea that these neurons are more active in the morning.
    \parencite{andreaniCircadianProgrammingEllipsoid2022} (Andreani et al 2022)

    \item SD/extended wake results in the upregulation of many synaptic proteins (Gilestro et al., 2009). Most notable is the presynaptic scaffolding protein BRP, which is important for synaptic release (Matkovic et al., 2013), and is upregulated in the R5 neurons following 12 hr of SD (Liu et al., 2016). KD of brp in R5 neurons decreases rebound response to SD (Huang et al., 2020), suggesting that it is necessary for accumulating and/or communicating homeostatic drive. We hypothesized that differences in the propensity for R5 to induce sleep rebound in the morning/evening may be due to changes in synaptic strength that can be observed by tracking levels of BRP.
    \parencite{andreaniCircadianProgrammingEllipsoid2022} (Andreani et al 2022)

    \item The calcium concentration in R5 neurons increases following twelve hours of SD, suggesting that extended wakefulness can induce calcium signaling in these neurons. Blocking the induction of calcium greatly reduces rebound, supporting a critical role for calcium signaling in behavioral output (Liu et al., 2016). Furthermore, R5 neurons display morning and evening cell-dependent peaks in calcium activity across the course of the day indicating that calcium is also modulated by the clock network (Liang et al., 2019)
    \parencite{andreaniCircadianProgrammingEllipsoid2022} (Andreani et al 2022)

    \item we observed very little gene expression significantly altered in response to our 2.5 hr sleep deprivation. On the other hand, we did identify elevated expression of activity-dependent and presynaptic genes in the morning independent of sleep deprivation. Consistent with this finding, we also observe elevated levels of the presynaptic protein BRP that is absent in the absence of Clk. These baseline changes are accompanied by an elevated calcium response to sleep deprivation in the morning mirroring the enhanced behavioral rebound in the morning.
    \parencite{andreaniCircadianProgrammingEllipsoid2022} (Andreani et al 2022)

    \item We observed significant upregulation of several genes involved in synaptic transmission (Syx1a, Rim, nSyb, unc-104, Srpk79D, para, CG5890) evincing a permissive active state for R5 neurons in the morning. This is accompanied by elevated levels of the key presynaptic protein BRP in the morning compared to evening.
    \parencite{andreaniCircadianProgrammingEllipsoid2022} (Andreani et al 2022)

    % Pimentel et al 2016
    \item In Drosophila, a crucial component of the machinery for sleep homeostasis is a cluster of neurons innervating the  dorsal fan-shaped body (dFB) of the central complex2,3. Artificial  activation of these cells induces sleep2, whereas reductions in  excitability cause insomnia3,4.
    \parencite{pimentelOperationHomeostaticSleep2016} (Pimentel et al 2016)

    \item dFB neurons in sleep-deprived flies tend to be electrically active, with high input resistances and long membrane time constants, while neurons in rested flies tend to be  electrically silent3.
    \parencite{pimentelOperationHomeostaticSleep2016} (Pimentel et al 2016)

    \item Here we demonstrate state switching by dFB neurons, identify dopamine as a neuromodulator that operates the switch, and delineate the  switching mechanism. Arousing dopamine4–8 caused transient hyperpolarization of dFB neurons within tens of milliseconds and lasting excitability suppression within minutes. Both effects were transduced by Dop1R2 receptors and mediated by potassium conductances. The switch to electrical silence involved the downregulation of voltage-gated A-type currents carried by Shaker and Shab, and the upregulation of voltage-independent leak currents through a two-pore-domain potassium channel that we term Sandman. Sandman is encoded by the CG8713 gene and translocates to the plasma membrane in response to dopamine. dFB-restricted interference with the expression of Shaker or Sandman decreased or increased sleep, respectively, by slowing the repetitive discharge of dFB neurons in the ON state or blocking their entry into the OFF state.
    \parencite{pimentelOperationHomeostaticSleep2016} (Pimentel et al 2016)

    \item Illumination at 630 nm, sustained for 1.5 s to release a bolus of dopamine (Extended Data Fig. 1), effectively stimulated locomotion (32/38 trials; Fig. 1a, b). dFB neurons paused in successful (but not in unsuccessful) trials (Fig. 1a, b), and their membrane potentials dipped by 2–13 mV (7.50 ± 0.56 mV; mean ± standard error of the mean (s.e.m.)) below the baseline during tonic activity.
    \parencite{pimentelOperationHomeostaticSleep2016} (Pimentel et al 2016)

    \item The tight correlation between the suppression of dFB neuron spiking and the initiation of movement
    \parencite{pimentelOperationHomeostaticSleep2016} (Pimentel et al 2016)

    \item Flies with enhanced dopaminergic transmission exhibit a short-sleeping phenotype that requires the presence of a D1-like  receptor in dFB neurons
    \parencite{pimentelOperationHomeostaticSleep2016} (Pimentel et al 2016)

    \item Loss of Dop1R2 increased sleep during the day and the late hours of the night, by prolonging sleep bouts without affecting their frequency
    \parencite{pimentelOperationHomeostaticSleep2016} (Pimentel et al 2016)

    \item Like optogenetically stimulated secretion, focal application of dopamine hyperpolarized the cells and suppressed their spiking
    \parencite{pimentelOperationHomeostaticSleep2016} (Pimentel et al 2016)

    \item While a single pulse of dopamine transiently hyperpolarized dFB neurons and inhibited their spiking, prolonged dopamine applications switched the cells from electrical excitability (ON) to quiescence (OFF)
    \parencite{pimentelOperationHomeostaticSleep2016} (Pimentel et al 2016)

    \item dFB neurons in the ON state expressed two types of potassium  current: voltage-dependent A-type16 and voltage-independent non-A-type currents. d Extended Data Fig. 6a-c). The current–voltage (I-V) relation of IA resembled that of Shaker, the  prototypical A-type channel.
    Non-A-type currents showed weak outward rectification with a reversal potential of -80 mV (Fig. 3e, g), consistent with potassium as the permeant ion, and no inactivation
    \parencite{pimentelOperationHomeostaticSleep2016} (Pimentel et al 2016)

    \item Switching the neurons OFF changed both types of potassium current. IA diminished by one-third (Fig. 3e, f), whereas Inon-A nearly quadrupled when quantified between resting potential and spike threshold.
    dFB neurons thus upregulate IA in the sleep-promoting ON state (Fig. 3e, f). When dopamine switches the cells OFF, voltage-dependent currents are attenuated and leak currents augmented
    depletion of voltage-gated A-type (KV) channels (which predominate in the ON state) should tip the cells towards the OFF state; conversely, loss of leak channels (which predominate in the OFF state) should favour the ON state.
    \parencite{pimentelOperationHomeostaticSleep2016} (Pimentel et al 2016)

    % Fernandez-Chiappe et al 2021
    \item small LNvs (sLNvs) are bursting neurons, and Ih is necessary to achieve the high-frequency bursting firing pattern
    characteristic of both types of LNvs in females.
    \parencite{fernandez-chiappeHighFrequencyNeuronalBursting2021} (Fernandez-Chiappe et al 2021)

    \item Circadian (circa: around, diem: day) rhythms are biological rhythms with a period of ;24 h that have evolved in essentially all organisms.
    \parencite{fernandez-chiappeHighFrequencyNeuronalBursting2021} (Fernandez-Chiappe et al 2021)

    \item The large LNvs (lLNvs), on the other hand, are highly relevant for arousal and the PDF they release provides wake promoting functions (Parisky et al., 2008; Shang et al., 2008; Sheeba et al., 2008a).
    \parencite{fernandez-chiappeHighFrequencyNeuronalBursting2021} (Fernandez-Chiappe et al 2021)

    \item we demonstrate that perturbing Ih causes a decrease in the frequency of LNvs bursting that is accompanied by a reduction in PDF immunoreactivity and in the complexity of sLNv axonal termini.
    \parencite{fernandez-chiappeHighFrequencyNeuronalBursting2021} (Fernandez-Chiappe et al 2021)

    % Donlea et al 2018
    \item We report that dFB neurons communicate via inhibitory transmitters, including allatostatin-A (AstA), with interneurons connecting the superior arch with the ellipsoid body of the central complex. These ‘‘helicon cells’’ express the galanin receptor homolog AstA-R1, respond to visual input, gate locomotion, and are inhibited by AstA, suggesting that dFB neurons promote rest by suppressing visually guided movement.
    \parencite{donleaRecurrentCircuitryBalancing2018} (Donlea et al 2018)

    \item Helicon cells provide excitation to R2 neurons of the ellipsoid body, whose activity-dependent plasticity signals rising sleep pressure to the dFB.
    \parencite{donleaRecurrentCircuitryBalancing2018} (Donlea et al 2018)

    \item The behavioral hallmarks of sleep are manifold. They include inactivity, reduced responsiveness to external stimuli, rapid reversibility, and homeostatic rebound after sleep loss. Any sleep control system must therefore fulfill a multitude of functionsblocking locomotor activity, gating sensory pathways, inhibiting arousal systems, relieving sleep pressure—and perhaps also directly influence processes germane to a fundamental purpose of sleep, be it metabolic recovery (Vyazovskiy and Harris, 2013; Walker et al., 1979), memory consolidation (Wilson and McNaughton, 1994), or synaptic scaling (Tononi and Cirelli, 2003).
    \parencite{donleaRecurrentCircuitryBalancing2018} (Donlea et al 2018)

    \item two dozen neurons (of a total of 100,000 in the brain) suffices to induce sleep in Drosophila (Donlea et al., 2011).
    \parencite{donleaRecurrentCircuitryBalancing2018} (Donlea et al 2018)

    \item We find that dFB neurons induce sleep via a range of inhibitory transmitters that include the neuropeptide allatostatin-A (AstA). Among the targets of AstA are a group of interneurons of the central complex that we term helicon cells. These neurons are inhibited by sleep-promoting AstA, excited by visual input, permissive for locomotion, and presynaptic to R2 ring neurons of the ellipsoid body, whose activity has been linked to the accumulation of sleep debt (Liu et al., 2016). dFB-mediated inhibition of helicon cells may thus account for three cardinal features of sleep: elevated visual thresholds, immobility, and the dissipation of sleep need.
    \parencite{donleaRecurrentCircuitryBalancing2018} (Donlea et al 2018)

    \item A Sleep-Promoting Signal from dFB Neurons
    \parencite{donleaRecurrentCircuitryBalancing2018} (Donlea et al 2018)

    \item Helicon Cells: Targets of dFB Neurons with Projections to the Ellipsoid Body
    \parencite{donleaRecurrentCircuitryBalancing2018} (Donlea et al 2018)

    \item dFB Neurons Inhibit Helicon Cells and Their Visual Responses
    \parencite{donleaRecurrentCircuitryBalancing2018} (Donlea et al 2018)

    \item helicon cells (Figure 4A) were found in one of two states: a DOWN state characterized by the near absence of spikes (firing rate < 1 Hz) and an UP state in which the neurons fired persistently, with occasionally metronomic precision, at rates of 16.9 ± 3.6 Hz (Figures 4B and 4C). An average voltage difference of 10.9 ± 2.3 mV (mean ± SEM, n = 10 cells) separated the membrane potential baselines of the two states.
    \parencite{donleaRecurrentCircuitryBalancing2018} (Donlea et al 2018)

    \item Spontaneous movements were initiated with approximately 4-fold higher probability when the recorded cell was in the UP rather than in the DOWN state (Figure 4E). Together, these results suggest that helicon cells play a permissive role in visually guided movement.
    \parencite{donleaRecurrentCircuitryBalancing2018} (Donlea et al 2018)

    \item Helicon Cells Gate Locomotion
    \parencite{donleaRecurrentCircuitryBalancing2018} (Donlea et al 2018)

    \item Helicon Cells Excite R2 Ring Neurons
    \parencite{donleaRecurrentCircuitryBalancing2018} (Donlea et al 2018)

    \item Helicon Cell Activation Induces Rebound Sleep
    \parencite{donleaRecurrentCircuitryBalancing2018} (Donlea et al 2018)

    \item These results demonstrate a sleep-promoting effect of inhibiting helicon cells, but they also suggest that helicon cells are only one of several dFB outputs used to induce sleep.
    \parencite{donleaRecurrentCircuitryBalancing2018} (Donlea et al 2018)

    \item The contours of an autoregulatory loop have thus emerged in which sleep-promoting dFB neurons communicate via helicon cells with R2 neurons, and the activity of these ring neurons is relayed back to dFB neurons (Figure 8). We imagine that, as sleep pressure builds during prolonged R2 neuron firing, activity-dependent plasticity (Liu et al., 2016) augments the excitatory drive to dFB neurons or instructs them to step up their intrinsic excitability. As a result, dFB neurons switch to the electrically active state and release inhibition. This pushes helicon cells into the hyperpolarized DOWN state (Figure 4), mutes their spiking, and deprives R2 neurons of a powerful source of excitation
    \parencite{donleaRecurrentCircuitryBalancing2018} (Donlea et al 2018)

    \item They include where precisely along the still unexplored R2dFB neuron interface sleep debt accrues, in what physical form it is stored, how its accumulation to threshold actuates the dFB switch, and how the accumulated sleep debt is cleared.
    \parencite{donleaRecurrentCircuitryBalancing2018} (Donlea et al 2018)

    % Raccuglia et al 2019
    \item We find that the power of these slowwave oscillations increases with sleep need and is subject to diurnal variation.
    \parencite{raccugliaNetworkSpecificSynchronizationElectrical2019} (Raccuglia et al 2019)
    
    \item Optical multi-unit voltage recordings reveal that single R5 neurons get synchronized by activating circadian input pathways.
    \parencite{raccugliaNetworkSpecificSynchronizationElectrical2019} (Raccuglia et al 2019)

    \item (R5) We show that this synchronization depends on NMDA receptor (NMDAR) coincidence detector function, and that an interplay of cholinergic and glutamatergic inputs regulates oscillatory frequency.
    Genetically targeting the coincidence detector function of NMDARs in R5, and thus the uncovered mechanism underlying synchronization, abolished network-specific compound slow-wave oscillations. It also disrupted sleep and facilitated light-induced wakening, establishing a role for slow-wave oscillations in regulating sleep and sensory gating.
    \parencite{raccugliaNetworkSpecificSynchronizationElectrical2019} (Raccuglia et al 2019)

    \item However, how specific neural networks contribute to generating compound oscillations and whether these oscillations represent a functional unit for sleep regulation largely remains unclear.
    \parencite{raccugliaNetworkSpecificSynchronizationElectrical2019} (Raccuglia et al 2019)

    \item However, in invertebrates, it is unknown whether electrical oscillations can gate specific behaviors and whether an electrophysiological sleep correlate, such as slow-wave oscillations, exists or is involved in sleep regulation. Local field potential (LFP) measurements in the Drosophila brain indicate that the frequency of large-scale compound neuronal activity is reduced during sleep [8-10], opening up the possibility that, comparable to vertebrates, slow oscillatory activity could be involved in mediating sleep.
    \parencite{raccugliaNetworkSpecificSynchronizationElectrical2019} (Raccuglia et al 2019)

    \item Here, we discover sleep-regulating, network-specific delta oscillations within the R5 [14-16] network of the Drosophila ellipsoid body, which is situated at a crossroad involved in sleep
    regulation [14, 17, 18] and sensory processing [15-20]
    \parencite{raccugliaNetworkSpecificSynchronizationElectrical2019} (Raccuglia et al 2019)

    \item (R5) Disrupting this synchronization and thus the emergence of compound delta oscillations affects the flies sleep patterns and alters sensory gating during sleep.
    \parencite{raccugliaNetworkSpecificSynchronizationElectrical2019} (Raccuglia et al 2019)

    \item Our work suggests that slow-wave oscillations and sleep could be fundamentally interconnected across phyla. Slow-wave oscillations may therefore potentially represent an evolutionarily conserved strategy for network mechanisms regulating internal states and sleep.
    \parencite{raccugliaNetworkSpecificSynchronizationElectrical2019} (Raccuglia et al 2019)

    \item We targeted expression of the GEVI ArcLight specifically to R5 (also sometimes referred to as R2) [14] neurons in the Drosophila brain. This defined network of 10–12 cells per hemisphere (Figure 1A) projects to the ellipsoid body and is involved in sleep regulation [14, 17, 18].
    \parencite{raccugliaNetworkSpecificSynchronizationElectrical2019} (Raccuglia et al 2019)

    \item In vivo recordings of the dendritic processes of R5 neurons (dorsal bulb; Figure 1A) identified electrical compound activity. Power analyses showed clear peaks around 1 Hz.
    \parencite{raccugliaNetworkSpecificSynchronizationElectrical2019} (Raccuglia et al 2019)

    \item Although, at early hours (ZT 0–3), no peak in the power spectrum was detected, delta oscillations around 1 Hz (integrated delta power at 0.5–1.5 Hz) became apparent during later hours of the day (ZT 8–12; Figures 1C–1E). Delta power peaked between ZT 13 and 16, which overlaps with the ZT that we measured as the animals mean onset of "consolidated" sleep
    \parencite{raccugliaNetworkSpecificSynchronizationElectrical2019} (Raccuglia et al 2019)

    \item Strikingly, delta power was increased after increasing the animals’ sleep pressure through sleep deprivation.
    \parencite{raccugliaNetworkSpecificSynchronizationElectrical2019} (Raccuglia et al 2019)

    \item Together, our data demonstrate that the power of delta oscillations is correlated to the animal’s behavioral onset of sleep while being subject to diurnal variation and homeostatic sleep regulation.
    \parencite{raccugliaNetworkSpecificSynchronizationElectrical2019} (Raccuglia et al 2019)

    \item Individual R5 neurons showed oscillatory activity with peak frequencies similar to the compound signal.
    Temporal correlation analysis between electrical patterns of simultaneously recorded R5 neurons showed that most depolarization phases occurred with a time lag of <50 ms (median = 13 ms; Figure 2G). Temporal overlap of single-unit activity could therefore be at the basis of the observed compound oscillations.
    \parencite{raccugliaNetworkSpecificSynchronizationElectrical2019} (Raccuglia et al 2019)

    \item We found that single-unit delta power was reduced in the morning. Moreover, the correlation between electrical patterns of single units was increased later in the day.
    \parencite{raccugliaNetworkSpecificSynchronizationElectrical2019} (Raccuglia et al 2019)

    \item NMDARs are coincidence detectors, and activation requires simultaneous ligand binding and membrane depolarization to remove Mg2+ ions blocking the channel pore.
    \parencite{raccugliaNetworkSpecificSynchronizationElectrical2019} (Raccuglia et al 2019)

    \item single-unit delta-band oscillations require network activity potentially generated by NMDAR-mediated signaling. Interestingly, sleep deprivation leads to an upregulation of NMDAR transcripts in R5 neurons.
    \parencite{raccugliaNetworkSpecificSynchronizationElectrical2019} (Raccuglia et al 2019)

    \item TuBu neurons convey sensory and circadian information, and functional connectivity has been demonstrated for subsets of ellipsoid body ring neurons.
    \parencite{raccugliaNetworkSpecificSynchronizationElectrical2019} (Raccuglia et al 2019)

    \item Activation of the TuBu neurons reinstated delta oscillations at high [Mg2+]e, and single units showed a reversible increase in delta power (Figures 3A and 3B).
    \parencite{raccugliaNetworkSpecificSynchronizationElectrical2019} (Raccuglia et al 2019)

    \item (TuBu activation) Importantly, we also observed a reduction of the time lag and an increased correlation of up-states between individual units (Figure 3C). This demonstrates that activity transmitted via TuBu neurons can synchronize R5 neurons.
    \parencite{raccugliaNetworkSpecificSynchronizationElectrical2019} (Raccuglia et al 2019)

    \item No delta activity was detected within the TuBu neurons (neither at the presynaptic compound level nor at the level of individual cell bodies; Figures S4A-S4D), suggesting that oscillations could be generated at the level of R5 neurons.
    \parencite{raccugliaNetworkSpecificSynchronizationElectrical2019} (Raccuglia et al 2019)

    \item removing NMDARs "from the equation" by applying APV completely prevented activation of R5 units (Figure S3C). Thus, NMDARs are required for oscillatory activity per se, and [Mg2+]e levels (and therefore likely NMDAR coincidence detection) were decisive as to whether the network stimulation would increase either single-unit oscillatory frequency or lead to multi-unit synchrony.
    \parencite{raccugliaNetworkSpecificSynchronizationElectrical2019} (Raccuglia et al 2019)

    \item acetylcholine can provide the coincident signal to "unblock" NMDARs required for synchronization of single units at the basis of delta-band oscillations.
    \parencite{raccugliaNetworkSpecificSynchronizationElectrical2019} (Raccuglia et al 2019)

    \item Our results indicate that NMDARs and, more specifically, the NMDAR Mg2+ block are crucially involved in generating R5-specific compound oscillations
    \parencite{raccugliaNetworkSpecificSynchronizationElectrical2019} (Raccuglia et al 2019)

    \item In vivo whole-cell patch-clamp recordings of single R5 neurons of control animals expressing non-mutated NMDAR subunit 1 showed rhythmic bursting at ZT 8–12, and power spectral analysis showed a clear peak around 1 Hz (Figures 5A–5C). This peak was not detected when analyzing recordings from animals expressing NMDARMg-/- in R5.
    \parencite{raccugliaNetworkSpecificSynchronizationElectrical2019} (Raccuglia et al 2019)

    \item ursts were unaltered (Figures 5D and 5E). Strikingly, in flies expressing NMDARMg-/-, inter-burst intervals did not follow regular patterns
    \parencite{raccugliaNetworkSpecificSynchronizationElectrical2019} (Raccuglia et al 2019)

    \item flies that expressed NMDARMg-/- in R5 neurons. Flies slept significantly less in total compared to controls, , and the number of sleep episodes was increased (Figure 6D) while sleep episode duration was decreased.
    Thus, flies no longer capable of multi-unit synchronization in R5 neurons woke up more frequently, and it took them longer to fall asleep.
    \parencite{raccugliaNetworkSpecificSynchronizationElectrical2019} (Raccuglia et al 2019)

    \item Expressing NMDARMg-/- in R5 significantly reduced rebound sleep after sleep deprivation
    \parencite{raccugliaNetworkSpecificSynchronizationElectrical2019} (Raccuglia et al 2019)

    \item Expressing NMDARMg-/-: the threshold for wakening was lower in the mutant and a significantly larger fraction of flies was wakened
    \parencite{raccugliaNetworkSpecificSynchronizationElectrical2019} (Raccuglia et al 2019)

    \item the mechanisms underlying compound delta oscillations in the R5 network not only regulate sleep drive but also the gating of stimulus-triggered wakening, potentially following an evolutionarily conserved strategy.
    \parencite{raccugliaNetworkSpecificSynchronizationElectrical2019} (Raccuglia et al 2019)

    \item In vertebrates, sleep and sleepiness are thought to be tightly interlinked with the synchronization of neuronal activity, resulting in increased compound slow-wave oscillations [2, 3]
    \parencite{raccugliaNetworkSpecificSynchronizationElectrical2019} (Raccuglia et al 2019)

    \item Our data suggest that compound delta oscillations specific to the sleep-regulating R5 network are generated by circadian drive transduced via TuBu neurons. We show that optogenetic activation of TuBu neurons increases single-unit power and synchronizes R5 neurons (Figure 3), which should result in an increase of compound delta power (Figure 1) and thus internal sleep drive.
    This is consistent with thermogenetic activation of TuBu neurons increasing the total amount of sleep in flies [15]. High levels of TuBu neuron output could be generated by altering activity in sleep-modulating DN1 circadian clock neurons [32], which form direct connections with the TuBu neurons [17].
    \parencite{raccugliaNetworkSpecificSynchronizationElectrical2019} (Raccuglia et al 2019)

    \item The R5 network also receives excitatory input from Helicon cells [18], a potential source of cholinergic input. We here provide evidence that nAChRs act as prime candidates to provide concurrent depolarization required for NMDAR coincidence detection in R5 neurons.
    \parencite{raccugliaNetworkSpecificSynchronizationElectrical2019} (Raccuglia et al 2019)

    \item Helicon cells also receive visual information and are part of a recurrent circuit mediating homeostatic sleep pressure regulation [18, 35]. Thus, R5 oscillatory activity is likely regulated via a complex interplay of sensory input [20], circadian rhythms [15, 17], and homeostatic sleep pressure regulation.
    \parencite{raccugliaNetworkSpecificSynchronizationElectrical2019} (Raccuglia et al 2019)

    \item Expression levels of NMDARs in R5 neurons have previously been associated with the regulation of sleep drive [14].
    \parencite{raccugliaNetworkSpecificSynchronizationElectrical2019} (Raccuglia et al 2019)

    \item Our data suggest that NMDAR coincidence detection gates neuronal synchronization of delta-wave activity within the R5 network to increase the power of sleep-relevant compound oscillations (Figures 3 and 5). Indeed, at the single-cell level, expressing Mg2+ block-deficient NMDARs in R5 neurons led to irregular activity patterns (Figure 5), which could be at the basis of impaired synchronization and disrupted compound oscillations.
    \parencite{raccugliaNetworkSpecificSynchronizationElectrical2019} (Raccuglia et al 2019)

    \item oscillatory activity in Drosophila R5 neurons is reminiscent of up- and down-states occurring at the level of mammalian cortical networks during deep sleep [36].
    We thus hypothesize that the oscillations observed here are comparable to sleep-regulating thalamocortical oscillations [37-39] as well as network-specific oscillations observed during sleep deprivation in vertebrates (local sleep) [2, 40, 41]. Thus, the R5 network could be functionally analogous to the thalamus, as network-specific synchronization of slowwave activity within the thalamus plays a crucial role in maintaining sleep [37-39] and sensory gating [42].
    \parencite{raccugliaNetworkSpecificSynchronizationElectrical2019} (Raccuglia et al 2019)

    \item whether cell-autonomous conductances contribute to sustained rhythmic activities of single R5 neuron remains an open question.
    \parencite{raccugliaNetworkSpecificSynchronizationElectrical2019} (Raccuglia et al 2019)

    % Manuscript
    \item Here, we describe a neural mechanism in Drosophila that creates neural filters
    that engender a brain state allowing for quiescent behavior by generating coherent slow-wave
    activity (SWA) between sleep-need- (R5)4 and locomotion-promoting neural networks.
    \parencite{raccugliaCoherentMultilevelNetwork2022} (Manuscript)

    \item coherent oscillations provide the mechanistic basis for a neural filter
    by temporally associating opposing signals resulting in reduced functional connectivity
    between locomotion-gating and navigational networks. We propose that the temporal
    pattern of SWA provides the structure to create a "breakable" filter, permitting the animal to
    enter a quiescent state, while providing the architecture for strong or salient stimuli to "break"
    the neural interaction, consequently allowing the animal to react.
    \parencite{raccugliaCoherentMultilevelNetwork2022} (Manuscript)

    \item  We recently identified that electrical SWA (0.5-
    1.5 Hz) in the R5 network of the Drosophila central complex is tied to undisrupted sleep, indicating that SWA is also a network-autonomous marker of sleep need across phyla
    \parencite{raccugliaCoherentMultilevelNetwork2022} (Manuscript)

    \item Apart from the R5
    network (with network here referring to interconnected R5 neurons with any in- and output
    connections) the Drosophila central complex comprises two additional prominent networks:
    that of the dorsal fan-shaped body (dFSB) and that of locomotion-promoting and visual
    signal-transducing helicon cells5.
    \parencite{raccugliaCoherentMultilevelNetwork2022} (Manuscript)

    \item We
    found that network activity of R5 and the dFSB showed increased synchronization at night compared to the day.
    we observed changes of electrical activity patterns in
    R5 and the dFSB between night and day, with an increase in power at night and also in the
    morning following sleep deprivation. (The plot shows only dFSB...)
    \parencite{raccugliaCoherentMultilevelNetwork2022} (Manuscript)

    \item Indeed, demonstrating that neither
    sensory stimuli nor VNC activity shaped dFSB SWA, we observed coherent nocturnal SWA ex
    vivo. Moreover, simultaneously acquired recordings of single dFSB
    neurons revealed low overall activity in the morning, while SWA markedly increased at night.
    However, activity was only loosely correlated between single cells4
    regardless of whether measured in the morning or at night, unlike
    what we previously observed for the R5 network. Therefore, changes in single-cell excitability
    of dFSB neurons might be the underlying source of compound oscillations, rather than the
    synchronization of single units.
    \parencite{raccugliaCoherentMultilevelNetwork2022} (Manuscript)

    \item  Optogenetic stimulation of R5 reliably induced or amplified dFSB presynaptic SWA
    following the end of our activation protocol (Extended Data Fig. 2a-c,g), indicating that
    synaptic output of R5 suffices to induce ongoing SWA.
    \parencite{raccugliaCoherentMultilevelNetwork2022} (Manuscript)

    \item The R5 stimulation-induced increase in
    power of the observed SWA in dFSB neurons was most pronounced at nighttime (Extended
    Data Fig. 2c), demonstrating the influence of the time of day on interactions between these
    neural circuits.
    \parencite{raccugliaCoherentMultilevelNetwork2022} (Manuscript)

    \item Optogenetic activation of the dFSB elicited pronounced SWA in
    R5 during stimulation at nighttime but none during the day.
    In a
    minority of cases, R5 SWA persisted even after optogenetic activation ceased (Extended Data
    Fig. 2i), suggesting that increased synaptic output as a result of increased activity of the dFSB
    at night can, in principle, entrain R5 oscillations and might thus have an effect on a longer time
    scale. 
    \parencite{raccugliaCoherentMultilevelNetwork2022} (Manuscript)

    \item optogenetically silencing the dFSB reduced oscillatory
    power of the R5 network (Extended Data Fig. 2k,l), further supporting the notion that
    increased activity within the neurons of the dFSB facilitates SWA in R5. 
    \parencite{raccugliaCoherentMultilevelNetwork2022} (Manuscript)

    \item Sometimes R5 SWA
    partially recovered after the start of optogenetic inhibition (Extended Data Fig. 2l), suggesting
    that other inputs to R5 or cell-autonomous processes stabilize R5 SWA at night.
    \parencite{raccugliaCoherentMultilevelNetwork2022} (Manuscript)

    \item prominent structures of the
    central complex exhibit similar, but not identical mutually interactive oscillations at night
    \parencite{raccugliaCoherentMultilevelNetwork2022} (Manuscript)

    \item At night, the dFSB increases activity to generate compound
    oscillations (Extended Data Fig. 1c). Importantly, these in turn switch on coherent oscillatory
    activity in the R5 network (Fig. 1).
    \parencite{raccugliaCoherentMultilevelNetwork2022} (Manuscript)

    \item In accordance with the
    locomotion of flies in this arena being subject to their sleep need, the mean velocity of sleep-
    deprived flies was significantly reduced compared to rested flies.
    \parencite{raccugliaCoherentMultilevelNetwork2022} (Manuscript)

    \item To mimic SWA in our behavioral assay, we stimulated R5 optogenetically at 1 Hz.
    We found that overall optogenetic stimulation of R5 reduced locomotor activity
    but flies still displayed short walking bouts as well as frequent
    grooming. Locomotion recovered within minutes after stimulation offset.
    This is in line with R5 activity entraining the involved neural networks to facilitate flies sleep4
    by potentially filtering out sensory stimuli.
    \parencite{raccugliaCoherentMultilevelNetwork2022} (Manuscript)

    \item Demonstrating that they could react to strong stimuli per se, flies
    reacted to an air puff during 1 Hz R5 optogenetic stimulation by increasing their walking velocity.
    These findings suggest that R5 activation at 1 Hz reduces locomotor activity
    while leaving the ability to walk and respond to strong stimuli. 
    \parencite{raccugliaCoherentMultilevelNetwork2022} (Manuscript)

    \item We therefore conclude that 1
    Hz R5 activation induces a behavioral state that we refer to as "quiescent", which is
    characterized by reduced locomotor activity, but also by periods of prolonged rest (that may
    include sleep, Extended Data Fig. 3k,l) and also grooming (see Supplemental Video 1).
    \parencite{raccugliaCoherentMultilevelNetwork2022} (Manuscript)

    \item Of note, stimulating R5 neurons at 0.1 Hz did not lead to a
    measurable impact on the flies’ behavior (Fig. 2i and Extended Data Fig. 3h). Therefore, the
    frequency or intensity of R5 activation mattered for inducing quiescence.
    \parencite{raccugliaCoherentMultilevelNetwork2022} (Manuscript)

    \item our analysis revealed that R5 neurons are highly interconnected with the helicon network that
    responds to visual stimuli. In addition, R5 neurons are one of the main pre- and
    postsynaptic partners of helicon cells.
    \parencite{raccugliaCoherentMultilevelNetwork2022} (Manuscript)

    \item Consistent with previous
    observations showing that dFSB neurons inhibit visually evoked responses in helicon cells5, we
    found that helicon cells are also connected to the dFSB
    \parencite{raccugliaCoherentMultilevelNetwork2022} (Manuscript)

    \item Simultaneous in vivo recordings of helicon and R5
    at their overlapping presynaptic sites confirmed that in both R5 and helicon networks SWA
    were observable, particularly at night.
    \parencite{raccugliaCoherentMultilevelNetwork2022} (Manuscript)

    \item We observed two states (Fig. 3c,d and Extended Data Fig. 5b-d): a synchronized state and a
    "shifted" state in which the electrical patterns of R5 were correlated with helicon activity, but
    preceding R5 activity by 50 to 200 ms
    \parencite{raccugliaCoherentMultilevelNetwork2022} (Manuscript)

    \item Turning back to ex vivo recordings, we only observed the synchronized state (Fig. 3g,h and
    Extended Data Fig. 5g), indicating that the shifted state could result from interferences with
    sensory input.
    \parencite{raccugliaCoherentMultilevelNetwork2022} (Manuscript)

    \item we first performed in vivo dual color
    voltage recordings during the midday. Correlation coefficients of R5 and helicon SWA were
    comparable to those of morning recordings. On the contrary, sleep deprivation and,
    therefore, high homeostatic sleep pressure, was sufficient to increase the correlation between
    R5 and helicon SWA (Fig. 3e) in the morning, without increasing oscillatory power in helicon
    (Extended Data Fig. 5c), which opens the possibility of homeostatic influence.
    \parencite{raccugliaCoherentMultilevelNetwork2022} (Manuscript)

    \item Strikingly,
    isolated homeostatic sleep drive (sleep depriving flies at night and activity measured in the
    following morning hours), led to animals only displaying the shifted and not the synchronized
    state (Extended Data Fig. 5b). Therefore, strong homeostatic drive appears to suffice for
    networks to enter the shifted state, but not to create the synchronized state.
    (Were the neurons bursting in the shifted state?)
    \parencite{raccugliaCoherentMultilevelNetwork2022} (Manuscript)

    \item In
    the absence of light, SWA profiles still showed an increased correlation at nighttime compared
    to morning hours (Fig. 3e), suggesting that, along with homeostatic, circadian drives could
    contribute to creating coherence between the networks’ electrical patterns.
    \parencite{raccugliaCoherentMultilevelNetwork2022} (Manuscript)

    \item Flies with a disrupted circadian clock that were kept in
    darkness no longer showed a statistically significant “nocturnal” increase in network
    coherence (Fig. 3e). Strikingly, however, control flies kept in darkness nearly exclusively showed synchronized
    profiles. As this differs from flies subject to light during the day (58.54%
    synchronized, n=41), this could be in line with an absence of sensory input facilitating the
    synchronized state.
    \parencite{raccugliaCoherentMultilevelNetwork2022} (Manuscript)

    \item  Taken together, the circadian clock and the absence of sensory
    input generate a nocturnal increase in network synchrony that could represent a closed filter
    state while visual input facilitates a shifted state that could represent open filter settings
    during quiescence.
    \parencite{raccugliaCoherentMultilevelNetwork2022} (Manuscript)

    \item However, as inhibitory pulse-coupled
    oscillators are known to usually stay out of sync27, we found that in order to simulate
    synchronization patterns similar to those experimentally observed, we required mixed
    excitatory and inhibitory output of R5 neurons.
    \parencite{raccugliaCoherentMultilevelNetwork2022} (Manuscript)

    \item In the model, we
    varied the relative strength of the inhibitory and excitatory synaptic coupling among R5
    neurons and observed that synchronization correlated with the strength of excitation
    compared to inhibition within the network
    \parencite{raccugliaCoherentMultilevelNetwork2022} (Manuscript)

    \item Strikingly, we identified several cholinergic R5 sites (Extended Data Fig. 6b), supporting an R5
    network with mixed inhibitory and excitatory output as source for the observed oscillations.
    \parencite{raccugliaCoherentMultilevelNetwork2022} (Manuscript)

    \item Helicon, as a purely excitatory recurrent network
    that is also connected to R5, was modeled using neurons that switch from a more depolarized
    and active state in the morning ("upstate") to a more hyperpolarized, and less active state
    ("downstate") at nighttime (Fig. 4a). Indeed, providing experimental support for our model,
    we found that single helicon cells were significantly more hyperpolarized at night compared
    to the morning (Extended Data Fig. 6d). Our simulation is in line with hyperpolarizing signals,
    potentially mediated by the dFSB5 and/or other sources, paving the path for R5-mediated
    entrainment of helicon activity at night (Fig. 4b,c) by rendering helicon less responsive to
    inhibitory (while still responsive to excitatory) input from R5 when helicons resting membrane
    potential is closer to the chloride reversal potential32. 
    \parencite{raccugliaCoherentMultilevelNetwork2022} (Manuscript)

    \item To further experimentally validate our model, we performed ex vivo helicon imaging
    experiments, while briefly optogenetically stimulating R5 at 1 Hz for 20 seconds during the
    day. We observed that even at this short time scale, transient, and sometimes persistent,
    entrainment worked per se (Extended Data Fig. 7g-i). Because R5 activation can also entrain
    dFSB activity during the day (Extended Data Fig. 2a-c), we suspect that this interaction would
    effectively set helicon cells to the downstate (night setting), allowing for entrainment of
    helicon by R5. Indeed, this is also in line with our behavior experiments, where, when
    stimulating R5 for a significantly longer period, we were able to induce behavioral quiescence
    that could even endure post stimulation (Fig. 2d-g).
    \parencite{raccugliaCoherentMultilevelNetwork2022} (Manuscript)

    \item "shifted state". We
    reasoned that a main difference between ex vivo and in vivo recordings would be the absence
    of sensory input or motor feedback (compare Fig. 3e, flies in DD). To simulate this, we applied
    additional input to helicon (see Supplementary information for details). Indeed, this
    additional input shifted the activity peaks of helicon and R5 networks by approximately 100
    ms (Extended Data Fig. 7f), closely resembling the experimentally observed "shifted state" (Fig.
    3c,d and Extended Data Fig. 5b). This is also in line with our experimental data, which show
    that the balance controls the degree of synchronization between excitatory and inhibitory
    drive and determines whether the networks are in the shifted or synchronized configuration.
    \parencite{raccugliaCoherentMultilevelNetwork2022} (Manuscript)

    \item While all genotypes
    showed slight arousal to red light, helicon stimulation led to strong arousal, whilst arousal
    during R5 stimulation was comparable to the control
    \parencite{raccugliaCoherentMultilevelNetwork2022} (Manuscript)

    \item Interestingly, compared to R5 activation alone, synchronous activation of R5 and helicon led
    to a somewhat stronger reduction of locomotor activity (Fig. 4g,h) and only a small fraction of
    flies was awakened by green light (Extended Data Fig. 8a-c). Indeed, activating R5 and helicon
    simultaneously even blocked air puff responses (Extended Data Fig. 8d-g), suggesting that R5
    and helicon synchronization can establish a strong neural filter that might also extend to other
    modalities
    \parencite{raccugliaCoherentMultilevelNetwork2022} (Manuscript)

    \item Together, these findings indicate that sleep drive mediated by R5 oscillations overrides
    gating of locomotion by helicon cells and further demonstrates that synchronization between
    these networks locks the locomotion-initiating networks into a state that, dependent on sleep
    need, reduces sensory processing and behavioral responsiveness.
    \parencite{raccugliaCoherentMultilevelNetwork2022} (Manuscript)

    \item  Our connectome analysis revealed that helicon cells
    as well as R5 neurons are connected to downstream EPG neurons (Fig. 5a), which represent
    the flys heading direction and initiate body turns towards or away from sensory stimuli and
    thus partake in reflecting the external world and controlling navigation
    \parencite{raccugliaCoherentMultilevelNetwork2022} (Manuscript)

    \item (Helicon and R5 antagonistically regulate downstream head direction neurons) Strikingly, we found that
    optogenetic R5 activation led to net hyperpolarization of EPG neurons, while activation of
    helicon cells led to strong depolarizations (Fig. 5f-h), indicating that R5 and helicon cells
    modulate EPG activity via antagonistic neurotransmitter inputs.
    \parencite{raccugliaCoherentMultilevelNetwork2022} (Manuscript)

    \item Depolarizing EPG neurons at 1 Hz, the frequency band of synchronized R5 and helicon, had no
    effect on locomotion (Extended Data Fig. 9a). In line with strong drive to EPG steering animal
    behavior away from quiescence, activating a set of EPG neurons at higher frequencies (10 Hz),
    approximating high excitatory input via helicon, altered locomotor activity (Fig. 5i,j and
    Extended Data Fig. 9b,c) along with turning behavior (Fig. 5j,k).
    \parencite{raccugliaCoherentMultilevelNetwork2022} (Manuscript)

    \item we directly tested the sensory filtering abilities of R5-based activity. optogenetic stimulation of
    R5 induces hyperpolarization of EPG in vivo. We next13
    widened the illumination to cover the brain and both eyes (Fig. 6b). In control flies, EPG
    neurons showed robust activation to visual input during the day (Fig. 6a,b). At night, however,
    visually-evoked depolarization was clearly attenuated. Strikingly, optogenetic activation of R5
    during the delivery of the visual input at daytime (Fig. 6c,d) attenuated visually-evoked
    responses in EPG, reminiscent of night-time recordings and confirming the visual filtering
    properties of R5.
    \parencite{raccugliaCoherentMultilevelNetwork2022} (Manuscript)

    \item Together, our study uncovers a mechanism in which circadian and homeostatic sleep need
    create coherent electrical activity across different networks at night (oscillating around 1 Hz,
    Fig. 1, 3). Facilitated through the dFSB (Fig. 1), the R5 network can overrule input of
    locomotion-promoting helicon to EPG (Fig. 4), by temporally associating helicon activity (Fig.
    3) and thus creating a neural filter (Fig. 6) to promote quiescent behavior (Fig. 2).
    \parencite{raccugliaCoherentMultilevelNetwork2022} (Manuscript)

    \item The occurrence of brain-wide SWA during deep sleep9,10 as well as network-specific SWA
    during local sleep in awake individuals39,40 suggests that coherent SWA could represent a
    neural filtering mechanism that regulates sensory processing and behavioral
    responsiveness
    \parencite{raccugliaCoherentMultilevelNetwork2022} (Manuscript)

    \item We here show
    a mechanism that creates SWA across networks involved in regulating sleep (e.g. homeostatic
    sleep regulation), setting up a sensory filter that attenuates visually evoked activity in
    downstream navigational neurons.
    \parencite{raccugliaCoherentMultilevelNetwork2022} (Manuscript)

    \item Therefore, sensory filtering and homeostatic sleep regulation can closely14
    interact, eventually allowing the animal to transfer from quiescent behaviors prior to sleep to
    sleep.
    \parencite{raccugliaCoherentMultilevelNetwork2022} (Manuscript)

    \item In the morning setting, the observed networks (dFSB-helicon-R5) act independent of each
    other, allowing gating of locomotion and updating of the head direction system (Fig. 6e). In
    the night setting, circadian and homeostatic regulation promote the entrainment of
    synchronized SWA between networks that opposingly regulate behavioral responses to visual
    stimuli. As a consequence, the functionally antagonistic inputs of neighboring helicon and R5
    synapses to the head direction system (EPG)6,7,38 cancel each other out, and set up a filter that
    reduces functional connectivity within the system to initiate a quiescent state that can allow
    for sleep need to transition into sleep.
    \parencite{raccugliaCoherentMultilevelNetwork2022} (Manuscript)

    \item SWA, comprised of oscillating up- (depolarized) and down-states (hyperpolarized)9, could
    follow a general architecture of breakable neural filters by providing a limited time window of
    transducing sensory information (during up-states)
    \parencite{raccugliaCoherentMultilevelNetwork2022} (Manuscript)

    \item Importantly, the synchronized state between R5 and helicon was only observed during night-
    time settings. Therefore, it is the interplay between diurnal/circadian factors and homeostatic
    regulation that allows for optimal regulation of sensory filtering depending on internal state
    and time of day.
    \parencite{raccugliaCoherentMultilevelNetwork2022} (Manuscript)

    \item Underlining this multi-layered regulation, artificial stimulation of the dFSB
    only induced SWA in R5 at night. This places the dFSB as a prime candidate to exert inhibitory
    drive to helicon5, as a prerequisite for coherence of R5 and helicon activity.
    \parencite{raccugliaCoherentMultilevelNetwork2022} (Manuscript)

    \item Underlining this multi-layered regulation, artificial stimulation of the dFSB
    only induced SWA in R5 at night. This places the dFSB as a prime candidate to exert inhibitory
    drive to helicon5, as a prerequisite for coherence of R5 and helicon activity
    \parencite{raccugliaCoherentMultilevelNetwork2022} (Manuscript)

    \item during SWA, our experiments suggest that flies can remain responsive to strong stimuli arguing that the filter can be
    bypassed or broken.
    \parencite{raccugliaCoherentMultilevelNetwork2022} (Manuscript)

    \item stimulating R5 and helicon in sync (representative of the
    synchronized configuration) showed to be more effective than stimulating R5 alone
    (representative of the synchronized or shifted configuration) in filtering out sensory stimuli
    and even extended the sensory modality of this neural filter, from a visual filter to also filtering
    mechanosensory input. Importantly, our experimental data and simulation suggest that the
    shifted state is facilitated by external sensory information.
    \parencite{raccugliaCoherentMultilevelNetwork2022} (Manuscript)

    \item Because SWA is associated with quiescence across phyla, we postulate that the neural
    interactions discovered here are not restricted to flies but could represent a general neural
    filtering mechanism43 that suppresses sensory processing to facilitate the shift from a world-
    driven to an internally-driven brain state
    \parencite{raccugliaCoherentMultilevelNetwork2022} (Manuscript)

\end{itemize}

\end{document}