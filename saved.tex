% \begin{itemize}
%     \item ... it is currently unclear how brain states arise that are able to dissociate an animal 
%     from its external world, allowing for quiescent behaviors, while retaining vigilance to salient 
%     sensory cues. Here, we describe a neural mechanism in Drosophila that creates neural filters 
%     that engender a brain state allowing for quiescent behavior by generating coherent slow-wave 
%     activity (SWA) between sleep-need- (R5) and locomotion-promoting neural networks5. 

%     \item These networks can regulate behavioral responsiveness by providing antagonistic inputs to downstream head
%     direction cells. Thus, coherent oscillations provide the mechanistic basis for a neural filter
%     by temporally associating opposing signals resulting in reduced functional connectivity
%     between locomotion-gating and navigational networks. We propose that the temporal
%     pattern of SWA provides the structure to create a ‘breakable’ filter, permitting the animal to
%     enter a quiescent state, while providing the architecture for strong or salient stimuli to ‘break’
%     the neural interaction, consequently allowing the animal to react.

%     \item Gating sensory information at thalamic relay stations is thought to be essential for maintaining
%     undisrupted deep sleep in mammals$^{2,8}$. In addition, network interactions generating
%     synchronized slow-wave activity (SWA, 0.5-4 Hz) have been observed during deep sleep

%     \item We recently identified that electrical SWA (0.5-1.5 Hz) in the R5 network of the
%     Drosophila central complex is tied to undisrupted sleep$^{3}$.

%     \item Apart from the R5 network$^{3,4}$ ... the Drosophila central complex comprises two
%     additional prominent networks: that of the dorsal fan-shaped body (dFSB)$,^{13,14}$ and
%     that of locomotion-promoting and visual signal-transducing helicon cells$^{5}$.

%     \item we set out to investigate whether the dFSB could
%     engage in shaping R5 SWA and thus establish an R5-based sensory filtering mechanism.    

%     \item ...network activity of R5 and the dFSB showed increased synchronization at night
%     (Fig. 1c, ~ 25\% overlap) compared to the day.

%     \item ... with an increase in power at night and also in the morning following sleep deprivation.
    
%     \item simultaneously acquired recordings of single dFSB
%     neurons revealed low overall activity in the morning, while SWA markedly increased at night
%     (Extended Data Fig. 1h-j).

%     \item wever, activity was only loosely correlated between single cells
%     regardless of whether measured in the morning or at night (Extended Data Fig. 1j), unlike
%     what we previously observed for the R5 network$^{3}$.
%     Therefore, changes in single-cell excitability
%     of dFSB neurons might be the underlying source of compound oscillations, rather than the
%     synchronization of single units.

%     \item Optogenetic stimulation of R5 reliably induced or amplified dFSB presynaptic SWA
%     following the end of our activation protocol

%     \item The R5 stimulation-induced increase in
%     power of the observed SWA in dFSB neurons was most pronounced at nighttime (Extended
%     Data Fig. 2c), demonstrating the influence of the time of day on interactions between these
%     neural circuits. 

%     \item optogenetically silencing the dFSB reduced oscillatory
%     power of the R5 network (Extended Data Fig. 2k,l), further supporting the notion that
%     increased activity within the neurons of the dFSB facilitates SWA in R5

%     \item At night, the dFSB increases activity to generate compound
%     oscillations (Extended Data Fig. 1c). Importantly, these in turn switch on coherent oscillatory
%     activity in the R5 network (Fig. 1)

%     \item the mean velocity of sleep-deprived flies was significantly reduced compared to rested flies
    
%     \item ...waking flies
%     up in our arena depends on the duration of the sensory stimulus, which is an important
%     hallmark of sleep1. In line with this, 5 s of green light did not affect locomotion in rested flies
%     (Extended Data Fig. 3c).

%     \item we stimulated R5 optogenetically at 1 Hz. 
    
%     \item We found that overall optogenetic stimulation of R5 reduced locomotor activity.
    
%     \item flies reacted to an air puff during 1 Hz R5 optogenetic stimulation by increasing their walking
%     velocity. These findings suggest that R5 activation at 1 Hz reduces locomotor activity
%     while leaving the ability to walk and respond to strong stimuli. We therefore conclude that 1
%     Hz R5 activation induces a behavioral state that we refer to as ‘quiescent’, which is
%     characterized by reduced locomotor activity, but also by periods of prolonged rest (that may
%     include sleep, Extended Data Fig. 3k,l) and also grooming (see Supplemental Video 1). These
%     markers for quiescence do not (or if at all, only partially) rely on external sensory stimuli but
%     are internally generated.

%     \item the frequency or intensity of R5 activation mattered for inducing quiescence (Fig. 2i and Extended
%     Data Fig. 3h)

%     \item R5 neurons are highly interconnected with the helicon network that
%     responds to visual stimuli$^{5}$.

%     \item Consistent with previous
%     observations showing that dFSB neurons inhibit visually evoked responses in helicon cells$^{5}$, we
%     found that helicon cells are also connected to the dFSB (Fig. 3a and Extended Data Fig. 4d,e)$^{22}$.

%     \item We next probed electrical in vivo network activity in helicon cells at night and found
%     spontaneous SWA .

%     \item Most interestingly, we found that electrical patterns of these neural
%     populations [Helicon \& R5] were significantly more correlated at night (Fig. 3d-e).

%     \item We observed two states (Fig. 3c,d and Extended Data Fig. 5b-d): a synchronized state and a
%     ‘shifted’ state in which the electrical patterns of R5 were correlated with helicon activity, but
%     preceding R5 activity by 50 to 200 ms 

%     \item Turning back to ex vivo recordings, we only observed the synchronized state (Fig. 3g,h and
%     Extended Data Fig. 5g), indicating that the shifted state could result from interferences with
%     sensory input.

%     \item 
% \end{itemize}