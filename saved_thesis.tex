\documentclass[12pt]{article}
\usepackage{template}

\addbibresource{references.bib}

\begin{document}

\section{My texts}

For example, in flies expression of the core clock genes \textit{tim} and \textit{per} are
downregulated by their own protein products \parencite{dubowyCircadianRhythmsSleep2017}.
Although it is not yet well understood how cycradian process modulates the transcriptome,
it has been demonstrated that neurons exhibit circadian rhythms in ion channel gene expressions
\parencite{doppSinglecellTranscriptomicsReveals2024,dubowyCircadianRhythmsSleep2017,andreaniCircadianProgrammingEllipsoid2022}.

\noindent\hrulefill

\section{From papers}

TuBu synchronizes R5 SWA and increases sleep, while not showing SWA itself \parencite{raccugliaNetworkSpecificSynchronizationElectrical2019,suarez-grimaltNeuralArchitectureSleep2021}

TuBu neurons convey sensory and circadian information \parencite{raccugliaNetworkSpecificSynchronizationElectrical2019}


Our data suggest that compound delta oscillations specific to the sleep-regulating R5
network are generated by circadian drive transduced via TuBu neurons. \parencite{raccugliaNetworkSpecificSynchronizationElectrical2019}


As our molecular clocks are not precisely running on a 24 h cycle, so-called clock neurons need light input to constantly reset the molecular
clocks (Dunlap 1999) \parencite{suarez-grimaltNeuralArchitectureSleep2021}
(Suarez et al 2021)

Optigenetic activation of DN1p clock neurons leads to oscillatory activity in ring neurons,
demonstraing the circadian influence of generating sleep need at the level of neural networks 
(Guo et al 2018) \parencite{suarez-grimaltNeuralArchitectureSleep2021}
(Suarez et al 2021)

Circadian time influences sensory processing. Secondly, the circadian time could determine
when and to what extent sensory processing leads to the accumulation of sleep need
or subjective tiredness \parencite{suarez-grimaltNeuralArchitectureSleep2021}
(Suarez et al 2021)

    
period (per) and timeless (tim) - expressed at higher levels in the early night
compared to the early day; cryptochrome (cry) and Clock (Clk) mRNA - the opposite.
Core clock genes are expressed and cycle specifically in Drosophila clock
neurons and glia. While Clk expression is restricted to clock neurons and glia,
other core clock genes per, tim and Cycle (Cyc) are expressed in more cell types.
No cell type expresses Clk without expression of other core circadian genes is
consistent with the notion that Clk is a circadian master regulator
\parencite{doppSinglecellTranscriptomicsReveals2024}.
(Dopp et al 2024)

Cell types involved in process S: the four annotated clusters with the highest
amount of sleep drive correlates were cell populations associated with sleep homeostasis.
121 correlates by dFB neurons. Similarly, Oct, Tyr and non-PAM DAN
neurons each had more than 100 sleep drive corelates. In contrast, the related
dopaminergic subtype of PAM neurons only had 14 correlates.
(Dopp et al 2024)


Flies with disrupted glial clock showed significantly reduced rebound sleep after SD compared to control flies
(Dopp et al 2024)

In Drosophila, the core molecular clock components are coexpressed only in a restricted
set of 150 neurons, which serve a function similar to the mammalian superchiasmatic nucleus
(SCN) in regulating circadian rhythms in behavioral activity.
\parencite{dubowyCircadianRhythmsSleep2017} (Dubowy and Sehgal 2017)

A molecular clock in a subset of DN1 is sufficient to drive morning anticipatory activity
in LD cycles, and, in certain temperature conditions, can drive evening anticipation as well 
(Y. Zhang et al. 2010).

Sleep homeostasis is often conceptualized as a continuous build-up of sleep need over
periods of wakefulness and dissipation over periods of sleep, such that the same mechanisms
should be invoked both when flies are spontaneously waking and during periods of forced 
wakefulness (sleep deprivation). However, recent work in Drosophila has called this view 
into question.
\parencite{dubowyCircadianRhythmsSleep2017} (Dubowy and Sehgal 2017)


One likely function of the circadian system is to suppress the onset of
sleep during times of the day when sleep pressure is high but when sleep would be
dangerous or maladaptive, thereby delaying it until the appropriate time.
For nearly four decades, the sleep field has conceptualized sleep through the
two-process model, which posits that sleep is governed by interactions between
homeostatic and circadian control processes.
In this model, sustained wakefulness produces a homeostatic sleep pressure
and increased slow-wave sleep, while a circadian system sets the thresholds
for sleep pressure that correspond to sleep or wakefulness.
\parencite{shaferRegulationDrosophilaSleep2021} (Shafer and Kenee 2021)


\color{red}
R5 neurons stimulate downstream neurons in the dorsal fan-shaped body (dFB), which are 
sufficient to produce sleep (Donlea et al., 2014; Donlea et al., 2011; Liu et al., 2016).
\parencite{andreaniCircadianProgrammingEllipsoid2022} (Andreani et al 2022)

R5 neurons promote sleep in response to deprivation by activating the sleep promoting
dFB (Liu et al., 2016).
\parencite{andreaniCircadianProgrammingEllipsoid2022} (Andreani et al 2022)

\color{black}

Flies with a disrupted circadian clock that were kept in
darkness no longer showed a statistically significant “nocturnal” increase in network
coherence (Fig. 3e). Strikingly, however, control flies kept in darkness nearly exclusively showed synchronized
profiles. As this differs from flies subject to light during the day (58.54%
synchronized, n=41), this could be in line with an absence of sensory input facilitating the
synchronized state.\parencite{raccugliaCoherentMultilevelNetwork2022} (Manuscript)


\end{document}